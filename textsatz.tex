%!TEX root = l2kurz.tex
% Siehe https://github.com/texdoc/l2kurz
 
\section{Setzen von Text}
 

\subsection{Deutschsprachige Texte}\label{deutsch}
\LaTeX{} wurde ursprünglich für den englischen Sprachraum entwickelt.
Für Texte, die in einer anderen Sprache als (amerikanischem)
Englisch verfasst sind, muss deshalb ein zusätzliches Paket 
(siehe Abschnitt~\ref{packages}) zur Sprachanpassung geladen werden.  
Für deutschsprachige Texte ist das normalerweise das Paket \texttt{babel} 
\begin{lstlisting}
 \usepackage[ngerman]{babel}
\end{lstlisting}
oder mit der Option \texttt{german} für traditionelle Rechtschreibung.
Der Grund für diese Unterscheidung ist die unterschiedliche Silbentrennung.
Eine ausführliche Beschreibung dieses Pakets findet man in \cite{babel}.



\subsection{Zeilen- und Seiten-Umbruch}

\subsubsection{Blocksatz}

\LaTeX{} setzt Text standardmäßig im Blocksatz, d.\,h.~mit Randausgleich,
wobei der Zeilen- und Seitenumbruch automatisch durchgeführt wird.
Dabei wird für jeden Absatz die
bestmögliche Aufteilung der Wörter auf die Zeilen bestimmt,
und wenn notwendig werden Wörter automatisch abgeteilt.
\begin{LTXexample}
Das Ende von Wörtern und
Sätzen wird durch Leerzeichen 
gekennzeichnet.
Hierbei spielt es keine Rolle,
ob man ein  oder           100
Leerzeichen eingibt.
 
Eine oder mehrere Leerzeilen
kennzeichnen das Ende von
Absätzen.
\end{LTXexample}

Üblicherweise werden in \LaTeX{} Absätze mit Einrückungen gekennzeichnet. 
Bei den Standardklassen kann dies mittels des Paketes \texttt{parskip}
deaktiviert werden und Absätze mit Abstand dazwischen kenntlich machen. 
Die sogenannten KOMA-Klassen bieten hierfür extra Optionen (siehe \cite{scrguide}).

Mit Hilfe der in Abschnitt~\ref{env} beschriebenen Umgebungen ist
es möglich, spezielle Textteile jeweils anders zu setzen.
 
Für Ausnahmefälle kann man den Umbruch außerdem mit den
folgenden Befehlen beeinflussen:
Der Befehl \lstinline|\\| oder \lstinline|\newline| bewirkt einen
Zeilenwechsel ohne neuen Absatz, der Befehl~\lstinline|\\*| einen
Zeilenwechsel, bei dem kein Seitenwechsel erfolgen darf.
Der Befehl \lstinline|\newpage| bewirkt einen Seitenwechsel.
Mit den Befehlen
\lstinline|\linebreak[|\textit{n}\lstinline|]|,
\lstinline|\nolinebreak[|\textit{n}\lstinline|]|,
\lstinline|\pagebreak[|\textit{n}\lstinline|]|   und
\lstinline|\nopagebreak[|\textit{n}\lstinline|]|
kann man angeben, ob an bestimmten Stellen ein Zeilen- bzw.\ %
Seitenwechsel eher günstig oder eher ungünstig ist, wobei
\textit{n} die Stärke der Beeinflussung angibt (1, 2, 3 oder 4).

Mit dem \LaTeX-Befehl \lstinline:\enlargethispage{:\textit{Länge}\lstinline:}:
lässt sich eine gegebene Seite um einen festen Betrag
verlängern oder verkürzen. Damit ist es möglich, noch
eine Zeile mehr auf eine Seite zu bekommen. 
(Zur Schreibweise von Längenangaben siehe Abschnitt~\ref{abst:horiz}.)
 
%PG: microtype erwähnt

\LaTeX\ bemüht sich, den Zeilenumbruch besonders schön zu machen.  Falls es
keine den strengen Regeln genügende Möglichkeit für einen glatten rechten Rand
findet, lässt es eine Zeile zu lang und gibt eine entsprechende Warnmeldung
aus (\texttt{over\-full hbox}). Das tritt insbesondere dann auf, wenn keine
geeignete Stelle für die Silbentrennung gefunden wird. Innerhalb der
\texttt{sloppypar}-Umgebung ist \LaTeX\ generell weniger streng in seinen
Ansprüchen und vermeidet solche überlange Zeilen, indem es die Wortabstände
stärker -- notfalls auch unschön~-- vergrößert. In diesem Fall werden zwar
Warnungen gemeldet (\texttt{under\-full hbox}), das Ergebnis ist aber meistens
durchaus brauchbar. Eine weitere Möglichkeit um \textit{schönere}  Absätze zu
erzeugen ist das Paket \texttt{microtype} für PDF\TeX{} und neuere Programme.
Es verändert einerseits die Breite der Buchstaben in einem so geringen Maß,
dass das dem Leser nicht auf"|fällt. Weiterhin erlaubt es gewisse Zeichen
etwas über den rechten Rand hinaus zu ragen (Trennstrich, Punkt, Komma,
\dots), so dass \LaTeX{} bessere Umbruchpunkte findet.


 
\subsubsection{Silbentrennung} \label{silb}
 
Falls die automatische Silbentrennung in einzelnen Fällen nicht
das richtige Ergebnis liefert, kann man diese Ausnahmen mit den
folgenden Befehlen richtigstellen.
 
Der Befehl \lstinline|\hyphenation| bewirkt, dass die darin
angeführten Wörter jedes Mal an den und nur an den mit
\lstinline|-| markierten Stellen abgeteilt werden können.
Er sollte im Vorspann stehen und eignet sich
\emph{nur} für Wörter, die keine indirekt kodierten Umlaute wie \lstinline|"a| enthalten.

\begin{example}
\hyphenation{ Eingabe-file
   Eingabe-files FORTRAN }
\end{example}

Der Befehl~\lstinline|\-| innerhalb eines Wortes bewirkt, dass dieses Wort
dieses eine Mal nur an den mit~\lstinline|\-| markierten Stellen  oder
unmittelbar nach einem Bindestrich abgeteilt werden kann. Mit dem Paket
\texttt{babel}\cite{babel} steht der Befehl~\lstinline:"-: zur
Verfügung, der auch Trennungen an anderen (nicht markierten) Stellen im Wort
erlaubt.

\begin{LTXexample}
Ein"-gabe"-file,
\LaTeX"=Eingabe"-file,
Häss"-lich"-keit
\end{LTXexample}

Der Befehl \lstinline|\mbox| bewirkt, dass das Argument überhaupt nicht
abgeteilt werden kann.

\begin{LTXexample}
Die Telefonnummer ist nicht mehr
\mbox{(02\,22) 56\,01-36\,94}. \\
\mbox{\textit{filename}} gibt den 
Dateinamen an.
\end{LTXexample}

Innerhalb des von \lstinline|\mbox| eingeschlossenen Text können
Wortabstände für den notwendigen Randausgleich bei
Blocksatz nicht mehr verändert werden.  Ist dies nicht
erwünscht, sollte man besser einzelne Wörter oder Wortteile
in \lstinline|\mbox| einschließen und diese mit einer Tilde~\lstinline|~|,
einem untrennbaren Wortzwischenraum (siehe
Abschnitt~\ref{abstaende}), verbinden.

% PG: c-k Trennungen etc. rausgenommen, nicht mehr notwendig


% Ich habe den ganzen Abschnitt über  nonfrenchspacing
% rausgenomen, da das mit ngerman/babel sowieso nicht der Fall ist. Und wer 
% es schafft, mit babel nonfrenchspacing anzuschalten, der kann sich den Rest
% auch anlesen. Es ist eine _kurz_anleitung


\subsection{Wortabstand} \label{abstaende}

In einigen Fällen kann man sich nicht auf die Automatik von \LaTeX{} verlassen: manchmal wird ein Umbruchpunkt eingefügt, an dem keiner sein soll und manchmal möchte man den Abstand zwischen zwei Wörtern (oder Buchstaben) verändern. Dafür werden unter anderem folgende Befehle bereit gestellt:

Eine \lstinline|~| (Tilde) bedeutet eine Leerstelle, an der kein Zeilenwechsel
erfolgen darf.

Mit \lstinline|\,| lässt sich ein kurzer Abstand erzeugen, wie er z.\,B.\ in
Abkürzungen vorkommt oder zwischen Zahlenwert und Maßeinheit.

 
\begin{LTXexample}
Das betrifft u.\,a.\ auch die \\
wissenschaftl.\ Mitarbeiter. \\
Dr.~Partl wohnt im 1.~Stock. \\
\dots\ 5\,cm breit.
\end{LTXexample}

% Abschnitt über frenchspacing raus. Das hier ist l2kurz!

\subsection{Spezielle Zeichen} \label{spezial}
 
\subsubsection{Anführungszeichen} \label{quotes}
 
Für Anführungszeichen ist \emph{nicht} das auf Schreibmaschinen
übliche Zeichen (\lstinline|"|) zu verwenden.
Im Buchdruck werden für öffnende und schließende
Anführungszeichen jeweils verschiedene Zeichen bzw.\ %
Zeichenkombinationen gesetzt.
Öffnende Anführungszeichen, wie sie im amerikanischen Englisch 
üblich sind, erhält man durch Eingabe von zwei Grave-Akzenten, 
schließende durch zwei Apostrophe.
\begin{LTXexample}
``No,'' he said,
``I don't know!''
\end{LTXexample}
"`Deutsche Gänsefüßchen"' sehen anders aus als ``amerikanische
Quotes''.  


Bei Benutzung des Paketes \texttt{babel} (siehe \ref{deutsch})
stehen die folgenden Befehle für 
deutsche Anführungszeichen zur Verfügung:
\lstinline|"`| (Doublequote und Grave-Akzent) für Anführungszeichen
unten,
und
\lstinline|"'| (Doublequote und Apostroph) für Anführungszeichen oben.
\exa
"`Nein,"' sagte er,
"`ich weiß nichts!"'
\exb
\begin{verbatim}
"`Nein,"' sagte er,
"`ich weiß nichts!"'
\end{verbatim}
\exc
In den Zeichensätzen mancher Rechner (z.\,B. Macintosh) sind die deutschen 
Anführungszeichen enthalten.  Das Paket \texttt{selinput} (siehe
Abschnitt~\ref{inputenc}) erlaubt dann, sie auch direkt einzugeben.

Das Paket \texttt{csquotes} erlaubt die indirekte Eingabe von Anführungszeichen, die automatisch korrekt geschachtelt werden. Zitate werden mit \lstinline|\enquote{|\textit{Text}\lstinline|}| ausgezeichnet:

\begin{LTXexample}
Er sagte \enquote{Da rief ich
\enquote{Hallo}}
\end{LTXexample}

Benutzt man durchgängig diese Form der Zitate, kann man mit einer Paketoption zu \texttt{csquotes} die Art der Anführungszeichen leicht für das gesamte Dokument verändern.

Eine gute Einführung in die Typographieregeln für Textsatz findet sich in der PDF"=Datei \emph{typokurz} von Christoph Bier\cite{typokurz}.

\subsubsection{Binde- und Gedankenstriche}
 
Im Schriftsatz werden unterschiedliche Striche für Bindestrich,
Gedankenstrich und Minus-Zeichen verwendet.
Die verschieden langen Striche werden in \LaTeX\ durch
Kombinationen von Minus-Zeichen angegeben. Der ganz lange
Gedankenstrich (\mbox{---}) wird im Deutschen nicht benutzt, im
Englischen wird er ohne Leerzeichen eingefügt.
\exa
O-Beine \\
10--18~Uhr \\
Paris--Dakar \\
Schalke 04 -- Hertha BSC \\
ja -- oder nein? \\
yes---or no? \\
0, 1 und $-1$
\exb
\begin{verbatim}
O-Beine
10--18~Uhr
Paris--Dakar
Schalke 04 -- Hertha BSC
ja -- oder nein?
yes---or no?
0, 1 und $-1$
\end{verbatim}
\exc
 
\subsubsection{Punkte}
 
Im Gegensatz zur Schreibmaschine, wo jeder Punkt und jedes Komma
mit einem der Buchstabenbreite entsprechenden Abstand versehen
ist, werden Punkte und Kommata im Buchdruck eng an das
vorangehende Zeichen gesetzt. Für Fortsetzungspunkte (drei
Punkte mit geeignetem Abstand) gibt es daher einen eigenen Befehl
\lstinline|\ldots| oder~\lstinline|\dots|.

\begin{LTXexample}
Nicht so ... sondern so: \\
Wien, Graz, \dots
\end{LTXexample}

 
\subsubsection{Ligaturen und Unterschneidungen}
 
Im Buchdruck ist es üblich, manche Buchstabenkombinationen
anders zu setzen als die Einzelbuchstaben.
\begin{beispiel}
{\large fi fl AV Te \dots}\quad
statt\quad {\large f\/i f\/l A\/V T\/e \dots}
\end{beispiel}
Mit Rücksicht auf die Lesbarkeit des Textes sollten
diese  Ligaturen und Unterschneidungen (kerning) 
unterdrückt werden, wenn die betreffenden Buchstabenkombinationen 
nach Vorsilben oder bei zusammengesetzten Wörtern zwischen den
Wortteilen auftreten.  Dazu dient der Befehl~\lstinline|\/|.

\begin{LTXexample}
Nicht Auflage (Au-fl-age) \\
sondern Auf\/lage (Auf-lage)
\end{LTXexample}

Mit dem Paket \texttt{babel} steht zusätzlich der  Befehl~\lstinline:"|: zur
Verfügung, der gleichzeitig eine Trennhilfe darstellt.

\begin{LTXexample}
Auf"|lage (Auf-lage)
\end{LTXexample}


Das Paket \texttt{babel}\cite{babel} macht noch einige weitere Befehle
verfügbar, die bestimmte Besonderheiten der deutschen Sprache
berücksichtigen.  Die wichtigsten von ihnen sind:
\lstinline|"~| für einen Bindestrich, an dem nicht umbrochen werden darf und  
\lstinline|"=| für einen Trennstrich, an dem ein Umbruch stattfinden darf, beispielsweise bei zusammengesetzten Hauptwörtern.


\begin{LTXexample}[firstline=2]
\obeylines
x"~beliebig
bergauf und "~ab
Breisgau"=Hochschwarzwald
\end{LTXexample}


\subsubsection{Symbole, Akzente und besondere Buchstaben}\label{symbole}

Einige der Zeichen, die bei der Eingabe eine Spezialbedeutung haben,
können durch das Voranstellen des
Zeichens \lstinline|\| (Backslash) ausgedruckt werden:
\begin{LTXexample}
\$ \& \% \# \_ \{ \}
\end{LTXexample}
Für andere gibt es besondere Befehle.  Sie gelten nur für normalen
Text; wie derartige Symbole innerhalb von mathematischen
Formeln gesetzt werden, erfahren Sie im Kapitel~\ref{math}:
\begin{LTXexample}[firstline=2]
\obeylines
\textasciitilde
\textasciicircum
\textbackslash 
\textbar  
\textless  
\textgreater
\end{LTXexample}

\LaTeX\ ermöglicht darüber hinaus die Verwendung von Akzenten 
und speziellen Buchstaben aus zahlreichen verschiedenen Sprachen, 
siehe die Tabellen~\ref{akzente}  und \ref{specials}.
Akzente werden darin jeweils am Beispiel
des Buchstabens~o gezeigt, können aber prinzipiell auf jeden
Buchstaben gesetzt werden.
Wenn ein Akzent auf ein i oder~j gesetzt werden soll, muss der
\mbox{i-Punkt} wegbleiben. Dies erreicht man mit den Befehlen
\lstinline|\i| und~\lstinline|\j|.
Es steht auch ein Befehl \lstinline|\textcircled| für 
eingekreiste Zeichen zur Verfügung.

\begin{LTXexample}
H\^otel, na\"\i ve, sm\o rebr\o d. \\
Die h\"assliche Stra\ss{}e.\\
!`Se\~norita!\\
\textcircled{x}
\end{LTXexample}


\begin{table}[tbp]
\caption{Akzente und spezielle Buchstaben} \label{akzente}
\centering
\begin{tabular}{@{}*{6}{l@{\quad}l@{\qquad}}@{}}
\a`o  &   \lstinline|\`o| & \a'o  & \lstinline|\'o| & \^o   &   \lstinline|\^o| &
\~o   &   \lstinline|\~o| & \a=o  & \lstinline|\=o| & \.o   &   \lstinline|\.o| \\
\u o  &   \lstinline|\u o| & \v o  & \lstinline|\v o| & \H o  &   \lstinline|\H o| &
\"o   &   \lstinline|\"o| & \c o  & \lstinline|\c o| & \d o  &   \lstinline|\d o| \\
\b o  &   \lstinline|\b o| & \r o  & \lstinline|\r o| & \t oo &   \lstinline|\t oo| \\[6pt]
\oe   &   \lstinline|\oe| & \OE   & \lstinline|\OE| & \ae   &   \lstinline|\ae| &
\AE   &   \lstinline|\AE| & \aa   & \lstinline|\aa| & \AA   &   \lstinline|\AA| \\
\o    &   \lstinline|\o| & \O    & \lstinline|\O| & \l    &   \lstinline|\l| &
\L    &   \lstinline|\L| & \i    & \lstinline|\i| & \j    &   \lstinline|\j| \\
\ss   &   \lstinline|\ss| \\
\end{tabular}

\end{table}
 
\begin{table}[tbp]
  \caption{Symbole} \label{specials}
   \begin{tabbing}
   \hspace{1cm}\=\hspace{3.15cm}\=  \hspace{1cm}\=\hspace{3.15cm}\=
   \hspace{1cm}\=\hspace{3.5cm}\=  \kill
!` \> \texttt{!{}`}      \> \dag \> \lstinline|\dag|            \> \texttrademark  \> \lstinline|\texttrademark|   \\         
?` \> \texttt{?{}`}      \> \ddag \> \lstinline|\ddag|          \> \textperiodcentered \> \lstinline|\textperiodcentered| \\ 
\S \> \lstinline|\S|          \> \P \> \lstinline|\P|                \> \textbullet    \> \lstinline|\textbullet| \\              
\pounds\> \lstinline|\pounds| \> \copyright \> \lstinline|\copyright|\>\textregistered  \> \lstinline|\textregistered| \\ 
   \end{tabbing}
\end{table}


Benutzt man das Paket \texttt{selinput} (siehe Abschnitt~\vref{inputenc}),
dann darf man diese Zeichen -- soweit sie im Zeichensatz des Betriebssystems
existieren -- auch direkt in das Eingabefile schreiben.

Mit dem Paket \texttt{babel} und der Option \texttt{ngerman} bzw \texttt{german}
können
Umlaute auch durch einfaches Voranstellen eines doppelten Anführungszeichen (\verb|"|) geschrieben werden, 
also z.\,B.\ \lstinline|"o| für~"`ö"';
für scharfes~s darf man \lstinline|"s| schreiben:

\begin{LTXexample}
Die h"assliche Stra"se
mu"s sch"oner werden.
\end{LTXexample}

Diese Notation wurde eingeführt, als die direkte Eingabe und
Anzeige von Umlauten auf vielen Rechnersystemen noch nicht möglich war.
Als Quasi-Standard zum plattformübergreifenden Austausch von 
\TeX- und \LaTeX"=Dokumenten ist sie aber nach wie vor nützlich.
 
\subsection{Kapitel und Überschriften}
 
Der Beginn eines Kapitels bzw.\ Unterkapitels und seine
Überschrift werden mit Befehlen der Form \lstinline|\section{...}|
angegeben. Dabei muss die logische Hierarchie eingehalten werden.

\pagebreak[3] %% Ansonsten sehr unschöner Seitenumbruch
\noindent Bei der Klasse \texttt{article}:
\begin{quote}
\lstinline|\part \section  \subsection  \subsubsection|
\end{quote}
Bei den Klassen \texttt{report} und \texttt{book}:
\begin{quote}
\lstinline|\part \chapter  \section  \subsection  \subsubsection|
\end{quote}
Artikel können also relativ einfach als Kapitel in ein Buch
eingebaut werden.  Die Abstände zwischen den Kapiteln, die
Nummerierung und die Schriftgröße der Überschrift werden von
\LaTeX\ automatisch bestimmt.

%\todo{MD: Sollte \texttt{\string\part} ergänzt werden?\\PG: ja}


Die Überschrift des gesamten Artikels bzw.\ die Titelseite des
Schriftstücks wird mit dem Befehl \lstinline|\maketitle| gesetzt.
Der Inhalt muss vorher mit den Befehlen \lstinline|\title|,
\lstinline|\author| und \lstinline|\date| vereinbart werden (vgl.\ 
Abbildung~\ref{dokument} auf Seite~\pageref{dokument}).

 
Der Befehl \lstinline|\tableofcontents| bewirkt, dass ein
Inhaltsverzeichnis ausgedruckt wird.
\LaTeX\ nimmt dafür immer die Überschriften und Seitennummern
von der jeweils letzten vorherigen Verarbeitung des Eingabefiles.
Bei einem neu erstellten oder um neue Kapitel erweiterten
Schriftstück muss man das Programms \LaTeX\ also mindestens
zweimal aufrufen, damit man die richtigen Angaben erhält.
 
Es gibt auch Befehle der Form \lstinline|\section*{...}|, bei denen
keine Nummerierung und keine Eintragung ins Inhaltsverzeichnis
erfolgen.

Mit den Befehlen \lstinline|\label| und~\lstinline|\ref| ist es möglich,
die von \LaTeX\ automatisch vergebenen Kapitelnummern im Text
anzusprechen.
Für \lstinline|\ref{...}| setzt \LaTeX\ die
mit \lstinline|\label{...}| definierte Nummer ein.
Auch hier wird immer die Nummer von der letzten vorherigen
Verarbeitung des Eingabefiles genommen.
Beispiel:
\begin{beispiel}
\begin{lstlisting}
\section{Algorithmen}
...
Der Beweis findet sich in Abschnitt~\ref{bew}.
...
\section{Beweise} \label{bew}
...
\end{lstlisting}
 \end{beispiel}
 
\subsection{Fußnoten}
 
Fußnoten\footnote{Das  ist eine Fußnote.} werden automatisch nummeriert und am
unteren Ende der Seite ausgedruckt.   Innerhalb von Gleitobjekten (siehe
Abschnitt~\ref{floats}),  Tabellen (\ref{tabular}) oder der
\texttt{tabbing}-Umgebung (\ref{tabbing}) ist der Befehl \lstinline|\footnote|
nicht erlaubt. Im \LaTeX{} Begleiter\cite{wonne} werden Möglichkeiten aufgezählt, diese
Einschränkung zu umgehen.

\begin{beispiel}
\begin{lstlisting}
Fußnoten\footnote{Das ist eine Fußnote.} werden \dots
\end{lstlisting}	
\end{beispiel}

 
\subsection{Hervorgehobener Text}
 
In maschinengeschriebenen Texten werden hervorzuhebende Texte
unterstrichen, im Buchdruck wird stattdessen ein auf"|fälliger
Schriftschnitt verwendet.
Der Befehl 
\begin{beispiel}
\lstinline|\emph{|\textit{text}\lstinline|}| 
\end{beispiel}
(emphasize) setzt seinen Parameter in einem auf"|fälligen Stil.
\LaTeX\ verwendet für den hervorgehobenen Text \textit{kursive}
Schrift.

\begin{LTXexample}
\emph{Werden innerhalb eines 
hervorgehobenen Textes 
\emph{nochmals} Passagen
hervorgehoben, so setzt
\LaTeX\ diese in einer 
\emph{aufrechten} Schrift.}
\end{LTXexample}



\subsection{Hochgestellter Text}
Hochgestellten Text in passender Größe generiert folgender Befehl:
\begin{quote}
\lstinline|\textsuperscript{|\textit{text}\lstinline|}|
\end{quote}

\begin{LTXexample}
le 2\textsuperscript{i\`eme}
r\`egime
\end{LTXexample}





\subsection{Umgebungen} \label{env}

Die Kennzeichnung von speziellen Textteilen, die anders als im
normalen Blocksatz gesetzt werden sollen, erfolgt mittels
sogenannter Umgebungen (environments) in der Form
\begin{quote}
\lstinline|\begin{|\textit{name}\lstinline|}|\quad
   \textit{text}\quad
   \lstinline|\end{|\textit{name}\lstinline|}|
\end{quote}
Umgebungen sind \emph{Gruppen}.
Sie können auch ineinander geschachtelt werden, dabei muss aber
die richtige Reihenfolge beachtet werden:
\begin{beispiel}
\begin{lstlisting}
\begin{aaa}
      \begin{bbb}
      ......
      \end{bbb} 
\end{aaa}
\end{lstlisting}	
\end{beispiel}


\subsubsection{Zitate (quote, quotation, verse)}
 
Die \texttt{quote}-Umgebung eignet sich für kürzere Zitate,
hervorgehobene Sätze und Beispiele.
Der Text wird links und rechts eingerückt.
%\exa
%Eine typographische Faustregel
%für die Zeilenlänge lautet:
%\begin{quote}
%
%Keine Zeile soll mehr als
%ca.\ 66~Buchstaben enthalten.
%\end{quote}
%Deswegen werden in Zeitungen
%mehrere Spalten nebeneinander
%verwendet.
%\exb 
\begin{LTXexample}
Eine typographische Faustregel
für die Zeilenlänge lautet:
\begin{quote}
Keine Zeile soll mehr als
ca.\ 66~Buchstaben enthalten.
\end{quote}
Deswegen werden in Zeitungen
mehrere Spalten nebeneinander
verwendet.
\end{LTXexample}

Die \texttt{quotation}-Umgebung unterscheidet sich in den
Standardklassen (vgl.\ Tabelle~\ref{docstyles} auf
Seite~\pageref{docstyles}) von der \texttt{quote}-Umgebung
dadurch, dass Absätze durch Einzüge gekennzeichnet werden.
Sie ist daher für längere Zitate, die aus mehreren Absätzen
bestehen, geeignet.

Die \texttt{verse}-Umgebung eignet sich für Gedichte und für
Beispiele, bei denen die Zeilenaufteilung wesentlich ist.  Die
Verse (Zeilen) werden durch~\lstinline|\\| getrennt, Strophen durch
Leerzeilen.


\subsubsection{Listen (itemize, enumerate, description)}
Die Umgebung \texttt{itemize} eignet sich für einfache Listen
(siehe Abbildung~\vref{item}).
Die Umgebung \texttt{enumerate} eignet sich für nummerierte
Aufzählungen (siehe Abbildung~\vref{enum}).
Die Umgebung \texttt{description} eignet sich für Beschreibungen
(siehe Abbildung~\vref{desc}). Mit dem Paket \texttt{enumitem} können die Umgebungen leicht den eigenen Bedürfnissen angepasst werden.

\begin{figure}[!htbp]
\begin{LTXexample}
Listen:
\begin{itemize}
 
\item Bei \texttt{itemize}
werden die Elemente ...
 
\item Listen kann man auch
verschachteln:
  \begin{itemize}
  \item Die maximale ...
  \item Bezeichnung und ...
  \end{itemize}
 
\item usw.
 
\end{itemize}
\end{LTXexample}
\caption{Beispiel für \texttt{itemize}} \label{item}
%\end{figure}
%
%
%\begin{figure}[!htbp]
\begin{LTXexample}
Nummerierte Listen:
\begin{enumerate}
 
\item Bei \texttt{enumerate}
werden die Elemente ...
 
\item Die Nummerierung ...
 
\item Listen kann man auch
verschachteln:
  \begin{enumerate}
  \item Die maximale ...
  \item Bezeichnung und ...
  \end{enumerate}
 
\item usw.
 
\end{enumerate}
\end{LTXexample}
\caption{Beispiel für \texttt{enumerate}} \label{enum}
\end{figure}
 
\begin{figure}[!htbp]  % <-------------           bang option of LaTeX2e 
\begin{LTXexample}
Kleine Tierkunde:
\begin{description}
\item[Gelse:]
   ein kleines Tier, das ...
\item[Gemse:]
   ein gro\ss es Tier, das ...
\item[Gürteltier:]
   ein mittelgro"s es Tier, das ...
\end{description}
\end{LTXexample}

\caption{Beispiel für \texttt{description}} \label{desc}
\end{figure}

 
\subsubsection
          [Flattersatz (flush\-left, flush\-right, center)]
          {Linksbündig, rechtsbündig, zentriert
                       (flush\-left, flush\-right, center)}

Die Umgebungen \texttt{Center}, \texttt{FlushLeft} und \texttt{FlushRight} aus
dem Paket \texttt{ragged2e} bewirken zentrierten, links"~, und rechtsbündigen
Satz. Die Varianten dieser Umgebungen (\texttt{center}, \texttt{flushleft} und
\texttt{flushright}), die ohne ein Zusatzpaket zur Verfügung stehen, bewirken
im Prinzip dasselbe, nur schaltet \LaTeX\ die Trennung fast vollständig aus.
Somit ergeben die letztgenannten Umgebungen einen sehr unruhigen Satz.

\begin{LTXexample}
% \usepackage{ragged2e}
\begin{FlushLeft}
Dies hier ist ein Blindtext zum
Testen von Textausgaben. Wer 
diesen Text liest, ist selbst 
schuld. Der Text gibt lediglich
den Grauwert der Schrift an.
\end{FlushLeft}
\end{LTXexample}


\subsubsection{Direkte Ausgabe (verbatim, verb)}
Zwischen \lstinline|\begin{verbatim}| und \lstinline|\end{verbatim}|
stehende Zeilen werden genauso ausgedruckt, wie sie eingegeben
wurden, d.\,h.\ mit allen Leerzeichen und Zeilenwechseln und ohne
Interpretation von Spezialzeichen und \LaTeX-Befehlen.  Dies
eignet sich z.\,B.\ für das Ausdrucken eines (kurzen)
Computer-Programms.

Innerhalb eines Absatzes können einzelne Zeichenkombinationen oder kurze
Textstücke ebenso "`wörtlich"' ausgedruckt werden, indem man sie zwischen
\lstinline.\verb|. und~\verb.|. einschließt. 

\begin{LTXexample}
Der \verb|\dots|-Befehl \dots
\end{LTXexample}
 
Die \texttt{verbatim}-Umgebung und der Befehl~\lstinline|\verb|
dürfen \emph{nicht} innerhalb von Parametern von anderen Befehlen
% und auch nicht innerhalb der \texttt{tabular}-Umgebung %% ??(br)
verwendet werden.


 
\subsubsection{Tabulatoren (tabbing)} \label{tabbing}

% PG: hehe, "Schreibmaschinen" - wer kennt die denn überhaupt noch?
In der \texttt{tabbing}-Umgebung kann man Tabulatoren ähnlich wie
an Schreibmaschinen setzen und verwenden.
Der Befehl~\lstinline|\=| setzt eine Tabulatorposition,
\lstinline|\kill| bedeutet, dass die "`Musterzeile"' nicht ausgedruckt werden
soll,
\lstinline|\>|~springt zur nächsten Tabulatorposition,
und \lstinline|\\| trennt die Zeilen. Entgegen der im nächsten Abschnitt 
vorgestellten Umgebungen \texttt{array} und \texttt{tabular} erlaubt
die \texttt{tabbing}-Umgebung einen Seitenumbruch.

%
%\exa
%\begin{tabbing}
%war einmal\quad \=
% Mittelteil\quad \= \kill
%links \> Mittelteil \> rechts\\
%Es \\
%war einmal \> und ist
% \> nicht mehr\\
%ein  \>  \> ausgestopfter\\
% \>  \> Teddybär
%\end{tabbing}
%\exb
\begin{LTXexample}
\begin{tabbing}
war einmal\quad \=
 Mittelteil\quad \= \kill
links \> Mittelteil \> rechts\\
Es \\
war einmal \> und ist
 \> nicht mehr\\
ein  \>  \> ausgestopfter\\
 \>  \> Teddybär
\end{tabbing}
\end{LTXexample}
%\exc


\subsubsection{Tabellen (tabular)} \label{tabular}
 
Tabellen lassen sich mit der \texttt{tabular}-Umgebung erzeugen. Dort kann man mit einer Tabellenpräambel bestimmen, wie die Spalten dargestellt werden. Die Spaltenbreite passt sich automatisch dem Inhalt der Tabelle an und muss nicht (außer bei p-Spalten) angegeben werden.
 
Im Parameter des Befehls \lstinline|\begin{tabular}{...}| wird das
Format der Tabelle angegeben.
Dabei bedeutet
\texttt{l}~eine Spalte mit linksbündigem Text,
\texttt{r}~eine mit rechtsbündigem,
\texttt{c}~eine mit zentriertem Text,
\lstinline|p{|\textit{breite}\lstinline|}| eine Spalte der angegebenen
Breite mit mehrzeiligem Text,
\lstinline.|.~einen senkrechten Strich.
 
Innerhalb der Tabelle bedeutet
\lstinline|&|~den Sprung in die nächste Tabellenspalte,
\lstinline|\\|~oder~\lstinline|\tabularnewline|~trennt die Zeilen,
\lstinline|\hline| (an Stelle einer Zeile) setzt einen waagrechten
Strich. Anstelle von \lstinline|\hline| bietet das Paket \texttt{booktabs} unterschiedliche Strichstärken, um den Tabellenkopf vom Tabellenkörper deutlicher zu trennen: \lstinline|\toprule|, \lstinline|\midrule|  und \lstinline|\bottomrule|. 

\begingroup
\def\arraystretch{1.1} 
\begin{LTXexample}
% \usepackage{booktabs}
\begin{tabular}[t]{rl}
\toprule
Wert & Zahlensystem \\
\midrule
7C0 & hexadezimal \\
3700 & oktal \\
11111000000 & binär \\
1984 & dezimal \\
\bottomrule
\end{tabular}
\end{LTXexample}
\endgroup
 
\endinput
