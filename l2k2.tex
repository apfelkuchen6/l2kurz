% master: l2kurz.tex
% L2K2.TEX - 2.Teil der LaTeX2e-Kurzbeschreibung v2.2
% L2K2.TEX - 2.Teil der LaTeX2e-Kurzbeschreibung Mainz 1994, 1995
% LK2.TEX  - 2.Teil der LaTeX-Kurzbeschreibung Graz-Wien 1987
% last changes: 2001-06-10 (WaS)
 
\section{Setzen von Text}
 

\subsection{Deutschsprachige Texte}\label{deutsch}
\LaTeX{} wurde ursprünglich für den englischen Sprachraum entwickelt.
Für Texte, die in einer anderen Sprache als (amerikanischem)
Englisch verfasst sind, muss deshalb ein zusätzliches Paket 
(siehe Abschnitt~\ref{packages}) zur Sprachanpassung geladen werden.  
Für deutschsprachige Texte ist das normalerweise das Paket \texttt{german} 
oder \texttt{ngerman}:
\begin{verse}
\verb:\usepackage{german}:
\end{verse}
oder
\begin{verse}
\verb:\usepackage{ngerman}:
\end{verse}
Für Texte in \emph{traditioneller} Rechtschreibung ist \texttt{german}
zu benutzen, für Texte in der \emph{neuen} deutschen Rechtschreibung
\texttt{ngerman}.
Der Grund für diese Unterscheidung ist die unterschiedliche Silbentrennung.
Eine ausführliche Beschreibung dieser Pakete findet man in \cite{germdoc}.  
Wenn im folgenden vom Paket \texttt{german} die Rede ist, 
so bezieht sich das normalerweise auch auf \texttt{ngerman}.


\subsection{Zeilen- und Seiten-Umbruch}

\subsubsection{Blocksatz}

Normaler Text wird im Blocksatz, d.\,h.~mit Randausgleich
gesetzt.  \LaTeX\ führt den Zeilen- und Seitenumbruch
automatisch durch.  Dabei wird für jeden Absatz die
bestmögliche Aufteilung der Wörter auf die Zeilen bestimmt,
und wenn notwendig werden Wörter automatisch abgeteilt.
\exa
\parindent=17pt\relax
\hbadness=4600\relax %% `underfull hbox'-Fehlermeldung aus
\noindent Das Ende von Wörtern und Sätzen wird durch Leerzeichen
gekennzeichnet.  Hierbei spielt es keine Rolle, ob man ein oder
100 Leerzeichen eingibt.
\par
Eine oder mehrere Leerzeilen
kennzeichnen das Ende von
Absätzen.
\exb
\begin{verbatim}
Das Ende von Wörtern und
Sätzen wird durch Leerzeichen 
gekennzeichnet.
Hierbei spielt es keine Rolle,
ob man ein  oder           100
Leerzeichen eingibt.
 
Eine oder mehrere Leerzeilen
kennzeichnen das Ende von
Absätzen.
\end{verbatim}
\exc
Wie die Absätze gesetzt werden, hängt von der Dokumentklasse ab: 
Die Klassen 
\texttt{article}, \texttt{report} und \texttt{book} kennzeichnen
Absätze durch Einrücken der ersten Zeile;
die Klasse \texttt{letter} beispielsweise lässt stattdessen 
zwischen den Absätzen einen kleinen vertikalen Abstand.

Mit Hilfe der in Abschnitt~\ref{env} beschriebenen Umgebungen ist
es möglich, spezielle Textteile jeweils anders zu setzen.
 
Für Ausnahmefälle kann man den Umbruch außerdem mit den
folgenden Befehlen beeinflussen:
Der Befehl \verb|\\| oder \verb|\newline| bewirkt einen
Zeilenwechsel ohne neuen Absatz, der Befehl~\verb|\\*| einen
Zeilenwechsel, bei dem kein Seitenwechsel erfolgen darf.
Der Befehl \verb|\newpage| bewirkt einen Seitenwechsel.
Mit den Befehlen
\verb|\linebreak[|\textit{n}\verb|]|,
\verb|\nolinebreak[|\textit{n}\verb|]|,
\verb|\pagebreak[|\textit{n}\verb|]|   und
\verb|\nopagebreak[|\textit{n}\verb|]|
kann man angeben, ob an bestimmten Stellen ein Zeilen- bzw.\ %
Seitenwechsel eher günstig oder eher ungünstig ist, wobei
\textit{n} die Stärke der Beeinflussung angibt (1, 2, 3 oder 4).

Mit dem \LaTeX-Befehl \verb:\enlargethispage{:\textit{Länge}\verb:}:
lässt sich eine gegebene Seite um einen festen Betrag
verlängern oder verkürzen. Damit ist es möglich, noch
eine Zeile mehr auf eine Seite zu bekommen. 
(Zur Schreibweise von Längenangaben siehe Abschnitt~\ref{abst:horiz}.)
 
\LaTeX\ bemüht sich, den Zeilenumbruch besonders schön zu
machen.  Falls es keine den strengen Regeln genügende
Möglichkeit für einen glatten rechten Rand findet, lässt es
eine Zeile zu lang und gibt eine entsprechende Fehlermeldung aus
(\texttt{over\-full hbox}).
Das tritt insbesondere dann auf, wenn keine geeignete Stelle
für die Silbentrennung gefunden wird.
Innerhalb der \texttt{sloppypar}-Umgebung ist \LaTeX\ generell
weniger streng in seinen Ansprüchen und vermeidet solche
überlange Zeilen, indem es die Wortabstände stärker --
notfalls auch unschön~-- vergrößert.
In diesem Fall werden zwar Warnungen gemeldet (\texttt{under\-full
hbox}), das Ergebnis ist aber meistens durchaus brauchbar.
 
 
\subsubsection{Silbentrennung} \label{silb}
 
Falls die automatische Silbentrennung in einzelnen Fällen nicht
das richtige Ergebnis liefert, kann man diese Ausnahmen mit den
folgenden Befehlen richtigstellen.
Das kann insbesondere bei zusammengesetzten oder fremdsprachigen
Wörtern notwendig werden; außerdem findet \LaTeX{} in Wörtern
mit Umlauten oder akzentuierten Buchstaben nicht alle zulässigen
Trennstellen.
 
Der Befehl \verb|\hyphenation| bewirkt, dass die darin
angeführten Wörter jedesmal an den und nur an den mit
\verb|-| markierten Stellen abgeteilt werden können.
Er sollte im Vorspann stehen und eignet sich
\emph{nur} für Wörter, die keine Umlaute, scharfes~s,
Ziffern oder sonstige Sonderzeichen enthalten.
\exa
~
\exb
\begin{verbatim}
\hyphenation{ Eingabe-file
   Eingabe-files FORTRAN }
\end{verbatim}
\exc
 
Der Befehl~\verb|\-| innerhalb eines Wortes bewirkt, dass dieses
Wort dieses eine Mal nur an den mit~\verb|\-|
markierten Stellen 
oder unmittelbar nach einem Bindestrich
abgeteilt werden kann.
Dieser Befehl eignet sich für \emph{alle} Wörter, auch für
solche, die Umlaute, scharfes~s, Ziffern oder sonstige
Sonderzeichen enthalten.
%(Mit dem Paket \texttt{german}(siehe \ref{deutsch}) steht eine
%weitere Möglichkeit zur Verfügung, nämlich der
%Befehl~\verb:"-:. Dieser erlaubt auch die Trennung an nicht
%explizit angegebenen Stellen im Wort.)
\exa
Eingabefile, \LaTeX-Eingabe-\\
file, Hässlichkeit
\exb
\begin{verbatim}
Ein\-gabe\-file,
\LaTeX-Eingabe\-file,
H"a"s\-lich\-keit
\end{verbatim}
\exc


Der Befehl \verb|\mbox{...}| bewirkt, dass das Argument überhaupt nicht
abgeteilt werden kann.
\exa
Die Telefonnummer ist nicht mehr
\mbox{(02\,22) 56\,01-36\,94}. \\
\mbox{\textit{filename}} gibt den 
Dateinamen an.
\exb 
\begin{verbatim}
Die Telefonnummer ist nicht mehr
\mbox{(02\,22) 56\,01-36\,94}. \\
\mbox{\textit{filename}} gibt den 
Dateinamen an.
\end{verbatim}
\exc
Innerhalb des von \verb|\mbox| eingeschlossenen Text können
Wortabstände für den notwendigen Randausgleich bei
Blocksatz nicht mehr verändert werden.  Ist dies nicht
erwünscht, sollte man besser einzelne Wörter oder Wortteile
in \verb|\mbox| einschließen und diese mit einer Tilde~\verb|~|,
einem untrennbaren Wortzwischenraum (siehe
Abschnitt~\ref{abstaende}), verbinden.


Das Paket \texttt{german} macht noch einige weitere Befehle
verfügbar, die bestimmte Besonderheiten der deutschen Sprache
berücksichtigen.  Die wichtigsten von ihnen sind:
\verb|"ck| für "`ck"', das als "`\mbox{k-k}"' abgeteilt wird,
\verb|"ff| für "`"ff"', das als "`\mbox{ff-f}"' abgeteilt wird
(und ebenso für andere Konsonanten)
und \verb|"~| für einen Bindestrich, an dem kein Zeilenumbruch
stattfinden~soll.
\exa
Drucker bzw. Druk-ker \\
Ro"lladen bzw. Roll-laden \\
x"~beliebig\\
bergauf und "~ab
\exb
\begin{verbatim}
Dru"cker
Ro"lladen
x"~beliebig
bergauf und "~ab
\end{verbatim}
\exc


\subsection{Wortabstand} \label{abstaende}
 
Um einen glatten rechten Rand zu erreichen, variiert \LaTeX\ die
Leerstellen zwischen den Wörtern etwas.
Nach Punkten, Fragezeichen u.\,a., die einen Satz beenden, wird
dabei ein etwas größerer Abstand erzeugt, was die Lesbarkeit
des Textes erhöht.
\LaTeX\ nimmt an, dass Punkte, die auf einen Großbuchstaben
folgen, eine Abkürzung bedeuten, und dass alle anderen Punkte
einen Satz beenden.
Ausnahmen von diesen Regeln muss man \LaTeX\ mit den folgenden
Befehlen mitteilen:

% Todo: bessere Beschreibung (schaltet nur den extra Leerraum nach einem Satzende bei \nonfrenchspacing aus)
Ein Backslash (\verb:\:) vor einem Leerzeichen bedeutet, dass diese
Leerstelle nicht verbreitert werden darf.

Eine \verb|~| (Tilde) bedeutet eine Leerstelle,
an der kein Zeilenwechsel erfolgen darf.

Mit \verb|\,| lässt sich ein kurzer Abstand erzeugen, wie er
z.\,B.\ in Abkürzungen vorkommt oder zwischen Zahlenwert und Maßeinheit.

Der Befehl~\verb|\@| vor einem Punkt bedeutet, dass dieser Punkt
einen Satz beendet, obwohl davor ein Großbuchstabe steht.
\exa
Das betrifft u.\,a.\ auch die \\
wissenschaftl.\ Mitarbeiter. \\
Dr.~Partl wohnt im 1.~Stock. \\
\dots\ 5\,cm breit. \\
Ich brauche Vitamin~C\@. Du nicht?
\exb
\begin{verbatim}
Das betrifft u.\,a.\ auch die \\
wissenschaftl.\ Mitarbeiter. \\
Dr.~Partl wohnt im 1.~Stock. \\
\dots\ 5\,cm breit. \\
Ich brauche Vitamin~C\@.
Du nicht?
\end{verbatim}
\exc
 
Außerdem gibt es die Möglichkeit, mit dem Befehl
\verb|\frenchspacing|
zu vereinbaren, dass die Abstände an Satzenden nicht anders
behandelt werden sollen als die zwischen Wörtern.
Diese Konvention ist im nicht-englischen Sprachraum verbreitet.
In diesem Fall brauchen die Befehle \verb|\ | und~\verb|\@| nicht
angegeben werden.
Mit dem Paket \texttt{german} ist \verb:\frenchspacing:
automatisch gewählt; das kann durch
\verb:\nonfrenchspacing:
wieder rückgängig gemacht werden -- so wie durchgängig im vorliegenden
Dokument!

 

\subsection{Spezielle Zeichen} \label{spezial}
 
\subsubsection{Anführungszeichen} \label{quotes}
 
Für Anführungszeichen ist \emph{nicht} das auf Schreibmaschinen
übliche Zeichen (\verb|"|) zu verwenden.
Im Buchdruck werden für öffnende und schließende
Anführungszeichen jeweils verschiedene Zeichen bzw.\ %
Zeichenkombinationen gesetzt.
öffnende Anführungszeichen, wie sie im amerikanischen Englisch 
üblich sind, erhält man durch Eingabe von zwei Grave-Akzenten, 
schließende durch zwei Apostrophe.
\exa
``No,'' he said,
``I don't know!''
\exb
\begin{verbatim}
``No,'' he said,
``I don't know!''
\end{verbatim}
\exc
"`Deutsche Gänsefüßchen"' sehen anders aus als ``amerikanische
Quotes''.  
% In Original-\LaTeX\ kann man versuchen, für deutsche
% Anführungszeichen unten (links) zwei Kommata und oben
% (rechts) zwei Grave-Akzente einzugeben, das Ergebnis ist aber
% nicht besonders schön.\footnote{Wenn die nach der Cork-Norm 
% kodierten Schriften verwendet werden, etwa mit Hilfe von 
% \texttt{\string\usepackage[T1]$\{$fontenc$\}$},
% ist die Eingabe durch zwei Kommata und zwei Graves möglich, die 
% Warnung bezüglich der Frage- und Ausrufezeichen bleibt aber richtig. 
% Deshalb sind Konventionen von \texttt{german.sty} auch dann zu 
% bevorzugen.}
% Statt \verb|!``| und \verb|?``| muss man \verb|!\/``| bzw.\ 
% \verb|?\/``| schreiben, weil man sonst die spanischen
% Sonderzeichen erhalten würde.
% \exa
% ,,Nein,`` sagte er,
% ,,ich wei\ss{} nichts!\/``
% \exb
% \begin{verbatim}
% ,,Nein,`` sagte er,
% ,,ich wei\ss{} nichts!\/``
% \end{verbatim}
% \exc
Bei Benutzung des Paketes \texttt{german} (siehe \ref{deutsch})
stehen die folgenden Befehle für 
deutsche Anführungszeichen zur Verfügung:
\verb|"`| (Doublequote und Grave-Akzent) für Anführungszeichen
unten,
und
\verb|"'| (Doublequote und Apostroph) für Anführungszeichen oben.
\exa
"`Nein,"' sagte er,
"`ich weiß nichts!"'
\exb
\begin{verbatim}
"`Nein,"' sagte er,
"`ich weiß nichts!"'
\end{verbatim}
\exc
In den Zeichensätzen mancher Rechner (z.\,B. Macintosh) sind die deutschen 
Anführungszeichen enthalten.  Das Paket \texttt{inputenc} (siehe
Abschnitt~\ref{inputenc}) erlaubt dann, sie auch direkt einzugeben.


\subsubsection{Binde- und Gedankenstriche}
 
Im Schriftsatz werden unterschiedliche Striche für Bindestrich,
Gedankenstrich und Minus-Zeichen verwendet.
Die verschieden langen Striche werden in \LaTeX\ durch
Kombinationen von Minus-Zeichen angegeben. Der ganz lange
Gedankenstrich (\mbox{---}) wird im Deutschen nicht benutzt, im
Englischen wird er ohne Leerzeichen eingefügt.
\exa
O-Beine \\
10--18~Uhr \\
Paris--Dakar \\
Schalke 04 -- Hertha BSC \\
ja -- oder nein? \\
yes---or no? \\
0, 1 und $-1$
\exb
\begin{verbatim}
O-Beine
10--18~Uhr
Paris--Dakar
Schalke 04 -- Hertha BSC
ja -- oder nein?
yes---or no?
0, 1 und $-1$
\end{verbatim}
\exc
 
\subsubsection{Punkte}
 
Im Gegensatz zur Schreibmaschine, wo jeder Punkt und jedes Komma
mit einem der Buchstabenbreite entsprechenden Abstand versehen
ist, werden Punkte und Kommata im Buchdruck eng an das
vorangehende Zeichen gesetzt. Für Fortsetzungspunkte (drei
Punkte mit geeignetem Abstand) gibt es daher einen eigenen Befehl
\verb|\ldots| oder~\verb|\dots|.
\exa
Nicht so ... sondern so: \\
Wien, Graz, \dots
\exb
\begin{verbatim}
Nicht so ... sondern so: \\
Wien, Graz, \dots
\end{verbatim}
\exc
 
\subsubsection{Ligaturen}
 
Im Buchdruck ist es üblich, manche Buchstabenkombinationen
anders zu setzen als die Einzelbuchstaben.
\begin{verse}
{\large fi fl AV Te \dots}\quad
statt\quad {\large f\/i f\/l A\/V T\/e \dots}
\end{verse}
Mit Rücksicht auf die Lesbarkeit des Textes sollten
diese  Ligaturen und Unterschneidungen (kerning) 
unterdrückt werden, wenn die betreffenden Buchstabenkombinationen 
nach Vorsilben oder bei zusammengesetzten Wörtern zwischen den
Wortteilen auftreten.  Dazu dient der Befehl~\verb|\/|.
\exa
Nicht Auflage (Au-fl-age) \\
sondern Auf\/lage (Auf-lage)
\exb
\begin{verbatim}
Nicht Auflage (Au-fl-age) \\
sondern Auf\/lage (Auf-lage)
\end{verbatim}
\exc
Mit dem Paket \texttt{german} steht zusätzlich der
Befehl~\verb:"|: zur Verfügung, der gleichzeitig eine
Trennhilfe darstellt.
\exa
Auf"|lage (Auf-lage)
\exb
\begin{verbatim}
Auf"|lage (Auf-lage)
\end{verbatim}
\exc

\subsubsection{Symbole, Akzente und besondere Buchstaben}\label{symbole}

Einige der Zeichen, die bei der Eingabe eine Spezialbedeutung haben,
können durch das Voranstellen des
Zeichens \verb|\| (Backslash) ausgedruckt werden:
\exa
\$ \& \% \# \_ \{ \}
\exb
\begin{verbatim}
\$ \& \% \# \_ \{ \}
\end{verbatim}
\exc
Für andere gibt es besondere Befehle.  Sie gelten nur für normalen
Text; wie derartige Symbole innerhalb von mathematischen
Formeln gesetzt werden, erfahren Sie im Kapitel~\ref{math}:
\exa
\textasciitilde \\
\textasciicircum \\
\textbackslash \\
\textbar \\ 
\textless\\
\textgreater
\exb
\begin{verbatim}
\textasciitilde
\textasciicircum
\textbackslash 
\textbar  
\textless  
\textgreater
\end{verbatim}
\exc

\LaTeX\ ermöglicht darüber hinaus die Verwendung von Akzenten 
und speziellen Buchstaben aus zahlreichen verschiedenen Sprachen, 
siehe die Tabellen~\ref{akzente}  und \ref{specials}.
Akzente werden darin jeweils am Beispiel
des Buchstabens~o gezeigt, können aber prinzipiell auf jeden
Buchstaben gesetzt werden.
Wenn ein Akzent auf ein i oder~j gesetzt werden soll, muss der
\mbox{i-Punkt} wegbleiben. Dies erreicht man mit den Befehlen
\verb|\i| und~\verb|\j|.
Es steht auch ein Befehl \verb|\textcircled| für 
eingekreiste Zeichen zur Verfügung.
\exa
\umlauthigh % <-------
Hôtel, na\"\i ve, smørebrød.\\[1\baselineskip]
\umlautlow
Die hässliche Straße.\\[1\baselineskip]
!`Señorita!\\
\textcircled{x}
\exb
\begin{verbatim}
H\^otel, na\"\i ve,
sm\o rebr\o d.
Die h\"assliche
Stra\ss{}e.
!`Se\~norita!
\textcircled{x}
\end{verbatim}
\exc

\begin{table}[tbp]
\caption{Akzente und spezielle Buchstaben} \label{akzente}
\begin{symbols}
\a`o  \>   \verb|\`o  | \> \a'o  \> \verb|\'o  | \> \^o   \>   \verb|\^o  | \\
\~o   \>   \verb|\~o  | \> \a=o  \> \verb|\=o  | \> \.o   \>   \verb|\.o  | \\
\u o  \>   \verb|\u o | \> \v o  \> \verb|\v o | \> \H o  \>   \verb|\H o | \\
\"o   \>   \verb|\"o  | \> \c o  \> \verb|\c o | \> \d o  \>   \verb|\d o | \\
\b o  \>   \verb|\b o | \> \r o  \> \verb|\r o | \> \t oo \>   \verb|\t oo| \\[6pt]
\oe   \>   \verb|\oe  | \> \OE   \> \verb|\OE  | \> \ae   \>   \verb|\ae  | \\
\AE   \>   \verb|\AE  | \> \aa   \> \verb|\aa  | \> \AA   \>   \verb|\AA  | \\
\o    \>   \verb|\o   | \> \O    \> \verb|\O   | \> \l    \>   \verb|\l   | \\
\L    \>   \verb|\L   | \> \i    \> \verb|\i   | \> \j    \>   \verb|\j   | \\
\ss   \>   \verb|\ss  | \\
\end{symbols}
\end{table}
 
\begin{table}[tbp]
  \caption{Symbole} \label{specials}
   \begin{tabbing}
   \hspace{1cm}\=\hspace{3.15cm}\=  \hspace{1cm}\=\hspace{3.15cm}\=
   \hspace{1cm}\=\hspace{3.5cm}\=  \kill
!` \> \texttt{!{}`}      \> \dag \> \verb|\dag|            \> \texttrademark  \> \verb|\texttrademark|   \\         
?` \> \texttt{?{}`}      \> \ddag \> \verb|\ddag|          \> \textperiodcentered \> \verb|\textperiodcentered| \\ 
\S \> \verb|\S|          \> \P \> \verb|\P|                \> \textbullet    \> \verb|\textbullet| \\              
\pounds\> \verb|\pounds| \> \copyright \> \verb|\copyright|\>\textregistered  \> \verb|\textregistered| \\ 
   \end{tabbing}
\end{table}


Benutzt man das Paket \texttt{inputenc} mit der passenden Option
für das jeweilige Betriebssystem, siehe Abschnitt~\ref{inputenc},
dann darf man diese Zeichen -- soweit sie im Zeichensatz des Betriebssystems
existieren -- auch direkt in das Eingabefile schreiben.

Mit dem Paket \texttt{german} %, siehe Abschnitt~\ref{deutsch}, 
können
Umlaute auch durch einfaches Voranstellen eines Doublequotes geschrieben werden, 
also z.\,B.\ \verb|"o| für~"`ö"';
für scharfes~s darf man \verb|"s| schreiben (ohne Probleme mit
nachfolgenden Leerstellen):
\exa
Die h"assliche Stra"se
muss schöner werden.
\exb
\begin{verbatim}
Die h"assliche Stra"se
mu"s sch"oner werden.
\end{verbatim}
\exc
Diese Notation wurde eingeführt, als die direkte Eingabe und
Anzeige von Umlauten auf vielen Rechnersystemen noch nicht möglich war.
Als Quasi-Standard zum plattformübergreifenden Austausch von 
\TeX- und \LaTeX"=Dokumenten ist sie aber nach wie vor nützlich und
für deutschsprachige Texte weit verbreitet; sie wird auch in einigen 
Beispielen dieser Kurzbeschreibung benutzt.
 


\subsection{Kapitel und Überschriften}
 
Der Beginn eines Kapitels bzw.\ Unterkapitels und seine
Überschrift werden mit Befehlen der Form \verb|\section{...}|
angegeben. Dabei muss die logische Hierarchie eingehalten werden.

\pagebreak[3] %% Ansonsten sehr unschöner Seitenumbruch
\noindent Bei der Klasse \texttt{article}:
\begin{quote}
\verb|\section  \subsection  \subsubsection|
\end{quote}
Bei den Klassen \texttt{report} und \texttt{book}:
\begin{quote}
\verb|\chapter  \section  \subsection  \subsubsection|
\end{quote}
Artikel können also relativ einfach als Kapitel in ein Buch
eingebaut werden.  Die Abstände zwischen den Kapiteln, die
Nummerierung und die Schriftgröße der Überschrift werden von
\LaTeX\ automatisch bestimmt.



Die Überschrift des gesamten Artikels bzw.\ die Titelseite des
Schriftstücks wird mit dem Befehl \verb|\maketitle| gesetzt.
Der Inhalt muss vorher mit den Befehlen \verb|\title|,
\verb|\author| und \verb|\date| vereinbart werden (vgl.\ 
Abbildung~\ref{dokument} auf Seite~\pageref{dokument}).

 
Der Befehl \verb|\tableofcontents| bewirkt, dass ein
Inhaltsverzeichnis ausgedruckt wird.
\LaTeX\ nimmt dafür immer die Überschriften und Seitennummern
von der jeweils letzten vorherigen Verarbeitung des Eingabefiles.
Bei einem neu erstellten oder um neue Kapitel erweiterten
Schriftstück muss man das Programms \LaTeX\ also mindestens
zweimal aufrufen, damit man die richtigen Angaben erhält.
 
Es gibt auch Befehle der Form \verb|\section*{...}|, bei denen
keine Nummerierung und keine Eintragung ins Inhaltsverzeichnis
erfolgen.

Mit den Befehlen \verb|\label| und~\verb|\ref| ist es möglich,
die von \LaTeX\ automatisch vergebenen Kapitelnummern im Text
anzusprechen.
Für \verb|\ref{...}| setzt \LaTeX\ die
mit \verb|\label{...}| definierte Nummer ein.
Auch hier wird immer die Nummer von der letzten vorherigen
Verarbeitung des Eingabefiles genommen.
Beispiel:
\begin{quote}
\begin{verbatim}
\section{Algorithmen}
...
Der Beweis findet sich in Abschnitt~\ref{bew}.
...
\section{Beweise} \label{bew}
...
\end{verbatim}
\end{quote}
 
 
\subsection{Fußnoten}
 
Fußnoten\footnote{Das 
ist eine Fußnote.} werden automatisch nummeriert
und am unteren Ende der Seite ausgedruckt.  
Innnerhalb von Gleitobjekten (siehe Abschnitt~\ref{floats}), 
Tabellen (\ref{tabular}) oder der \texttt{tabbing}-Umgebung (\ref{tabbing})
ist der Befehl \verb|\footnote| nicht erlaubt.
\exa
~
\exb
\begin{verbatim}
Fußnoten\footnote{Das ist eine
Fußnote.} werden \dots
\end{verbatim}
\exc
 
 
 
\subsection{Hervorgehobener Text}
 
In maschinengeschriebenen Texten werden hervorzuhebende Texte
unterstrichen, im Buchdruck wird stattdessen ein auf"|fälliger
Schriftschnitt verwendet.
Der Befehl 
\begin{quote}
\verb|\emph{|\textit{text}\verb|}| 
\end{quote}
(emphasize) setzt seinen Parameter in einem auf"|fälligen Stil.
\LaTeX\ verwendet für den hervorgehobenen Text \textit{kursive}
Schrift.

\exa 
\emph{Werden innerhalb eines hervorgehobenen Textes
\emph{nochmals} Passagen hervorgehoben, so setzt \LaTeX\ diese in
einer \emph{aufrechten} Schrift.}
\exb
\begin{verbatim}
\emph{Werden innerhalb eines 
hervorgehobenen Textes 
\emph{nochmals} Passagen
hervorgehoben, so setzt
\LaTeX\ diese in einer 
\emph{aufrechten} Schrift.}
\end{verbatim}
\exc


\subsection{Hochgestellter Text}
Hochgestellten Text in passender Größe generiert folgender Befehl:
\begin{quote}
\verb|\textsuperscript{|\textit{text}\verb|}|
\end{quote}
\exa
le 2\textsuperscript{i\`eme} r\'egime
\exb
\begin{verbatim}
le 2\textsuperscript{i\`eme}
r\'egime
\end{verbatim}
\exc




\subsection{Umgebungen} \label{env}

Die Kennzeichnung von speziellen Textteilen, die anders als im
normalen Blocksatz gesetzt werden sollen, erfolgt mittels
sogenannter Umgebungen (environments) in der Form
\begin{quote}
\verb|\begin{|\textit{name}\verb|}|\quad
   \textit{text}\quad
   \verb|\end{|\textit{name}\verb|}|
\end{quote}
Umgebungen sind \emph{Gruppen}.
Sie können auch ineinander geschachtelt werden, dabei muss aber
die richtige Reihenfolge beachtet werden:
\begin{quote}
\verb|\begin{aaa}|\\
\verb|  ... \begin{bbb} ... \end{bbb} ... |\\
\verb|\end{aaa}|
\end{quote}


\subsubsection{Zitate (quote, quotation, verse)}
 
Die \texttt{quote}-Umgebung eignet sich für kürzere Zitate,
hervorgehobene Sätze und Beispiele.
Der Text wird links und rechts eingerückt.
\exa
Eine typographische Faustregel
für die Zeilenlänge lautet:
\begin{quote}

Keine Zeile soll mehr als
ca.\ 66~Buchstaben enthalten.
\end{quote}
Deswegen werden in Zeitungen
mehrere Spalten nebeneinander
verwendet.
\exb 
\begin{verbatim}
Eine typographische Faustregel
für die Zeilenlänge lautet:
\begin{quote}
Keine Zeile soll mehr als
ca.\66~Buchstaben enthalten.
\end{quote}
Deswegen werden in Zeitungen
mehrere Spalten nebeneinander
verwendet.
\end{verbatim}
\exc

Die \texttt{quotation}-Umgebung unterscheidet sich in den
Standardklassen (vgl.\ Tabelle~\ref{docstyles} auf
Seite~\pageref{docstyles}) von der \texttt{quote}-Umgebung
dadurch, dass Absätze durch Einzüge gekennzeichnet werden.
Sie ist daher für längere Zitate, die aus mehreren Absätzen
bestehen, geeignet.

Die \texttt{verse}-Umgebung eignet sich für Gedichte und für
Beispiele, bei denen die Zeilenaufteilung wesentlich ist.  Die
Verse (Zeilen) werden durch~\verb|\\| getrennt, Strophen durch
Leerzeilen.


\subsubsection{Listen (itemize, enumerate, description)}
 
Die Umgebung \texttt{itemize} eignet sich für einfache Listen
(siehe Abbildung~\ref{item}).
Die Umgebung \texttt{enumerate} eignet sich für nummerierte
Aufzählungen (siehe Abbildung~\ref{enum}).
Die Umgebung \texttt{description} eignet sich für Beschreibungen
(siehe Abbildung~\ref{desc}).

\begin{figure}[!htbp]
\oben{\textwidth}
\exa
Listen:
\begin{itemize}
\item Bei \texttt{itemize} werden
die Elemente durch Punkte und andere Symbole gekennzeichnet. 
\item Listen kann man auch
verschachteln:
  \begin{itemize}
  \item Die maximale Schachtelungstiefe
  ist~4.
  \item
  Bezeichnung und Einrückung der Elemente
  wechseln automatisch.
  \end{itemize}
\item usw.
\end{itemize}
\exb
\begin{verbatim}
Listen:
\begin{itemize}
 
\item Bei \texttt{itemize}
werden die Elemente ...
 
\item Listen kann man auch
verschachteln:
  \begin{itemize}
  \item Die maximale ...
  \item Bezeichnung und ...
  \end{itemize}
 
\item usw.
 
\end{itemize}
\end{verbatim}
\exc
\unten
\caption{Beispiel für \texttt{itemize}} \label{item}
\end{figure}


\begin{figure}[!htbp]
\oben{\textwidth}
\exa
Nummerierte Listen:
\begin{enumerate}
\hbadness=4500\relax %% `underfull hbox'-Fehlermeldung aus
\item Bei \texttt{enumerate} werden
die Elemente mit Ziffern oder Buchstaben nummeriert.
\item Die Nummerierung erfolgt
automatisch.
\item Listen kann man auch
verschachteln:
  \begin{enumerate}
  \item Die maximale Schachtelungstiefe
  ist~4.
  \item Bezeichnung und Einrückung der Elemente
  wechseln automatisch.
  \end{enumerate}
\item usw.
\end{enumerate}
\exb
\begin{verbatim}
Nummerierte Listen:
\begin{enumerate}
 
\item Bei \texttt{enumerate}
werden die Elemente ...
 
\item Die Nummerierung ...
 
\item Listen kann man auch
verschachteln:
  \begin{enumerate}
  \item Die maximale ...
  \item Bezeichnung und ...
  \end{enumerate}
 
\item usw.
 
\end{enumerate}
\end{verbatim}
\exc
\unten
\caption{Beispiel für \texttt{enumerate}} \label{enum}
\end{figure}
 
\begin{figure}[!htbp]  % <-------------           bang option of LaTeX2e 
%\begin{figure}[htbp]
\oben{\textwidth}
\exa
Kleine Tierkunde:
\begin{description}
\item[Gelse:]
   ein kleines Tier, das
   östlich des Semmering Touristen verjagt.
\item[Gemse:]
   ein großes Tier, das
   westlich des Semmering von Touristen verjagt wird.
\item[Gürteltier:]
   ein mittelgroßes Tier, das
   hier nur wegen der Länge seines Namens vorkommt.
\end{description}
\exb
\begin{verbatim}
Kleine Tierkunde:
\begin{description}
\item[Gelse:]
   ein kleines Tier, das ...
\item[Gemse:]
   ein großes Tier, das ...
\item[Gürteltier:]
   ein mittelgroßes Tier, das ...
\end{description}
\end{verbatim}
\exc
\unten
\caption{Beispiel für \texttt{description}} \label{desc}
\end{figure}
 
{\hfuzz=2pt\relax % Overfull hbox (hyphen, therefore tolerable) 
                  % warning off
\subsubsection
          [Flattersatz (flush\-left, flush\-right, center)]
          {Linksbündig, rechtsbündig, zentriert
                       (flush\-left, flush\-right, center)}
}

Die Umgebungen \texttt{flushleft} und \texttt{flushright}
bewirken links- bzw.\ rechtsbündigen Satz ohne Randausgleich 
("`Flattersatz"') und ohne Trennungen, 
\texttt{center} setzt den Text in
die Mitte der Zeile.
Die einzelnen Zeilen werden durch~\verb|\\| getrennt.
Wenn man \verb|\\| nicht angibt, bestimmt \LaTeX\ automatisch die
Zeilenaufteilung% (siehe Abbildung~\ref{jandl}) auf Seite~\pageref{jandl})
.
%\begin{figure}[!ht]
%\oben{\textwidth} 
\exa
\begin{flushleft}
links \\
Backbord
\end{flushleft}
\exb
\begin{verbatim}
\begin{flushleft}
links \\
Backbord
\end{flushleft}
\end{verbatim}
\exc
\exa
\begin{flushright}
rechts  \\
Steuerbord
\end{flushright}
\exb
\begin{verbatim}
\begin{flushright}
rechts  \\
Steuerbord
\end{flushright}
\end{verbatim}
\exc
\exa
\begin{center}
Im \\ Reich \\ der \\ Mitte
\end{center}
\exb
\begin{verbatim}
\begin{center}
Im \\ Reich \\ der \\ Mitte
\end{center}
\end{verbatim}
\exc
%\unten
%\caption{Linksbündig, rechtsbündig und zentriert}
%\label{jandl}
%\end{figure} 


\subsubsection{Direkte Ausgabe (verbatim, verb)}
 
Zwischen \verb|\begin{verbatim}| und \verb|\end{verbatim}|
stehende Zeilen werden genauso ausgedruckt, wie sie eingegeben
wurden, d.\,h.\ mit allen Leerzeichen und Zeilenwechseln und ohne
Interpretation von Spezialzeichen und \LaTeX-Befehlen.  Dies
eignet sich z.\,B.\ für das Ausdrucken eines (kurzen)
Computer-Programms.

Innerhalb eines Absatzes können einzelne Zeichenkombinationen
oder kurze Textstücke ebenso "`wörtlich"' ausgedruckt
werden, indem man sie zwischen \verb.\verb|. und~\verb.|.
einschließt.
Mit diesen Befehlen wurden z.\,B.\ alle \LaTeX-Befehle in der
vorliegenden Beschreibung gesetzt.
\exa
Der \verb|\dots|-Befehl \dots
\exb
\begin{verbatim}
Der \verb|\dots|-Befehl \dots
\end{verbatim}
\exc
 
Die \texttt{verbatim}-Umgebung und der Befehl~\verb|\verb|
dürfen \emph{nicht} innerhalb von Parametern von anderen Befehlen
% und auch nicht innerhalb der \texttt{tabular}-Umgebung %% ??(br)
verwendet werden.


 
\subsubsection{Tabulatoren (tabbing)} \label{tabbing}
 
In der \texttt{tabbing}-Umgebung kann man Tabulatoren ähnlich wie
an Schreibmaschinen setzen und verwenden.
Der Befehl~\verb|\=| setzt eine Tabulatorposition,
\verb|\kill| bedeutet, dass die "`Musterzeile"' nicht ausgedruckt werden
soll,
\verb|\>|~springt zur nächsten Tabulatorposition,
und \verb|\\| trennt die Zeilen.
%
\par\vspace{0pt plus5\baselineskip}
\pagebreak[3]\vspace{0pt plus-5\baselineskip}
%
\exa
\begin{tabbing}
war einmal\quad \=
 Mittelteil\quad \= \kill
links \> Mittelteil \> rechts\\
Es \\
war einmal \> und ist
 \> nicht mehr\\
ein  \>  \> ausgestopfter\\
 \>  \> Teddybär
\end{tabbing}
\exb
\begin{verbatim}
\begin{tabbing}
war einmal\quad \=
 Mittelteil\quad \= \kill
links \> Mittelteil \> rechts\\
Es \\
war einmal \> und ist
 \> nicht mehr\\
ein  \>  \> ausgestopfter\\
 \>  \> Teddybär
\end{tabbing}
\end{verbatim}
\exc


\subsubsection{Tabellen (tabular)} \label{tabular}
 
Die \texttt{tabular}-Umgebung dient zum Setzen von Tabellen, bei
denen \LaTeX\ automatisch die benötigte Spaltenbreite
bestimmt, und bei der auch spezielle Eigenschaften wie
Rechtsbündigkeit und Hilfslinien vereinbart werden können.
 
Im Parameter des Befehls \verb|\begin{tabular}{...}| wird das
Format der Tabelle angegeben.
Dabei bedeutet
\texttt{l}~eine Spalte mit linksbündigem Text,
\texttt{r}~eine mit rechtsbündigem,
\texttt{c}~eine mit zentriertem Text,
\verb|p{|\textit{breite}\verb|}| eine Spalte der angegebenen
Breite mit mehrzeiligem Text,
\verb.|.~einen senkrechten Strich.
 
Innerhalb der Tabelle bedeutet
\verb|&|~den Sprung in die nächste Tabellenspalte,
\verb|\\|~trennt die Zeilen,
\verb|\hline| (an Stelle einer Zeile) setzt einen waagrechten
Strich.
 
          \vspace{0pt plus 1cm}
\exa
\begin{tabular}[t]{|rl|}
\hline
7C0 & hexadezimal \\
3700 & oktal \\
11111000000 & binär \\
\hline\hline
1984 & dezimal \\
\hline
\end{tabular}
\exb
\begin{verbatim}
\begin{tabular}{|rl|}
\hline
7C0 & hexadezimal \\
3700 & oktal \\
11111000000 & binär \\
\hline\hline
1984 & dezimal \\
\hline
\end{tabular}
\end{verbatim}
\exc
 
\endinput
