% master: l2kurz.tex
% L2K2.TEX - 2.Teil der LaTeX2e-Kurzbeschreibung v2.2
% L2K2.TEX - 2.Teil der LaTeX2e-Kurzbeschreibung Mainz 1994, 1995
% LK2.TEX  - 2.Teil der LaTeX-Kurzbeschreibung Graz-Wien 1987
% last changes: 2001-06-10 (WaS)
 
\section{Setzen von Text}
 

\subsection{Deutschsprachige Texte}\label{deutsch}
\LaTeX{} wurde urspr"unglich f"ur den englischen Sprachraum entwickelt.
F"ur Texte, die in einer anderen Sprache als (amerikanischem)
Englisch verfa"st sind, mu"s deshalb ein zus"atzliches Paket 
(siehe Abschnitt~\ref{packages}) zur Sprachanpassung geladen werden.  
F"ur deutschsprachige Texte ist das normalerweise das Paket \texttt{german} 
oder \texttt{ngerman}:
\begin{verse}
\verb:\usepackage{german}:
\end{verse}
oder
\begin{verse}
\verb:\usepackage{ngerman}:
\end{verse}
F"ur Texte in \emph{traditioneller} Rechtschreibung ist \texttt{german}
zu benutzen, f"ur Texte in der \emph{neuen} deutschen Rechtschreibung
\texttt{ngerman}.
Der Grund f"ur diese Unterscheidung ist die unterschiedliche Silbentrennung.
Eine ausf"uhrliche Beschreibung dieser Pakete findet man in \cite{germdoc}.  
Wenn im folgenden vom Paket \texttt{german} die Rede ist, 
so bezieht sich das normalerweise auch auf \texttt{ngerman}.


\subsection{Zeilen- und Seiten-Umbruch}
 
\newenvironment{specialparskip}{%
  \par
  \setlength{\parindent}{0pt}%
  \setlength{\parskip}{5pt plus 2pt minus 1pt}%
}{%
  \par
}
 
\subsubsection{Blocksatz}

Normaler Text wird im Blocksatz, d.\,h.~mit Randausgleich
gesetzt.  \LaTeX\ f"uhrt den Zeilen- und Seiten\-umbruch
automatisch durch.  Dabei wird f"ur jeden Absatz die
best"-m"ogliche Aufteilung der W"orter auf die Zeilen bestimmt,
und wenn notwendig werden W"orter automatisch abgeteilt.
\exa
\parindent=17pt\relax
\hbadness=4600\relax %% `underfull hbox'-Fehlermeldung aus
\noindent Das Ende von W"ortern und S"atzen wird durch Leerzeichen
gekennzeichnet.  Hierbei spielt es keine Rolle, ob man ein oder
100 Leerzeichen eingibt.
\par
Eine oder mehrere Leerzeilen
kennzeichnen das Ende von
Ab"-s"atzen.
\exb
\begin{alltt}
Das Ende von W\"ortern und
S\"atzen wird durch Leerzeichen 
gekennzeichnet.
Hierbei spielt es keine Rolle,
ob man ein  oder           100
Leerzeichen eingibt.
 
Eine oder mehrere Leerzeilen
kennzeichnen das Ende von
Abs\"atzen.
\end{alltt}
\exc
Wie die Ab"-s"atze gesetzt werden, h"angt von der Dokumentklasse ab: 
Die Klassen 
\texttt{article}, \texttt{report} und \texttt{book} kennzeichnen
Ab"-s"atze durch Ein"-r"u"cken der ersten Zeile;
die Klasse \texttt{letter} beispielsweise l"a"st stattdessen 
zwischen den Abs"atzen einen kleinen vertikalen Abstand.

Mit Hilfe der in Abschnitt~\ref{env} beschriebenen Umgebungen ist
es m"oglich, spezielle Textteile jeweils anders zu setzen.
 
F"ur Ausnahmef"alle kann man den Umbruch au"serdem mit den
folgenden Befehlen beeinflussen:
Der Befehl \verb|\\| oder \verb|\newline| bewirkt einen
Zeilenwechsel ohne neuen Absatz, der Befehl~\verb|\\*| einen
Zeilenwechsel, bei dem kein Seitenwechsel erfolgen darf.
Der Befehl \verb|\newpage| bewirkt einen Seitenwechsel.
Mit den Befehlen
\verb|\linebreak[|\textit{n}\verb|]|,
\verb|\nolinebreak[|\textit{n}\verb|]|,
\verb|\pagebreak[|\textit{n}\verb|]|   und
\verb|\nopagebreak[|\textit{n}\verb|]|
kann man angeben, ob an bestimmten Stellen ein Zeilen- bzw.\ %
Seitenwechsel eher g"unstig oder eher un"-g"unstig ist, wobei
\textit{n} die St"arke der Beeinflussung angibt (1, 2, 3 oder 4).

Mit dem \LaTeX-Befehl \verb:\enlargethispage{:\textit{L"ange}\verb:}:
l"a"st sich eine gegebene Seite um einen festen Betrag
ver"-l"angern oder ver"-k"urzen. Damit ist es m"og"-lich, noch
eine Zeile mehr auf eine Seite zu bekommen. 
(Zur Schreibweise von L"angenangaben siehe Abschnitt~\ref{abst:horiz}.)
 
\LaTeX\ bem"uht sich, den Zeilenumbruch besonders sch"on zu
machen.  Falls es keine den strengen Regeln ge"-n"ugende
M"og"-lichkeit f"ur einen glatten rechten Rand findet, l"a"st es
eine Zeile zu lang und gibt eine entsprechende Fehlermeldung aus
(\texttt{over\-full hbox}).
Das tritt insbesondere dann auf, wenn keine geeignete Stelle
f"ur die Silbentrennung gefunden wird.
Innerhalb der \texttt{sloppypar}-Umgebung ist \LaTeX\ generell
weniger streng in seinen Anspr"uchen und vermeidet solche
"uber"-lange Zeilen, indem es die Wort"-ab"-st"ande st"arker --
notfalls auch un\-sch"on~-- ver"-gr"o"sert.
In diesem Fall werden zwar Warnungen gemeldet (\texttt{under\-full
hbox}), das Ergebnis ist aber meistens durchaus brauchbar.
 
 
\subsubsection{Silbentrennung} \label{silb}
 
Falls die automatische Silbentrennung in einzelnen F"allen nicht
das richtige Ergebnis liefert, kann man diese Ausnahmen mit den
folgenden Befehlen richtigstellen.
Das kann insbesondere bei zusammengesetzten oder fremdsprachigen
W"ortern notwendig werden; au"serdem findet \LaTeX{} in W"ortern
mit Umlauten oder akzentuierten Buchstaben nicht alle zul"assigen
Trennstellen.
 
Der Befehl \verb|\hyphenation| bewirkt, da"s die darin
an"-ge"-f"uhrten W"orter jedesmal an den und nur an den mit
\verb|-| markierten Stellen abgeteilt werden k"onnen.
Er sollte im Vorspann stehen und eignet sich
\emph{nur} f"ur W"orter, die keine Umlaute, scharfes~s,
Ziffern oder sonstige Sonderzeichen enthalten.
\exa
~
\exb
\begin{verbatim}
\hyphenation{ Eingabe-file
   Eingabe-files FORTRAN }
\end{verbatim}
\exc
 
Der Befehl~\verb|\-| innerhalb eines Wortes bewirkt, da"s dieses
Wort dieses eine Mal nur an den mit~\verb|\-|
markierten Stellen 
oder unmittelbar nach einem Bindestrich
abgeteilt werden kann.
Dieser Befehl eignet sich f"ur \emph{alle} W"orter, auch f"ur
solche, die Umlaute, scharfes~s, Ziffern oder sonstige
Sonderzeichen enthalten.
%(Mit dem Paket \texttt{german}(siehe \ref{deutsch}) steht eine
%weitere M"og"-lichkeit zur Ver"-f"ugung, n"am"-lich der
%Befehl~\verb:"-:. Dieser erlaubt auch die Trennung an nicht
%explizit angegebenen Stellen im Wort.)
\exa
Ein\-gabe\-file, \LaTeX-Eingabe-\\
file, H"a"s\-lich\-keit
\exb
\begin{flushleft}\ttfamily
Ein\bs-gabe\bs-file,\\
\bs{}LaTeX-Eingabe\bs-file,\\
H"a"s\bs-lich\bs-keit
\end{flushleft}
\exc


Der Befehl \verb|\mbox{...}| bewirkt, da"s das Argument "uberhaupt nicht
abgeteilt werden kann.
\exa
Die Telefonnummer ist nicht mehr
\mbox{(02\,22) 56\,01-36\,94}. \\
\mbox{\textit{filename}} gibt den 
Dateinamen an.
\exb 
\begin{verbatim}
Die Telefonnummer ist nicht mehr
\mbox{(02\,22) 56\,01-36\,94}. \\
\mbox{\textit{filename}} gibt den 
Dateinamen an.
\end{verbatim}
\exc
Innerhalb des von \verb|\mbox| eingeschlossenen Text k"onnen
Wort"-ab"-st"ande f"ur den notwendigen Randausgleich bei
Blocksatz nicht mehr ver"-"andert werden.  Ist dies nicht
er"-w"unscht, sollte man besser einzelne W"orter oder Wort"-teile
in \verb|\mbox| einschlie"sen und diese mit einer Tilde~\verb|~|,
einem untrennbaren Wortzwischenraum (siehe
Abschnitt~\ref{abstaende}), verbinden.


Das Paket \texttt{german} macht noch einige weitere Befehle
ver"-f"ug"-bar, die bestimmte Besonderheiten der deutschen Sprache
ber"ucksichtigen.  Die wichtigsten von ihnen sind:
\verb|"ck| f"ur "`ck"', das als "`\mbox{k-k}"' abgeteilt wird,
\verb|"ff| f"ur "`"ff"', das als "`\mbox{ff-f}"' abgeteilt wird
(und ebenso f"ur andere Konsonanten)
und \verb|"~| f"ur einen Bindestrich, an dem kein Zeilenumbruch
stattfinden~soll.
\exa
Dru"cker bzw. Druk-ker \\
Ro"lladen bzw. Roll-laden \\
x"~beliebig\\
bergauf und "~ab
\exb
\begin{verbatim}
Dru"cker
Ro"lladen
x"~beliebig
bergauf und "~ab
\end{verbatim}
\exc


\subsection{Wortabstand} \label{abstaende}
 
Um einen glatten rechten Rand zu erreichen, variiert \LaTeX\ die
Leerstellen zwischen den W"ortern etwas.
Nach Punkten, Fragezeichen u.\,a., die einen Satz beenden, wird
dabei ein etwas gr"o"serer Abstand erzeugt, was die Lesbarkeit
des Textes er"-h"oht.
\LaTeX\ nimmt an, da"s Punkte, die auf einen Gro"sbuchstaben
folgen, eine Ab"-k"urzung bedeuten, und da"s alle anderen Punkte
einen Satz beenden.
Ausnahmen von diesen Regeln mu"s man \LaTeX\ mit den folgenden
Befehlen mitteilen:

Ein Backslash (\verb:\:) vor einem Leerzeichen bedeutet, da"s diese
Leerstelle nicht verbreitert werden darf.

Eine \verb|~| (Tilde) bedeutet eine Leerstelle,
an der kein Zeilenwechsel erfolgen darf.

Mit \verb|\,| l"a"st sich ein kurzer Abstand erzeugen, wie er
z.\,B.\ in Ab"-k"urzungen vorkommt oder zwischen Zahlenwert und Ma"seinheit.

Der Befehl~\verb|\@| vor einem Punkt bedeutet, da"s dieser Punkt
einen Satz beendet, obwohl davor ein Gro"sbuchstabe steht.
\exa
Das betrifft u.\,a.\ auch die \\
wissenschaftl.\ Mitarbeiter. \\
Dr.~Partl wohnt im 1.~Stock. \\
\dots\ 5\,cm breit. \\
Ich brauche Vitamin~C\@. Du nicht?
\exb
\begin{verbatim}
Das betrifft u.\,a.\ auch die \\
wissenschaftl.\ Mitarbeiter. \\
Dr.~Partl wohnt im 1.~Stock. \\
\dots\ 5\,cm breit. \\
Ich brauche Vitamin~C\@.
Du nicht?
\end{verbatim}
\exc
 
Au"serdem gibt es die M"oglichkeit, mit dem Befehl
\verb|\frenchspacing|
zu vereinbaren, da"s die Ab"-st"ande an Satz"-enden nicht anders
behandelt werden sollen als die zwischen W"or"-tern.
Diese Konvention ist im nicht-englischen Sprachraum verbreitet.
In diesem Fall brauchen die Befehle \verb|\ | und~\verb|\@| nicht
angegeben werden.
Mit dem Paket \texttt{german} ist \verb:\frenchspacing:
automatisch ge"-w"ahlt; das kann durch
\verb:\nonfrenchspacing:
wieder r"uckg"angig gemacht werden -- so wie durchg"angig im vorliegenden
Dokument!

 

\subsection{Spezielle Zeichen} \label{spezial}
 
\subsubsection{Anf"uhrungszeichen} \label{quotes}
 
F"ur Anf"uhrungszeichen ist \emph{nicht} das auf Schreibmaschinen
"ubliche Zeichen (\verb|"|) zu verwenden.
Im Buchdruck werden f"ur "offnende und schlie"sende
An"-f"uh"-rungs"-zeichen jeweils verschiedene Zeichen bzw.\ %
Zeichenkombinationen gesetzt.
"Offnende An"-f"uhrungs"-zeichen, wie sie im amerikanischen Englisch 
"ublich sind, er"-h"alt man durch Eingabe von zwei Grave-Akzenten, 
schlie"sende durch zwei Apostrophe.
\exa
``No,'' he said,
``I don't know!''
\exb
\begin{verbatim}
``No,'' he said,
``I don't know!''
\end{verbatim}
\exc
"`Deutsche G"ansef"u"schen"' sehen anders aus als ``amerikanische
Quotes''.  
% In Original-\LaTeX\ kann man versuchen, f"ur deutsche
% An"-f"uhrungs"-zeichen unten (links) zwei Kommata und oben
% (rechts) zwei Grave-Akzente einzugeben, das Ergebnis ist aber
% nicht besonders sch"on.\footnote{Wenn die nach der Cork-Norm 
% kodierten Schriften verwendet werden, etwa mit Hilfe von 
% \texttt{\string\usepackage[T1]$\{$fontenc$\}$},
% ist die Eingabe durch zwei Kommata und zwei Graves m"oglich, die 
% Warnung bez"uglich der Frage- und Ausrufezeichen bleibt aber richtig. 
% Deshalb sind Konventionen von \texttt{german.sty} auch dann zu 
% bevorzugen.}
% Statt \verb|!``| und \verb|?``| mu"s man \verb|!\/``| bzw.\ 
% \verb|?\/``| schreiben, weil man sonst die spanischen
% Sonderzeichen erhalten w"urde.
% \exa
% ,,Nein,`` sagte er,
% ,,ich wei\ss{} nichts!\/``
% \exb
% \begin{verbatim}
% ,,Nein,`` sagte er,
% ,,ich wei\ss{} nichts!\/``
% \end{verbatim}
% \exc
Bei Benutzung des Paketes \texttt{german} (siehe \ref{deutsch})
stehen die folgenden Befehle f"ur 
deutsche An"-f"uhrungs"-zeichen zur Ver"-f"ugung:
\verb|"`| (Doublequote und Grave-Akzent) f"ur An"-f"uhrungs"-zeichen
unten,
und
\verb|"'| (Doublequote und Apostroph) f"ur An"-f"uh"-rungszeichen oben.
\exa
"`Nein,"' sagte er,
"`ich wei"s nichts!"'
\exb
\begin{alltt}
"`Nein,"' sagte er,
"`ich wei\ss{} nichts!"'
\end{alltt}
\exc
In den Zeichens"atzen mancher Rechner (z.\,B. Macintosh) sind die deutschen 
Anf"uhrungszeichen enthalten.  Das Paket \texttt{inputenc} (siehe
Abschnitt~\ref{inputenc}) erlaubt dann, sie auch direkt einzugeben.


\subsubsection{Binde- und Gedankenstriche}
 
Im Schriftsatz werden unterschiedliche Striche f"ur Bindestrich,
Gedankenstrich und Minus-Zeichen verwendet.
Die verschieden langen Striche werden in \LaTeX\ durch
Kombinationen von Minus-Zeichen angegeben. Der ganz lange
Gedankenstrich (\mbox{---}) wird im Deutschen nicht benutzt, im
Englischen wird er ohne Leerzeichen ein"-ge"-f"ugt.
\exa
O-Beine \\
10--18~Uhr \\
Paris--Dakar \\
Schalke 04 -- Hertha BSC \\
ja -- oder nein? \\
yes---or no? \\
0, 1 und $-1$
\exb
\begin{alltt}
O-Beine
10--18~Uhr
Paris--Dakar
Schalke 04 -- Hertha BSC
ja -- oder nein?
yes---or no?
0, 1 und $-1$
\end{alltt}
\exc
 
\subsubsection{Punkte}
 
Im Gegensatz zur Schreibmaschine, wo jeder Punkt und jedes Komma
mit einem der Buchstabenbreite entsprechenden Abstand versehen
ist, werden Punkte und Kommata im Buchdruck eng an das
vorangehende Zeichen gesetzt. F"ur Fortsetzungspunkte (drei
Punkte mit geeignetem Abstand) gibt es daher einen eigenen Befehl
\verb|\ldots| oder~\verb|\dots|.
\exa
Nicht so ... sondern so: \\
Wien, Graz, \dots
\exb
\begin{verbatim}
Nicht so ... sondern so: \\
Wien, Graz, \dots
\end{verbatim}
\exc
 
\subsubsection{Ligaturen}
 
Im Buchdruck ist es "ublich, manche Buchstabenkombinationen
anders zu setzen als die Einzelbuchstaben.
\begin{verse}
{\large fi fl AV Te \dots}\quad
statt\quad {\large f\/i f\/l A\/V T\/e \dots}
\end{verse}
Mit R"ucksicht auf die Lesbarkeit des Textes sollten
diese  Ligaturen und Unterschneidungen (kerning) 
unterdr"uckt werden, wenn die betreffenden Buchstabenkombinationen 
nach Vorsilben oder bei zusammengesetzten W"ortern zwischen den
Wortteilen auftreten.  Dazu dient der Befehl~\verb|\/|.
\exa
Nicht Auflage (Au-fl-age) \\
sondern Auf\/lage (Auf-lage)
\exb
\begin{verbatim}
Nicht Auflage (Au-fl-age) \\
sondern Auf\/lage (Auf-lage)
\end{verbatim}
\exc
Mit dem Paket \texttt{german} steht zus"atzlich der
Befehl~\verb:"|: zur Ver"-f"ugung, der gleichzeitig eine
Trennhilfe darstellt.
\exa
Auf"|lage (Auf-lage)
\exb
\begin{verbatim}
Auf"|lage (Auf-lage)
\end{verbatim}
\exc

\subsubsection{Symbole, Akzente und besondere Buchstaben}\label{symbole}

Einige der Zeichen, die bei der Eingabe eine Spezialbedeutung haben,
k"onnen durch das Voranstellen des
Zeichens \verb|\| (Back\-slash) ausgedruckt werden:
\exa
\$ \& \% \# \_ \{ \}
\exb
\begin{verbatim}
\$ \& \% \# \_ \{ \}
\end{verbatim}
\exc
F"ur andere gibt es besondere Befehle.  Sie gelten nur f"ur normalen
Text; wie derartige Symbole innerhalb von mathematischen
Formeln gesetzt werden, erfahren Sie im Kapitel~\ref{math}:
\exa
\textasciitilde \\
\textasciicircum \\
\textbackslash \\
\textbar \\ 
\textless\\
\textgreater
\exb
\begin{verbatim}
\textasciitilde
\textasciicircum
\textbackslash 
\textbar  
\textless  
\textgreater
\end{verbatim}
\exc

\LaTeX\ erm"oglicht dar"uber hinaus die Verwendung von Akzenten 
und speziellen Buchstaben aus zahlreichen verschiedenen Sprachen, 
siehe die Tabellen~\ref{akzente}  und \ref{specials}.
Akzente werden darin jeweils am Beispiel
des Buchstabens~o gezeigt, k"onnen aber prinzipiell auf jeden
Buchstaben gesetzt werden.
Wenn ein Akzent auf ein i oder~j gesetzt werden soll, mu"s der
\mbox{i-Punkt} wegbleiben. Dies erreicht man mit den Befehlen
\verb|\i| und~\verb|\j|.
Es steht auch ein Befehl \verb|\textcircled| f"ur 
eingekreiste Zeichen zur Verf"ugung.
\exa
\umlauthigh % <-------
H\^otel, na\"\i ve, sm\o rebr\o d.\\[1\baselineskip]
\umlautlow
Die h\"a\ss{}liche Stra\ss{}e.\\[1\baselineskip]
!`Se\~norita!\\
\textcircled{x}
\exb
\begin{verbatim}
H\^otel, na\"\i ve,
sm\o rebr\o d.
Die h\"a\ss{}liche
Stra\ss{}e.
!`Se\~norita!
\textcircled{x}
\end{verbatim}
\exc

\begin{table}[tbp]
\caption{Akzente und spezielle Buchstaben} \label{akzente}
\begin{symbols}
\a`o  \>   \verb|\`o  | \> \a'o  \> \verb|\'o  | \> \^o   \>   \verb|\^o  | \\
\~o   \>   \verb|\~o  | \> \a=o  \> \verb|\=o  | \> \.o   \>   \verb|\.o  | \\
\u o  \>   \verb|\u o | \> \v o  \> \verb|\v o | \> \H o  \>   \verb|\H o | \\
\"o   \>   \verb|\"o  | \> \c o  \> \verb|\c o | \> \d o  \>   \verb|\d o | \\
\b o  \>   \verb|\b o | \> \r o  \> \verb|\r o | \> \t oo \>   \verb|\t oo| \\[6pt]
\oe   \>   \verb|\oe  | \> \OE   \> \verb|\OE  | \> \ae   \>   \verb|\ae  | \\
\AE   \>   \verb|\AE  | \> \aa   \> \verb|\aa  | \> \AA   \>   \verb|\AA  | \\
\o    \>   \verb|\o   | \> \O    \> \verb|\O   | \> \l    \>   \verb|\l   | \\
\L    \>   \verb|\L   | \> \i    \> \verb|\i   | \> \j    \>   \verb|\j   | \\
\ss   \>   \verb|\ss  | \\
\end{symbols}
\end{table}
 
\begin{table}[tbp]
  \caption{Symbole} \label{specials}
   \begin{tabbing}
   \hspace{1cm}\=\hspace{3.15cm}\=  \hspace{1cm}\=\hspace{3.15cm}\=
   \hspace{1cm}\=\hspace{3.5cm}\=  \kill
!` \> \texttt{!{}`}      \> \dag \> \verb|\dag|            \> \texttrademark  \> \verb|\texttrademark|   \\         
?` \> \texttt{?{}`}      \> \ddag \> \verb|\ddag|          \> \textperiodcentered \> \verb|\textperiodcentered| \\ 
\S \> \verb|\S|          \> \P \> \verb|\P|                \> \textbullet    \> \verb|\textbullet| \\              
\pounds\> \verb|\pounds| \> \copyright \> \verb|\copyright|\>\textregistered  \> \verb|\textregistered| \\ 
   \end{tabbing}
\end{table}


Benutzt man das Paket \texttt{inputenc} mit der passenden Option
f"ur das jeweilige Betriebssystem, siehe Abschnitt~\ref{inputenc},
dann darf man diese Zeichen -- soweit sie im Zeichensatz des Betriebssystems
existieren -- auch direkt in das Eingabefile schreiben.

Mit dem Paket \texttt{german} %, siehe Abschnitt~\ref{deutsch}, 
k"onnen
Umlaute auch durch einfaches Voranstellen eines Doublequotes geschrieben werden, 
also z.\,B.\ \verb|"o| f"ur~"`"o"';
f"ur scharfes~s darf man \verb|"s| schreiben (ohne Probleme mit
nachfolgenden Leerstellen):
\exa
\begin{sloppypar}\hbadness=1138 %% `underfull hbox'-Fehlermeldung aus 
Die h"a"sliche Stra"se
mu"s sch"oner werden.
\end{sloppypar}
\exb
\begin{verbatim}
Die h"a"sliche Stra"se
mu"s sch"oner werden.
\end{verbatim}
\exc
Diese Notation wurde eingef"uhrt, als die direkte Eingabe und
Anzeige von Umlauten auf vielen Rechnersystemen noch nicht m"oglich war.
Als Quasi-Standard zum plattform\-"uber\-greifenden Austausch von 
\TeX- und \LaTeX"=Dokumenten ist sie aber nach wie vor n"utzlich und
f"ur deutschsprachige Texte weit verbreitet; sie wird auch in einigen 
Beispielen dieser Kurzbeschreibung benutzt.
 


\subsection{Kapitel und "Uberschriften}
 
Der Beginn eines Kapitels bzw.\ Unterkapitels und seine
"Uberschrift werden mit Befehlen der Form \verb|\section{...}|
angegeben. Dabei mu"s die logische Hierarchie eingehalten werden.

\pagebreak[3] %% Ansonsten sehr unsch"oner Seitenumbruch
\noindent Bei der Klasse \texttt{article}:
\begin{quote}
\verb|\section  \subsection  \subsubsection|
\end{quote}
Bei den Klassen \texttt{report} und \texttt{book}:
\begin{quote}
\verb|\chapter  \section  \subsection  \subsubsection|
\end{quote}
Artikel k"onnen also relativ einfach als Kapitel in ein Buch
eingebaut werden.  Die Ab"-st"ande zwischen den Kapiteln, die
Numerierung und die Schrift"-gr"o"se der "Uberschrift werden von
\LaTeX\ automatisch bestimmt.



Die "Uberschrift des gesamten Artikels bzw.\ die Titelseite des
Schrift\-st"ucks wird mit dem Befehl \verb|\maketitle| gesetzt.
Der Inhalt mu"s vorher mit den Befehlen \verb|\title|,
\verb|\author| und \verb|\date| vereinbart werden (vgl.\ 
Abbildung~\ref{dokument} auf Seite~\pageref{dokument}).

 
Der Befehl \verb|\tableofcontents| bewirkt, da"s ein
Inhaltsverzeichnis ausgedruckt wird.
\LaTeX\ nimmt daf"ur immer die "Uberschriften und Seitennummern
von der jeweils letzten vorherigen Verarbeitung des Eingabefiles.
Bei einem neu erstellten oder um neue Kapitel erweiterten
Schrift"-st"uck mu"s man das Programms \LaTeX\ also mindestens
zweimal aufrufen, damit man die richtigen Angaben er"-h"alt.
 
Es gibt auch Befehle der Form \verb|\section*{...}|, bei denen
keine Numerierung und keine Eintragung ins Inhaltsverzeichnis
erfolgen.

Mit den Befehlen \verb|\label| und~\verb|\ref| ist es m"oglich,
die von \LaTeX\ automatisch vergebenen Kapitelnummern im Text
anzusprechen.
F"ur \verb|\ref{...}| setzt \LaTeX\ die
mit \verb|\label{...}| definierte Nummer ein.
Auch hier wird immer die Nummer von der letzten vorherigen
Verarbeitung des Eingabefiles genommen.
Beispiel:
\begin{quote}
\begin{verbatim}
\section{Algorithmen}
...
Der Beweis findet sich in Abschnitt~\ref{bew}.
...
\section{Beweise} \label{bew}
...
\end{verbatim}
\end{quote}
 
 
\subsection{Fu"snoten}
 
Fu"snoten\footnote{Das 
ist eine Fu"snote.} werden automatisch numeriert
und am unteren Ende der Seite ausgedruckt.  
Innnerhalb von Gleitobjekten (siehe Abschnitt~\ref{floats}), 
Tabellen (\ref{tabular}) oder der \texttt{tabbing}-Umgebung (\ref{tabbing})
ist der Befehl \verb|\footnote| nicht erlaubt.
\exa
~
\exb
\begin{alltt}
Fu\ss{}noten\verb+\footnote{Das ist eine+
Fu\ss{}note.\verb+}+ werden \dots
\end{alltt}
\exc
 
 
 
\subsection{Hervorgehobener Text}
 
In maschinengeschriebenen Texten werden hervorzuhebende Texte
unterstrichen, im Buchdruck wird stattdessen ein auf"|f"alliger
Schriftschnitt verwendet.
Der Befehl 
\begin{quote}
\verb|\emph{|\textit{text}\verb|}| 
\end{quote}
(emphasize) setzt seinen Parameter in einem auf"|f"alligen Stil.
\LaTeX\ verwendet f"ur den hervorgehobenen Text \textit{kursive}
Schrift.

\exa 
\emph{Werden innerhalb eines hervorgehobenen Textes
\emph{nochmals} Passagen hervorgehoben, so setzt \LaTeX\ diese in
einer \emph{aufrechten} Schrift.}
\exb
\begin{verbatim}
\emph{Werden innerhalb eines 
hervorgehobenen Textes 
\emph{nochmals} Passagen
hervorgehoben, so setzt
\LaTeX\ diese in einer 
\emph{aufrechten} Schrift.}
\end{verbatim}
\exc


\subsection{Hochgestellter Text}
Hochgestellten Text in passender Gr"o"se generiert folgender Befehl:
\begin{quote}
\verb|\textsuperscript{|\textit{text}\verb|}|
\end{quote}
\exa
le 2\textsuperscript{i\`eme} r\'egime
\exb
\begin{verbatim}
le 2\textsuperscript{i\`eme}
r\'egime
\end{verbatim}
\exc




\subsection{Umgebungen} \label{env}

Die Kennzeichnung von speziellen Textteilen, die anders als im
normalen Blocksatz gesetzt werden sollen, erfolgt mittels
sogenannter Umgebungen (environments) in der Form
\begin{quote}
\verb|\begin{|\textit{name}\verb|}|\quad
   \textit{text}\quad
   \verb|\end{|\textit{name}\verb|}|
\end{quote}
Umgebungen sind \emph{Gruppen}.
Sie k"onnen auch ineinander geschachtelt werden, dabei mu"s aber
die richtige Reihenfolge beachtet werden:
\begin{quote}
\verb|\begin{aaa}|\\
\verb|  ... \begin{bbb} ... \end{bbb} ... |\\
\verb|\end{aaa}|
\end{quote}


\subsubsection{Zitate (quote, quotation, verse)}
 
Die \texttt{quote}-Umgebung eignet sich f"ur k"urzere Zitate,
hervorgehobene S"atze und Beispiele.
Der Text wird links und rechts ein"-ge"-r"uckt.
\exa
Eine typographische Faustregel
f"ur die Zeilen"-l"ange lautet:
\begin{quote}
\hbadness=2500\relax %% `underfull hbox'-Fehlermeldung aus
Keine Zeile soll mehr als
ca.\ 66~Buchstaben enthalten.
\end{quote}
Deswegen werden in Zeitungen
mehrere Spalten nebeneinander
verwendet.
\exb 
\begin{alltt}
Eine typographische Faustregel
f\"ur die Zeilenl\"ange lautet:
\verb+\begin{quote}+
Keine Zeile soll mehr als
ca.\bs{} 66~Buchstaben enthalten.
\verb+\end{quote}+
Deswegen werden in Zeitungen
mehrere Spalten nebeneinander
verwendet.
\end{alltt}
\exc

Die \texttt{quotation}-Umgebung unterscheidet sich in den
Standardklassen (vgl.\ Tabelle~\ref{docstyles} auf
Seite~\pageref{docstyles}) von der \texttt{quote}-Umgebung
dadurch, da"s Ab"-s"atze durch Ein"-z"uge gekennzeichnet werden.
Sie ist daher f"ur l"angere Zitate, die aus mehreren Ab"-s"atzen
bestehen, geeignet.

Die \texttt{verse}-Umgebung eignet sich f"ur Gedichte und f"ur
Beispiele, bei denen die Zeilenaufteilung wesentlich ist.  Die
Verse (Zeilen) werden durch~\verb|\\| getrennt, Strophen durch
Leerzeilen.


\subsubsection{Listen (itemize, enumerate, description)}
 
Die Umgebung \texttt{itemize} eignet sich f"ur einfache Listen
(siehe Abbildung~\ref{item}).
Die Umgebung \texttt{enumerate} eignet sich f"ur numerierte
Auf"-z"ahlungen (siehe Abbildung~\ref{enum}).
Die Umgebung \texttt{description} eignet sich f"ur Beschreibungen
(siehe Abbildung~\ref{desc}).

\begin{figure}[!htbp]
\oben{\textwidth}
\exa
Listen:
\begin{itemize}
\item Bei \texttt{itemize} werden
die Elemente durch Punkte und andere Symbole gekennzeichnet. 
\item Listen kann man auch
verschachteln:
  \begin{itemize}
  \item Die maximale Schachtelungstiefe
  ist~4.
  \item
  Bezeichnung und Ein\-r"u"ckung der Elemente
  wechseln automatisch.
  \end{itemize}
\item usw.
\end{itemize}
\exb
\begin{verbatim}
Listen:
\begin{itemize}
 
\item Bei \texttt{itemize}
werden die Elemente ...
 
\item Listen kann man auch
verschachteln:
  \begin{itemize}
  \item Die maximale ...
  \item Bezeichnung und ...
  \end{itemize}
 
\item usw.
 
\end{itemize}
\end{verbatim}
\exc
\unten
\caption{Beispiel f"ur \texttt{itemize}} \label{item}
\end{figure}


\begin{figure}[!htbp]
\oben{\textwidth}
\exa
Numerierte Listen:
\begin{enumerate}
\hbadness=4500\relax %% `underfull hbox'-Fehlermeldung aus
\item Bei \texttt{enumerate} werden
die Elemente mit Ziffern oder Buchstaben numeriert.
\item Die Numerierung erfolgt
automatisch.
\item Listen kann man auch
verschachteln:
  \begin{enumerate}
  \item Die maximale Schachtelungstiefe
  ist~4.
  \item Bezeichnung und Ein\-r"u"ckung der Elemente
  wechseln automatisch.
  \end{enumerate}
\item usw.
\end{enumerate}
\exb
\begin{verbatim}
Numerierte Listen:
\begin{enumerate}
 
\item Bei \texttt{enumerate}
werden die Elemente ...
 
\item Die Numerierung ...
 
\item Listen kann man auch
verschachteln:
  \begin{enumerate}
  \item Die maximale ...
  \item Bezeichnung und ...
  \end{enumerate}
 
\item usw.
 
\end{enumerate}
\end{verbatim}
\exc
\unten
\caption{Beispiel f"ur \texttt{enumerate}} \label{enum}
\end{figure}
 
\begin{figure}[!htbp]  % <-------------           bang option of LaTeX2e 
%\begin{figure}[htbp]
\oben{\textwidth}
\exa
Kleine Tierkunde:
\begin{description}
\item[Gelse:]
   ein kleines Tier, das
   "ostlich des Semmering Touristen verjagt.
\item[Gemse:]
   ein gro"ses Tier, das
   westlich des Semmering von Touristen verjagt wird.
\item[G"urteltier:]
   ein mittelgro"ses Tier, das
   hier nur wegen der L"ange seines Namens vorkommt.
\end{description}
\exb
\begin{alltt}
Kleine Tierkunde:
\verb+\begin{description}+
\verb+\item[Gelse:]+
   ein kleines Tier, das ...
\verb+\item[Gemse:]+
   ein gro\ss{}es Tier, das ...
\verb+\item[+G\"urteltier:\verb+]+
   ein mittelgro\ss{}es Tier,
   das ...
\verb+\end{description}+
\end{alltt}
\exc
\unten
\caption{Beispiel f"ur \texttt{description}} \label{desc}
\end{figure}
 
{\hfuzz=2pt\relax % Overfull hbox (hyphen, therefore tolerable) 
                  % warning off
\subsubsection
          [Flattersatz (flush\-left, flush\-right, center)]
          {Linksb"undig, rechtsb"undig, zentriert
                       (flush\-left, flush\-right, center)}
}

Die Umgebungen \texttt{flushleft} und \texttt{flushright}
bewirken links- bzw.\ rechts\-b"undigen Satz ohne Randausgleich 
("`Flattersatz"') und ohne Trennungen, 
\texttt{center} setzt den Text in
die Mitte der Zeile.
Die einzelnen Zeilen werden durch~\verb|\\| getrennt.
Wenn man \verb|\\| nicht angibt, bestimmt \LaTeX\ automatisch die
Zeilenaufteilung% (siehe Abbildung~\ref{jandl}) auf Seite~\pageref{jandl})
.
%\begin{figure}[!ht]
%\oben{\textwidth} 
\exa
\begin{flushleft}
links \\
Backbord
\end{flushleft}
\exb
\begin{verbatim}
\begin{flushleft}
links \\
Backbord
\end{flushleft}
\end{verbatim}
\exc
\exa
\begin{flushright}
rechts  \\
Steuerbord
\end{flushright}
\exb
\begin{verbatim}
\begin{flushright}
rechts  \\
Steuerbord
\end{flushright}
\end{verbatim}
\exc
\exa
\begin{center}
Im \\ Reich \\ der \\ Mitte
\end{center}
\exb
\begin{verbatim}
\begin{center}
Im \\ Reich \\ der \\ Mitte
\end{center}
\end{verbatim}
\exc
%\unten
%\caption{Linksb"undig, rechtsb"undig und zentriert}
%\label{jandl}
%\end{figure} 


\subsubsection{Direkte Ausgabe (verbatim, verb)}
 
Zwischen \verb|\begin{verbatim}| und \verb|\end{verbatim}|
stehende Zeilen werden genauso ausgedruckt, wie sie eingegeben
wurden, d.\,h.\ mit allen Leerzeichen und Zeilenwechseln und ohne
Interpretation von Spezialzeichen und \LaTeX-Befehlen.  Dies
eignet sich z.\,B.\ f"ur das Ausdrucken eines (kurzen)
Computer-Programms.

Innerhalb eines Absatzes k"onnen einzelne Zeichenkombinationen
oder kurze Text"-st"u"cke ebenso "`w"ortlich"' ausgedruckt
werden, indem man sie zwischen \verb.\verb|. und~\verb.|.
einschlie"st.
Mit diesen Befehlen wurden z.\,B.\ alle \LaTeX-Befehle in der
vorliegenden Beschreibung gesetzt.
\exa
Der \verb|\dots|-Befehl \dots
\exb
\begin{verbatim}
Der \verb|\dots|-Befehl \dots
\end{verbatim}
\exc
 
Die \texttt{verbatim}-Umgebung und der Befehl~\verb|\verb|
d"urfen \emph{nicht} innerhalb von Parametern von anderen Befehlen
% und auch nicht innerhalb der \texttt{tabular}-""Umgebung %% ??(br)
verwendet werden.


 
\subsubsection{Tabulatoren (tabbing)} \label{tabbing}
 
In der \texttt{tabbing}-Umgebung kann man Tabulatoren "ahnlich wie
an Schreibmaschinen setzen und verwenden.
Der Befehl~\verb|\=| setzt eine Tabulatorposition,
\verb|\kill| bedeutet, da"s die "`Musterzeile"' nicht ausgedruckt werden
soll,
\verb|\>|~springt zur n"achsten Tabulatorposition,
und \verb|\\| trennt die Zeilen.
%
\par\vspace{0pt plus5\baselineskip}
\pagebreak[3]\vspace{0pt plus-5\baselineskip}
%
\exa
\begin{tabbing}
war einmal\quad \=
 Mittelteil\quad \= \kill
links \> Mittelteil \> rechts\\
Es \\
war einmal \> und ist
 \> nicht mehr\\
ein  \>  \> ausgestopfter\\
 \>  \> Teddyb"ar
\end{tabbing}
\exb
\begin{verbatim}
\begin{tabbing}
war einmal\quad \=
 Mittelteil\quad \= \kill
links \> Mittelteil \> rechts\\
Es \\
war einmal \> und ist
 \> nicht mehr\\
ein  \>  \> ausgestopfter\\
 \>  \> Teddyb"ar
\end{tabbing}
\end{verbatim}
\exc


\subsubsection{Tabellen (tabular)} \label{tabular}
 
Die \texttt{tabular}-Umgebung dient zum Setzen von Tabellen, bei
denen \LaTeX\ automatisch die be"-n"otigte Spaltenbreite
bestimmt, und bei der auch spezielle Eigenschaften wie
Rechts"-b"undigkeit und Hilfslinien vereinbart werden k"onnen.
 
Im Parameter des Befehls \verb|\begin{tabular}{...}| wird das
Format der Tabelle angegeben.
Dabei bedeutet
\texttt{l}~eine Spalte mit links"-b"undigem Text,
\texttt{r}~eine mit rechts"-b"undigem,
\texttt{c}~eine mit zentriertem Text,
\verb|p{|\textit{breite}\verb|}| eine Spalte der angegebenen
Breite mit mehrzeiligem Text,
\verb.|.~einen senkrechten Strich.
 
Innerhalb der Tabelle bedeutet
\verb|&|~den Sprung in die n"achste Tabellenspalte,
\verb|\\|~trennt die Zeilen,
\verb|\hline| (an Stelle einer Zeile) setzt einen waagrechten
Strich.
 
          \vspace{0pt plus 1cm}
\exa
\begin{tabular}[t]{|rl|}
\hline
7C0 & hexadezimal \\
3700 & oktal \\
11111000000 & bin"ar \\
\hline\hline
1984 & dezimal \\
\hline
\end{tabular}
\exb
\begin{verbatim}
\begin{tabular}{|rl|}
\hline
7C0 & hexadezimal \\
3700 & oktal \\
11111000000 & bin"ar \\
\hline\hline
1984 & dezimal \\
\hline
\end{tabular}
\end{verbatim}
\exc
 
\endinput
