% master: l2kurz.tex
% L2K3.TEX - 3.Teil der LaTeX2e-Kurzbeschreibung v2.2
% L2K3.TEX - 3.Teil der LaTeX2e-Kurzbeschreibung Mainz 1994, 1995
% LK3.TEX  - 3.Teil der LaTeX-Kurzbeschreibung Graz-Wien 1987
% last changes: 2001-06-10

\section{Setzen von mathematischen Formeln} \label{math}
 
\subsection{Allgemeines}
 
Mathematische Textteile innerhalb eines Absatzes werden zwischen
\verb|\(| und~\verb|\)| oder zwischen \verb|$| und~\verb|$| oder
zwischen \verb|\begin{math}| und \verb|\end{math}|
eingeschlossen.
Als mathematische Texte gelten sowohl komplette mathematische
Formeln als auch einzelne Variablennamen, die sich auf Formeln
beziehen, griechische Buchstaben und diverse Sonderzeichen.\todo{PG: die nachfolgenden Elemente sollten \$...\$ benutzen, die Variante mit \cs{(} und \cs{)} sind nicht robust}
\exa
Seien \(a\) und \(b\) die Katheten
und \(c\) die Hypotenuse,
dann gilt \(c^{2}=a^{2}+b^{2}\)
(Satz des Pythagoras).
\exb
\begin{verbatim}
Seien \(a\) und \(b\) die Katheten
und \(c\) die Hypotenuse,
dann gilt \(c^{2}=a^{2}+b^{2}\)
(Satz des Pythagoras).
\end{verbatim}
\exc
\exa
\TeX\ spricht man wie
 \(\tau\epsilon\chi\) aus.\\[6pt]
Mit \(\heartsuit\)-lichen
 Grüßen
\exb
\begin{verbatim}
\TeX\ spricht man wie
 \(\tau\epsilon\chi\) aus.\\
Mit \(\heartsuit\)-lichen
 Grüßen
\end{verbatim}
\exc
 
Größere mathematische Formeln oder Gleichungen setzt man besser
in eigene Zeilen. Wenn sie \emph{keine} Gleichungsnummer erhalten 
sollen, stellt man sie dazu zwischen \verb|\begin{displaymath}| und
\verb|\end{displaymath}| oder zwischen \verb|\[| und~\verb|\]|; 
wenn sie eine Gleichungsnummer erhalten sollen, stellt man sie
zwischen \verb|\begin{equation}| und \verb|\end{equation}|.

\exa
Seien \(a\) und \(b\) die Katheten
und \(c\) die Hypotenuse,
dann gilt
\begin{equation}
c = \sqrt{  a^{2}+b^{2}  }
\end{equation}
(Satz des Pythagoras).
\exb
\begin{verbatim}
Seien \(a\) und \(b\) die Katheten
und \(c\) die Hypotenuse,
dann gilt
\begin{equation}
c = \sqrt{  a^{2}+b^{2}  }
\end{equation}
(Satz des Pythagoras).
\end{verbatim}
\exc

Mit \verb|\label| und \verb|\ref| kann man die Gleichungsnummern
im Text ansprechen.

\exa
\begin{equation} \label{eps}
\varepsilon > 0
\end{equation}
 
Aus (\ref{eps}) folgt \dots
\exb
\begin{verbatim}
\begin{equation} \label{eps}
\varepsilon > 0
\end{equation}
 
Aus (\ref{eps}) folgt \dots
\end{verbatim}
\exc
 
 
 
Das Setzen im mathematischen Modus unterscheidet sich vom
Text-Modus vor allem durch folgende Punkte:
\begin{enumerate}
\item Leerzeilen sind verboten (Mathematische Formeln müssen
  innerhalb eines Absatzes stehen).

\item Leerstellen und Zeilenwechsel haben bei der Eingabe keine
  Bedeutung, alle Abstände werden nach der Logik der
  mathematischen Ausdrücke automatisch bestimmt oder müssen
  durch spezielle Befehle wie \verb|\,| oder~\verb|\qquad|
  angegeben werden.
\exa
\begin{equation}
\forall x \in \mathbf{R}:
\qquad x^{2} \geq 0
\end{equation}
\exb
\begin{verbatim}
\begin{equation}
\forall x \in \mathbf{R}:
\qquad x^{2} \geq 0
\end{equation}
\end{verbatim}
\exc
 
\item Jeder einzelne Buchstabe wird als Name einer Variablen
  betrachtet und entsprechend gesetzt (kursiv mit
  zusätzlichem Abstand).  Will man innerhalb eines
  mathematischen Textes normalen Text (in aufrechter Schrift, mit
  Wortabständen) setzen, muss man diesen in
  \verb|\textnormal{...}| einschließen.
\exa
\begin{equation}
x^{2} \geq 0\qquad
\textnormal{für alle }
x \in \mathbf{R}
\end{equation}
\exb
\begin{verbatim}
\begin{equation}
x^{2} \geq 0\qquad
\textnormal{f"ur alle }
x \in \mathbf{R}
\end{equation}
\end{verbatim}
\exc
 
 
\end{enumerate}
 
\subsection{Elemente in mathematischen Formeln}
 
In diesem Abschnitt werden die wichtigsten Elemente, die in
mathematischen Formeln verwendet werden, kurz beschrieben.  Eine
Liste aller verfügbaren Symbole enthält
Kapitel~\ref{symbols}.
 
\bigskip

Kleine \textbf{griechische Buchstaben} werden als \verb|\alpha|,
\verb|\beta|, \verb|\gamma|, \verb|\delta|, usw.\ eingegeben,
große griechische Buchstaben als \verb|\mathrm{A}|,
\verb|\mathrm{B}|, \verb|\Gamma|, \verb|\Delta|, usw.

\exa
\(\lambda, \xi, \pi, \mu,
 \Phi, \Omega \)
\exb
\begin{verbatim}
\(\lambda, \xi, \pi, \mu,
 \Phi, \Omega \)
\end{verbatim}
\exc
 
Weiters gibt es eine Fülle von \textbf{mathematischen Symbolen}:
von \(\in\) über \(\Rightarrow\) bis~\(\infty\) (siehe
Kapitel~\ref{symbols}).
 
\bigskip

Neben der voreingestellten Kursivschrift für die Variablen
bietet \LaTeX\ eine Auswahl von mathematischen \textbf{Alphabeten} an:
\exa
\(\mathrm{ABCabc}\) \\
\(\mathbf{ABCabc}\) \\
\(\mathsf{ABCabc}\) \\
\(\mathtt{ABCabc}\) \\
\(\mathcal{ABC}\)
\exb
\begin{verbatim}
\(\mathrm{ABCabc}\)
\(\mathbf{ABCabc}\)
\(\mathsf{ABCabc}\)
\(\mathtt{ABCabc}\)
\(\mathcal{ABC}\)
\end{verbatim}
\exc
Die kalligraphischen Buchstaben (\verb:\mathcal:) gibt es nur als
Großbuchstaben. Mit dem Paket \texttt{amsymb} \cite{ch8} stehen
auch Alphabete für Mengenzeichen und Frakturschrift zur Verfügung.
Lokal können noch weitere installiert sein -- siehe \local.


\bigskip

\textbf{Exponenten und Indizes} können mit den Zeichen \verb|^|
und \verb|_| hoch- bzw.\ tiefgestellt werden.
\exa
\(a_{1}\) \qquad
\(x^{2}\) \qquad
\(e^{-\alpha t}\) \qquad
\(a^{3}_{ij}\)
\exb
\begin{verbatim}
\(a_{1}\) \qquad
\(x^{2}\) \qquad
\(e^{-\alpha t}\) \qquad
\(a^{3}_{ij}\)
\end{verbatim}
\exc
 
Das \textbf{Wurzelzeichen} wird mit \verb|\sqrt|, \textit{n}-te
Wurzeln werden mit \verb|\sqrt[|\textit{n}\verb|]| eingegeben.
Die Größe des Wurzelzeichens wird von \LaTeX\ automatisch
gewählt.
\exa
\(\sqrt{x}\) \qquad
\(\sqrt{ x^{2}+\sqrt{y} }\)
\qquad \(\sqrt[3]{2}\)
\exb
\begin{verbatim}
\(\sqrt{x}\) \qquad
\(\sqrt{ x^{2}+\sqrt{y} }\)
\qquad \(\sqrt[3]{2}\)
\end{verbatim}
\exc
 
Die Befehle \verb|\overline| und \verb|\underline| bewirken
\textbf{waagrechte Striche} direkt über bzw.\ unter einem
Ausdruck.
\exa
\(\overline{m+n}\)
\exb
\begin{verbatim}
\(\overline{m+n}\)
\end{verbatim}
\exc
 
Die Befehle \verb|\overbrace| und \verb|\underbrace| bewirken
\textbf{waagrechte Klammern} über bzw.\ unter einem Ausdruck.
\exa
\(\underbrace{a+b+\cdots+z}_{26}\)
\exb
\begin{verbatim}
\(\underbrace{a+b+\cdots+z}_{26}\)
\end{verbatim}
\exc
 
Um mathematische \textbf{Akzente} wie Pfeile oder Schlangen auf
Variablen zu setzen, gibt es die in Tabelle~\ref{mathakz} auf
Seite~\pageref{mathakz} angeführten Befehle.
Längere Tilden und Dacherln, die sich über mehrere (bis zu~3)
Zeichen erstrecken können, erhält man mit \verb|\widetilde|
bzw.\ \verb|\widehat|.
Ableitungszeichen werden mit \verb|'| (Apostroph) eingegeben.\todo{PG: Hier würde ich statt displaymath auf \cs{[} und \cs{]} zurückgreifen, ist kürzer und besser lesbar (und wird in der Praxis vermutlich mehr benutzt?)}
\exa
\begin{displaymath}
y=x^{2} \qquad
y'=2x   \qquad
y''=2
\end{displaymath}
\exb
\begin{verbatim}
\begin{displaymath}
y=x^{2} \qquad
y'=2x   \qquad
y''=2
\end{displaymath}
\end{verbatim}
\exc
 
Mathematische \textbf{Funktionen} werden in der Literatur
üblicherweise nicht kursiv (wie die Namen von Variablen),
sondern in "`normaler"' Schrift dargestellt.
\LaTeX\ stellt die folgenden Befehle für mathematische
Funktionen zur Verfügung:
\begin{verbatim}
\arccos   \cos    \csc   \exp   \ker     \limsup  \min   \sinh
\arcsin   \cosh   \deg   \gcd   \lg      \ln      \Pr    \sup
\arctan   \cot    \det   \hom   \lim     \log     \sec   \tan
\arg      \coth   \dim   \inf   \liminf  \max     \sin   \tanh
\end{verbatim}
Für die Modulo-Funktion gibt es zwei verschiedene Befehle:
\verb|\bmod| für den binären Operator \(a \bmod b\) und
\verb|\pmod{...}| für die Angabe in der Form \(x\equiv a
\pmod{b}\).
 
\exa
\begin{displaymath}
\lim_{x \to 0} \frac{\sin x}{x}
=1
\end{displaymath}
\exb
\begin{verbatim}
\begin{displaymath}
\lim_{x \to 0} \frac{\sin x}{x}
=1
\end{displaymath}
\end{verbatim}
\exc
 
Ein \textbf{Bruch} (fraction) wird mit dem Befehl
\verb|\frac{...}{...}| gesetzt.  Für einfache Brüche kann man
aber auch den Operator~\verb|/| verwenden.
\exa
\(1\frac{1}{2}\)~Stunden
\begin{displaymath}
\frac{ x^{2} }{ k+1 } \qquad
x^{ \frac{2}{k+1} } \qquad
x^{ 1/2 }
\end{displaymath}
\exb
\begin{verbatim}
\(1\frac{1}{2}\)~Stunden
\begin{displaymath}
\frac{ x^{2} }{ k+1 } \qquad
x^{ \frac{2}{k+1} } \qquad
x^{ 1/2 }
\end{displaymath}
\end{verbatim}
\exc

\textbf{Binomial-Koeffizienten} können in der Form
\verb|{...\choose...}| gesetzt werden.
Mit dem Befehl~\verb|\atop| erhält man das Gleiche ohne
Klammern.
\exa
\begin{displaymath}
{ n \choose k } \qquad
{ x\atop y+2 }
\end{displaymath}
\exb
\begin{verbatim}
\begin{displaymath}
{ n \choose k } \qquad
{ x\atop y+2 }
\end{displaymath}
\end{verbatim}
\exc

\medskip

Das \textbf{Integralzeichen} wird mit \verb|\int| eingegeben, das
\textbf{Summenzeichen} mit \verb|\sum|.
Die obere und untere Grenze wird mit \verb|^| bzw.~\verb|_| wie
beim \mbox{Hoch-}\slash Tiefstellen angegeben.
 
Normalerweise werden die Grenzen neben das Integralzeichen
gesetzt (um Platz zu sparen), durch Einfügen des Befehls
\verb|\limits| wird erreicht, dass die Grenzen oberhalb und
unterhalb des Integralzeichens gesetzt werden.
 
Beim Summenzeichen hingegen werden die Grenzen bei der Angabe von
\verb|\nolimits| oder im laufenden Text neben das Summenzeichen
gesetzt, ansonsten aber unter- und oberhalb.
\exa
\begin{displaymath}
\sum_{i=1}^{n} \qquad
\int_{0}^{\frac{\pi}{2}} \qquad
\int \limits_{-\infty}^{+\infty}
\end{displaymath}
\exb
\begin{verbatim}
\begin{displaymath}
\sum_{i=1}^{n} \qquad
\int_{0}^{\frac{\pi}{2}} \qquad
\int \limits_{-\infty}^{+\infty}
\end{displaymath}
\end{verbatim}
\exc
 
Für \textbf{Klammern} und andere Begrenzer gibt es in \TeX\ 
viele verschiedene Symbole
(z.\,B.~\([\;\langle\;\|\;\updownarrow\)).
Runde und eckige Klammern können mit den entsprechenden Tasten
eingegeben werden, geschwungene mit~\verb|\{|, die anderen mit
speziellen Befehlen (z.\,B.~\verb|\updownarrow|).
 
Setzt man den Befehl \verb|\left| vor öffnende Klammern und den
Befehl \verb|\right| vor schließende, so wird automatisch die
richtige Größe gewählt.\todo{MD: in gleicher Zeile!}
\exa
\begin{displaymath}
1 + \left( \frac{1}{ 1-x^{2} }
    \right) ^3
\end{displaymath}
\exb
\begin{verbatim}
\begin{displaymath}
1 + \left( \frac{1}{ 1-x^{2} }
    \right) ^3
\end{displaymath}
\end{verbatim}
\exc
 
In manchen Fällen möchte man die Größe der Klammern lieber
selbst festlegen, dazu sind die Befehle
\verb|\bigl|,
\verb|\Bigl|,
\verb|\biggl| und
\verb|\Biggl| anstelle von \verb|\left|
und analog \verb|\bigr| etc.\ anstelle von \verb|\right|
anzugeben.
\exa
\begin{displaymath}
\Bigl( (x+1) (x-1) \Bigr) ^{2}
\end{displaymath}
\exb
\begin{verbatim}
\begin{displaymath}
\Bigl( (x+1) (x-1) \Bigr) ^{2}
\end{displaymath}
\end{verbatim}
\exc
 
Um in Formeln \textbf{3~Punkte} (z.\,B.\ für \(1,2,\ldots,n\))
auszugeben, gibt es die Befehle
\verb|\ldots| und \verb|\cdots|.
\verb|\ldots| setzt die Punkte auf die Grundlinie (low),
\verb|\cdots| setzt sie in die Mitte der Zeilenhöhe
(centered).
Außerdem gibt es die Befehle
\verb|\vdots| für vertikal und
\verb|\ddots| für diagonal angeordnete Punkte.
\exa
\begin{displaymath}
x_{1},\ldots,x_{n} \qquad
x_{1}+\cdots+x_{n}
\end{displaymath}
\exb
\begin{verbatim}
\begin{displaymath}
x_{1},\ldots,x_{n} \qquad
x_{1}+\cdots+x_{n}
\end{displaymath}
\end{verbatim}
\exc


\subsection{Nebeneinander Setzen}
 
Wenn man mit den von \TeX\ gewählten \textbf{Abständen}
innerhalb von Formeln nicht zufrieden ist, kann man sie mit
expliziten Befehlen verändern. Die wichtigsten sind
\verb|\,| für einen sehr kleinen Abstand,
\verb*|\ | für einen mittleren,
\verb|\quad| und \verb|\qquad| für große Abstände sowie\todo{MD: \texttt{\string\quad=1em \string\qquad=2em} . Sollte erwähnt werden, denn so erkennt man die Schriftabhängigkeit}
\verb|\!| für die Verkleinerung eines Abstands.
\exa
\begin{displaymath}
F_{n} = F_{n-1} + F_{n-2}
 \qquad n \ge 2
\end{displaymath}
\exb
\begin{verbatim}
\begin{displaymath}
F_{n} = F_{n-1} + F_{n-2}
 \qquad n \ge 2
\end{displaymath}
\end{verbatim}
\exc
 
\exa
\begin{displaymath}
\int\!\!\!\int_{D} \mathrm{d}x\,\mathrm{d}y
\quad \textnormal{statt} \quad
\int\int_{D} \mathrm{d}x \mathrm{d}y
\end{displaymath}
\exb
\begin{verbatim}
\begin{displaymath}
\int\!\!\!\int_{D} 
\mathrm{d}x\,\mathrm{d}y
\quad \textnormal{statt} \quad
\int\int_{D} 
\mathrm{d}x \mathrm{d}y
\end{displaymath}
\end{verbatim}
\exc


\subsection{Übereinander Setzen}

Für \textbf{Matrizen} u.\,ä.\ gibt es die
\texttt{array}-Umgebung, die ähnlich wie die
\texttt{tabular}-Umgebung funktioniert.
Der Befehl~\verb|\\| trennt die Zeilen.
\exa
\begin{displaymath}
\mathbf{X} =
\left( \begin{array}{ccc}
x_{11} & x_{12} & \ldots \\
x_{21} & x_{22} & \ldots \\
\vdots & \vdots & \ddots
\end{array} \right)
\end{displaymath}
\exb
\begin{verbatim}
\begin{displaymath}
\mathbf{X} =
\left( \begin{array}{ccc}
x_{11} & x_{12} & \ldots \\
x_{21} & x_{22} & \ldots \\
\vdots & \vdots & \ddots
\end{array} \right)
\end{displaymath}
\end{verbatim}
\exc

Für \textbf{mehrzeilige} Formeln oder Gleichungssysteme
verwendet man die Umgebungen \texttt{eqnarray} und
\texttt{eqnarray*} statt \texttt{equation}. \todo{PG: eqnarray ist böse und sollte nicht verwendet werden. Besser: align. Zumindest habe ich das so in der letzten TUGboat gelesen.\\MD: Der komplette Abschnitt muss überarbeitet werden}
Bei \texttt{eqnarray} erhält jede Zeile eine eigene
Gleichungsnummer, bei \texttt{eqnarray*} wird ebenso wie bei
\texttt{displaymath} \emph{keine} Nummer hinzugefügt.
Für Gleichungssysteme, die \emph{eine} gemeinsame Nummer
erhalten sollen, kann man eine \texttt{array}-Umgebung innerhalb
der \texttt{equation}-Umgebung verwenden.

Die Umgebungen \texttt{eqnarray} und \texttt{eqnarray*}
funktionieren wie eine 3-spaltige Tabelle der Form~\verb|{rcl}|,
wobei die mittlere Spalte für das Gleichheits- oder
Ungleichheitszeichen verwendet wird, nach dem die Zeilen
ausgerichtet werden sollen.
Der Befehl~\verb|\\| trennt die Zeilen.
\exa
\begin{eqnarray}
f(x) & = & \cos x       \\
f'(x) & = & -\sin x     \\
\int_{0}^{x} f(y)\,\mathrm{d}y &
 = & \sin x
\end{eqnarray}
\exb
\begin{verbatim}
\begin{eqnarray}
f(x) & = & \cos x       \\
f'(x) & = & -\sin x     \\
\int_{0}^{x} f(y)\,\mathrm{d}y &
 = & \sin x
\end{eqnarray}
\end{verbatim}
\exc
 
\textbf{Zu lange Gleichungen} werden von \LaTeX\ \textit{nicht}
automatisch abgeteilt.
Der Autor muss bestimmen, an welcher Stelle abgeteilt und wie
weit eingerückt werden soll.
Meistens verwendet man dafür eine der beiden folgenden
Varianten:
\exa
\begin{eqnarray}
\sin x & = & x -\frac{x^{3}}{3!}
     +\frac{x^{5}}{5!} - {} % empty group for correct spacing
                            % around the `-'
                    \nonumber\\
 & &{} -\frac{x^{7}}{7!} + \cdots % ditto
\end{eqnarray}
\exb
\begin{verbatim}
\begin{eqnarray}
\sin x & = & x -\frac{x^{3}}{3!}
     +\frac{x^{5}}{5!} - {} 
                    \nonumber\\
 & &{} -\frac{x^{7}}{7!} + \cdots
\end{eqnarray}
\end{verbatim}
\exc
\exa
\begin{eqnarray}
\lefteqn{ \cos x = 1
     -\frac{x^{2}}{2!} + {} }
                    \nonumber\\
 & & {} +\frac{x^{4}}{4!}
     -\frac{x^{6}}{6!} + \cdots
\end{eqnarray}
\exb
\begin{verbatim}
\begin{eqnarray}
\lefteqn{ \cos x = 1
     -\frac{x^{2}}{2!} +{} }
                    \nonumber\\
 & &{} +\frac{x^{4}}{4!}
     -\frac{x^{6}}{6!} + \cdots
\end{eqnarray}
\end{verbatim}
\exc
Der Befehl \verb|\nonumber| bewirkt, dass an diese Stelle keine
Gleichungsnummer gesetzt wird.
Der Befehl \verb|\lefteqn| ermöglicht Ausnahmen von der
Spaltenaufteilung innerhalb \texttt{eqnarray}. 
Genauere Informationen enthält das \manual.

\endinput
