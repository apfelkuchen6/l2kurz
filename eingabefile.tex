\section{Eingabefile}
Das Eingabefile für \LaTeX{} ist ein Textfile mit der Endung \verb+.tex+. Es 
wird mit einem Editor erstellt und enthält sowohl den Text, der gedruckt werden 
soll, als auch die Befehle, aus denen \LaTeX\ erfährt, wie der Text gesetzt 
werden soll.\todo{MD: Sollte hier ein Hinweis auf Editoren gegeben werden?
bspw: \href{http://tex.stackexchange.com/questions/339/latex-editors-ides}{LaTeX Editors/IDEs}\\PG: ich würde erwähnen, dass es spezielle Editoren dafür gibt, z.B. (einen erwähnen, wie beispielsweise TeXmaker)}


\subsection{Leerstellen}
 
"`Unsichtbare"' Zeichen wie das Leerzeichen, Tabulatoren und das Zeilenende 
werden von \LaTeX{} einheitlich als Leerzeichen behandelt. \emph{Mehrere}
Leerzeichen werden wie \emph{ein} Leerzeichen behandelt. Wenn man andere als 
die normalen Wort- und Zeilenabstände will, kann man dies also nicht durch die 
Eingabe von zusätzlichen Leerzeichen oder Leerzeilen erreichen, sondern nur mit 
entsprechenden \LaTeX-Befehlen.

Eine Leerzeile zwischen Textzeilen bedeutet das Ende eines Absatzes.
\emph{Mehrere} Leerzeilen werden wie \emph{eine} Leerzeile behandelt.
 
 
\subsection{\LaTeX-Befehle und Gruppen}
 
Die meisten \LaTeX-Befehle haben eines der beiden folgenden Formate: Entweder 
sie beginnen mit einem Backslash~(\verb|\|) und haben dann einen nur aus 
Buchstaben bestehenden Namen, der durch ein oder mehrere Leerzeichen oder durch
ein nachfolgendes Sonderzeichen oder eine Ziffer beendet wird; oder sie bestehen
aus einem Backslash und genau einem Sonderzeichen oder einer \todo{MD: Ich finde Ziffer verwirrt irgendwie?\\PG: ja, Ziffer können wir rausnehmen. Hier und im Satz davor.}Ziffer.
Groß- und Kleinbuchstaben haben auch in Befehlsnamen \emph{verschiedene} 
Bedeutung. Wenn man nach einem Befehlsnamen eine Leerstelle erhalten will, muss 
man~\verb|{}| zur Beendigung des Befehlsnamens oder einen eigenen Befehl für die
Leerstelle verwenden.

% PG: würde ich drin lassen
\renewcommand{\today}{35.~Mai 2012}  % to make sure that the
% line breaks look good, regardless of the date of printing.
\begin{LTXexample}
Heute ist der \today.
Oder: Heute ist der \today .
Falsch ist:
 Am \today regnet es.
Richtig:
 Am \today{} scheint die Sonne.
 Oder: Am \today\ schneit es.
\end{LTXexample}
 
Manche Befehle haben Parameter, die zwischen geschwungenen Klammern angegeben 
werden müssen. Manche Befehle haben Parameter, die weggelassen oder zwischen
eckigen Klammern angegeben werden können. Manche Befehle haben Varianten, die 
durch das Hinzufügen eines Sterns an den Befehlsnamen unterschieden werden.

Geschwungene Klammern können auch dazu verwendet werden, Gruppen (\emph{groups})
zu bilden. Die Wirkung von Befehlen, die innerhalb von Gruppen oder Umgebungen 
(\emph{environments}) angegeben werden, endet immer mit dem Ende der Gruppe 
bzw.\ der Umgebung.  Im obigen Beispiel ist~\verb|{}| eine leere Gruppe, die 
außer der Beendigung des Befehlsnamens \texttt{today} keine Wirkung hat.
 
\subsection{Kommentare}
 
Alles, was hinter einem Prozentzeichen (\verb|%|) steht (bis zum Ende der 
Eingabezeile), wird von \LaTeX{} ignoriert. Dies kann für Notizen des Autors 
verwendet werden, die nicht oder noch nicht ausgedruckt werden sollen.
\begin{LTXexample}
Das ist ein % dummes
% Besser: ein lehrreiches <----
Beispiel. 
\end{LTXexample}
 
\subsection{Aufbau}
 
Der erste Befehl in einem \LaTeX-Eingabefile muss der Befehl
\begin{quote}
\verb|\documentclass|
\end{quote}
sein. Er legt fest, welche Art von Schriftstück überhaupt erzeugt werden soll
(Bericht, Buch, Brief usw.). Danach können weitere Befehle folgen, die für das
gesamte Dokument gelten sollen.  Dieser Teil des Dokuments wird auch als 
\emph{Vorspann} oder \emph{Präambel} bezeichnet.  Mit dem Befehl
\begin{quote}
\verb|\begin{document}|
\end{quote}
endet der Vorspann, und es beginnt das Setzen des Schriftstücks. Nun folgen der 
Text und alle \LaTeX-Befehle, die das Ausdrucken des Schriftstücks bewirken.
\todo{MD: Sollte hier ein Hinweis hin, dass Anweisungen im Header definiert werden sollten\\PG: steht doch oben?!?}
Die Eingabe muss mit dem Befehl
\begin{quote}
\verb|\end{document}|
\end{quote}
beendet werden. Falls nach diesem Befehl noch Eingaben folgen, werden sie von
\LaTeX{} ignoriert.
 
Abbildung~\ref{mini} zeigt ein \emph{minimales} \LaTeX-File. Ein etwas 
komplizierteres File ist in Abbildung~\ref{dokument} skizziert.
 
\begin{figure}[hbp] %\small
\begin{lminipage}{6cm}
\begin{lstlisting}
\documentclass{article}
\begin{document}
 Small is beautiful.
\end{document} 
\end{lstlisting}
\end{lminipage}
\caption{Ein minimales \LaTeX-File} \label{mini}
\end{figure}

% im folgenden Beispiel sollten Umlaute im Eingabefile auftreten!
\begin{figure}[hbtp]
\todo[inline]{MD: Wirklich als float definieren? Ich würde es schöner finden, wenn wir miniage weglassen. PG: Gute Idee, kein Float, keine Minipage, einfach das Listing} 
\begin{lminipage}{10cm}
\begin{lstlisting}
\documentclass[11pt,a4paper,ngerman]{article}
\usepackage[utf8]{inputenc}
\usepackage[T1]{fontenc}
\usepackage{babel}
\date{29. Februar 1998}
\author{H.~Partl}
\title{Über kurz oder lang}

\begin{document}
\maketitle
\begin{abstract}
Beispiel für einen wissenschaftlichen Artikel
in deutscher Sprache.
\end{abstract}
\tableofcontents

\section{Start}
Hier beginnt mein schönes Werk ...

\section{Ende}
... und hier endet es.

\end{document}
\end{lstlisting}
\end{lminipage}
\caption{Aufbau eines Artikels} \label{dokument}
\end{figure}
 
\todo[inline]{MD: Irgendwie fehlt mir in den nächsten Ausführungen der Hinweis auf
Paketdokumentationen\\PG: finde ich nicht so dramatisch. Ggf. in dem Abschnitt über Hilfe kann texdoc erwähnt werden!?}
 
\subsection{Dokumentklassen}\label{docsty}
 
Die am Beginn des Eingabefiles  mit
\begin{beispiel}
\verb|\documentclass[|\textit{optionen}\verb|]{|%
  \textit{klasse}\verb|}|
\end{beispiel}
definierte "`Klasse"' eines Dokumentes enthält Vereinbarungen über das Layout 
und die logischen Strukturen, z.\,B.\ die Gliederungseinheiten (Kapitel etc.\@), 
die für alle Dokumente dieses Typs gemeinsam sind.

Zwischen den geschwungenen Klammern \emph{muss} genau eine Dokumentklasse
angegeben werden.  Tabelle~\ref{docstyles} auf S.~\pageref{docstyles} führt 
Klassen auf, die in jeder \LaTeX-Installation existieren sollten. Im \local\ 
können weitere verfügbare Klassen angegeben sein.  \todo{MD: Klassen, die für den Einstieg empfehlenswert sind\\PG: Liste kommt doch später}
 
Zwischen den eckigen Klammern \emph{können}, durch Kommata getrennt, eine oder 
mehrere Optionen für das Klassenlayout angegeben werden. Die wichtigsten 
Optionen sind in der Tabelle~\ref{options} auf S.~\pageref{options} angeführt.
Das Eingabefile für diese Beschreibung beginnt z.\,B.\ mit:
\begin{beispiel}
\verb|\documentclass[11pt,a4paper]{article}|
\end{beispiel}

\begin{table}[hbpt]
\caption{Dokumentklassen} \label{docstyles}
%%%%%%%%%%%%%%%%%%%%%%%%%%%%%%%%%%%%%%%%%%%%%%%%%%%%%%%%%%%%%%%%%%%%%%%%%%%%%%%
\todo[inline]{MD: Vorschlag: memoir, scrlttr2, beamer und powerdot. Der Hinweis 
Local Guide ist m. E. nicht nötig (siehe Tabelle).\newline PG: memoir würde ich 
nicht  nehmen, zu Umfangreich, keine dt. Anl. Letter würde ich auch nicht nehmen, 
welcher Anfänger will was über Briefe wissen?}
%%%%%%%%%%%%%%%%%%%%%%%%%%%%%%%%%%%%%%%%%%%%%%%%%%%%%%%%%%%%%%%%%%%%%%%%%%%%%%%
\begin{lminipage}{11cm}
\begin{ttdescription}%\small
\item [article] für Artikel in wissenschaftlichen Zeitschriften,
  kürzere Berichte u.\,v.\,a.
 
\item [report] für längere Berichte, die aus mehreren Kapiteln
  bestehen, Diplomarbeiten, Dissertationen u.\,ä.
 
\item [book] für Bücher

\item[scrartcl, scrreprt, scrbook]\quad Die sog. KOMA-Klassen 
sind Varianten der o.\,g. Klassen
mit besserer Anpassung an DIN-Papierformate und "`europäische"'
Typographie. 
(Nicht überall vorhanden, siehe \local.)

% \item [proc] für Konferenzbände (Proceedings)
% ist nicht einmal im LaTeX-Handbich beschrieben!

\item [letter] für Briefe (siehe auch Abschnitt~\ref{briefe})

\item [foils] für Folien oder Präsentationen.
(Nicht überall vorhanden, siehe \local.)
  
\end{ttdescription}
\end{lminipage}
\end{table}

\begin{table}[hbpt]
\caption[Klassenoptionen]{Klassenoptionen (Alternativen sind durch \texttt{|}
  getrennt)} \label{options}
\begin{lminipage}{11cm}
\begin{ttdescription}%\small
\item [10pt|11pt|12pt] wählt die normale Schriftgröße des Dokuments aus.
  10\,pt hohe Schrift ist die Voreinstellung; diese Beschreibung benutzt 11\,pt.

\item[a4paper] für Papier im DIN\,A4-Format. Ohne diese
  Option nimmt \LaTeX\ amerikanisches Papierformat an.
 
\item [fleqn] für linksbündige statt zentrierte mathematische
  Gleichungen
 
\item [leqno] für Gleichungsnummern links statt rechts von jeder
  nummerierten Gleichung
 
\item [titlepage|notitlepage] legt fest, ob Titel und Zusammenfassung
  auf einer eigenen Seite erscheinen sollen.  \texttt{titlepage} ist
  die Voreinstellung für die Klassen \texttt{report} und \texttt{book}.
 
\item [onecolumn|twocolumn] für ein- oder zweispaltigen Satz.
 Die Voreinstellung ist immer \texttt{onecolumn}.  
 Die Klassen \texttt{letter} und \texttt{slides} kennen \emph{keinen}
 zweispaltigen Satz.
 
\item [oneside|twoside] legt fest, ob die Seiten für ein- oder
  zweiseitigen  Druck gestaltet werden sollen.  
  \texttt{oneside} ist die Voreinstellung für
  alle Klassen außer \texttt{book}.
  
\end{ttdescription}
\end{lminipage}
\end{table}



\subsection{Pakete}\label{packages}
 
Mit dem Befehl
\begin{beispiel}
\verb:\usepackage[:\textit{optionen}\verb:]{:%
  \textit{pakete}\verb:}:
\end{beispiel}
können im Vorspann ergänzende Makropakete (packages) geladen werden, die das 
Layout der Dokumentklasse modifizieren oder zusätzliche Funktionalität 
bereitstellen. Eine Auswahl von Paketen findet sich in der Tabelle~\ref{pack} 
auf S.~\pageref{pack}. Das Eingabefile für diese Beschreibung enthält 
beispielsweise:
\todo{MD: Ich halte nicht soviel von verschachteltem Laden. Die Optionenübergabe fehlt völlig\\PG: finde ich nicht so dramatisch, aber können wir entweder rauslassen oder einen Kommentar hinzufügen im Sinne von ...'wenn keine Optionen übergeben werden (oder wenn die Optionen für alle Pakete gelten sollen), findet man auch manchmal die Kurzform... ' -- egal}
\begin{beispiel}
\verb|\usepackage{babel,latexsym|\\
\verb|            graphicx,textcomp,hyperref}|
\end{beispiel}
%%%%%%%%%%%%%%%%%%%%%%%%%%%%%%%%%%%%%%%%%%%%%%%%%%%%%%%%%%%%%%%%%%%%%%%%%%%%%%%
\todo{MD: stimmt nicht mehr.}
%%%%%%%%%%%%%%%%%%%%%%%%%%%%%%%%%%%%%%%%%%%%%%%%%%%%%%%%%%%%%%%%%%%%%%%%%%%%%%%
\begin{table}[htbp]
\caption{Pakete (eine Auswahl)}\label{pack}
\begin{lminipage}{11cm}
%%%%%%%%%%%%%%%%%%%%%%%%%%%%%%%%%%%%%%%%%%%%%%%%%%%%%%%%%%%%%%%%%%%%%%%%%%%%%%%
\todo[inline]{PG: m.E. können latexsym und dcolumn raus, dafür listings, 
              csquotes, microtype und ...? rein. tabularx finde ich nicht
              schlecht.\newline MD: Um zeitgemäß zu bleiben, ist es sinnvoll}
%%%%%%%%%%%%%%%%%%%%%%%%%%%%%%%%%%%%%%%%%%%%%%%%%%%%%%%%%%%%%%%%%%%%%%%%%%%%%%%
\begin{ttdescription}%\small
\setlength{\itemsep}{.5\itemsep plus1pt minus1pt}
% \item[alltt] Definiert eine Variante der \texttt{verbatim}-Umgebung
\item[amsmath, amssymb] Mathematischer Formelsatz mit erweiterten Fähigkeiten,
  zusätzliche mathematische Schriften und Symbole; Beschreibung siehe
  \cite{ch8}.
%\item[array] Verbesserte und erweiterte Versionen der Umgebungen
%  \texttt{array}, \texttt{tabular} und \texttt{tabular*}.
\item[babel] Anpassungen für viele verschiedene Sprachen. Die
  gewählten Sprachen werden als Optionen angegeben.
\item[color] Unterstützung für Farbausgabe;
  Beschreibung  siehe~\cite{grfguide} und~\cite{grfcomp}.
\item[dcolumn] Unterstützt auf Dezimaltrennzeichen ausgerichtete
  Spalten in den Umgebungen \texttt{array} und \texttt{tabular}
\item[fontenc] Erlaubt die Verwendung von Schriften mit
  unterschiedlicher Kodierung (Zeichenvorrat, Anordnung).
\item[fancyhdr] Flexible Gestaltung von Kopf- und Fußzeilen.
\item[geometry] Manipulation des Seitenlayouts.
% (n)german rausgenommen
\item[graphicx] Einbindung von extern erzeugten Graphiken.
  Die umfangreichen Möglichkeiten dieses Pakets werden 
  in~\cite{grfguide} und~\cite{grfcomp} beschrieben.
\item[hyperref] Ermöglicht Hyperlinks zwischen Textstellen und zu
  externen Dokumenten; besonders sinnvoll einsetzbar, 
  wenn mit \TeX\ eine Ausgabedatei im \textsc{pdf}- oder \textsc{html}-Format 
  erzeugt wird.
\item[inputenc] Deklaration der Zeichenkodierung im
  Eingabefile.
\item[latexsym] Erlaubt einige besondere Symbole wie~\(\Box\),
  die mit \LaTeX~2.09 standardmäßig verfügbar waren.
\item[longtable]
  für Tabellen über mehrere Seiten mit automatischem Seitenumbruch.
\item[makeidx] Unterstützt das Erstellen eines Index.
\item[multicol] Mehrspaltiger Satz mit Kolumnenausgleich.
%\item[showkeys] Druckt die Namen aller verwendeten \verb:\label:s,
%  \verb:\ref:s und \verb:\pageref:s im Text aus.
%\item[tabularx] für Tabellen mit automatisch an den vorhandenen
%  Platz angepasster Breite der Spalten.
\item[textcomp] Bindet Schriften mit zusätzlichen Textsymbolen ein.
%\item[verbatim] Flexible Erweiterung der \texttt{verbatim}-Umgebung.
\end{ttdescription}
\end{lminipage}
\end{table}


\subsection{Eingabezeichensatz}\label{inputenc}

Bei jedem \LaTeX-System dürfen mindestens die folgenden Zeichen zur Eingabe von 
Text verwendet werden:
\begin{quote}
  \ttfamily
  a\dots z A\dots Z 0\dots 9 \\
  . : ; , ? ! ` ' ( ) [ ] - / * @ + =
\end{quote}
Die folgenden Eingabezeichen haben für \LaTeX{} eine Spezialbedeutung oder sind 
nur innerhalb von mathematischen Formeln erlaubt:
\begin{quote}
\verb.$ & % # _ { }  ~  ^  "  \  | < >.
\end{quote}
Für Zeichen, die über obige Liste hinausgehen, beispielsweise die Umlaute, sind 
je nach Betriebssystem des verwendeten Computers unterschiedliche Kodierungen in
Gebrauch.  Damit auch diese Zeichen im Eingabefile benutzt werden dürfen,  muss 
man das Paket \texttt{inputenc} laden und dabei die jeweilige Kodierung als 
Option angeben: \verb:\usepackage[:\textit{codepage}\verb:]{inputenc}:.
Mögliche Angaben für \textit{codepage} sind u.\,a.:
%%%%%%%%%%%%%%%%%%%%%%%%%%%%%%%%%%%%%%%%%%%%%%%%%%%%%%%%%%%%%%%%%%%%%%%%%%%%%%%
\todo{MD: hier sollte auf selinput verwiesen werden. Der Nutzer braucht nicht
auf Kodierung zu achten \newline
PG: gute Idee. Der ganze Quatsch mit den mit den codepages verwirrt nur 
die Anfänger}
%%%%%%%%%%%%%%%%%%%%%%%%%%%%%%%%%%%%%%%%%%%%%%%%%%%%%%%%%%%%%%%%%%%%%%%%%%%%%%%
\begin{ttdescription}
  \item[latin1] Latin-1 (ISO~8859-1), gebräuchlich unter \textsc{Unix} und VMS
  \item[latin9] Latin-9 (ISO~8859-15), Erweiterung von Latin-1, u.\,a. mit 
                Eurozeichen
  \item[ansinew] Microsoft Codepage 1252 für Windows
  \item[cp850] IBM Codepage 850, üblich unter OS/2
  \item[applemac] \textsc{Macintosh}-Kodierung
\end{ttdescription}
Falls \LaTeX{} ein eingegebenes Zeichen nicht darstellen kann, was meist für die
sogenannten "`Pseudografik-Zeichen"' gilt,  bekommt man eine entsprechende 
Fehlermeldung. Auch sind manche Zeichen nur im Text, andere nur in 
mathematischen Formeln erlaubt.

Man beachte, dass der in der \emph{Ausgabe} darstellbare Zeichenvorrat von 
\LaTeX{} nicht davon abhängt, welche Zeichen als \emph{Eingabe} erlaubt sind:
Für jedes überhaupt darstellbare Zeichen -- also auch diejenigen, die nicht im 
Zeichensatz des jeweiligen Betriebssystems enthalten sind -- gibt es einen 
\LaTeX-Befehl oder eine Ersatzdarstellung, die ausschließlich mit ASCII-Zeichen 
auskommt.  Näheres darüber erfahren Sie in Abschnitt \ref{spezial}.

\endinput
