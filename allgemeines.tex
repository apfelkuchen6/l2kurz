% master: l2kurz.tex
% l2k1.tex - 1.Teil der LaTeX2e-Kurzbeschreibung v2.3
% 2003-04-10 (WaS)
\fcolorbox{black}{yellow}{%
\parbox{.9\linewidth}{MD:
Wie sieht es eigentlich mit der Lizenz aus. Wir ändern ja wohl knap 50\% bis 80\%. Sollte es somit nicht eher
ein xl2kurz werden? Was noch ergänzt werden sollte:
\begin{itemize}
\item Literaturempfehlungen 
\item Nutzung von Dokumentationen
\item DANTE-FAQ sowie UK-FAQ
\item Fehlersucher -- Stichwort Minimalbeispiel
\item Online-Hilfe // Foren
\end{itemize}
}
}

 

\section{Allgemeines}
 
\subsection{The Name of the Game}
 
\subsubsection{\TeX}

\TeX\ (sprich "`Tech"', kann auch "`TeX"' geschrieben werden) ist
ein Computerprogamm von Donald E.~Knuth~\cite{texbook,schwarz}.
Es dient zum Setzen 
% und Drucken 
von Texten und mathematischen Formeln.
 
\subsubsection{\LaTeX}
 
\LaTeX\ (sprich "`Lah-tech"' oder "`Lej-tech"', kann auch
"`LaTeX"' geschrieben werden) ist ein auf \TeX\ auf\/bauendes 
Computerprogramm und wurde von Leslie Lamport~\cite{manual,wonne} 
geschrieben.  Es vereinfacht den Umgang mit \TeX, indem es 
entsprechend der logischen Struktur des Dokuments auf vorgefertigte
Layout-Elemente zurückgreift.

\subsubsection{\LaTeXe}

\LaTeXe\ (sprich "`\LaTeX\ zwei e"') ist die aktuelle Variante von
\LaTeX\ seit dem 1.~Juni 1994.  (Die vorherige hieß \LaTeX~2.09.)
Wenn hier von \LaTeX\ gesprochen wird, so ist normalerweise dieses
\LaTeXe{} gemeint.\todo{PG: interessiert das heute noch jemanden? GGf. Verweis auf latex3}

Neue Versionen
von \LaTeXe{} (z.\,B. mit Fehlerberichtigungen oder Ergänzungen)
erscheinen jährlich im Juni\todo{PG: stimmt das noch?\\MD: Ist nicht mehr gültig. Dafür gibt es jetzt nur die Bug-Database.}; die vorliegende Beschreibung setzt 
mindestens diejenige vom Juni 2001 voraus.  

\subsection{Grundkonzept}
 
\subsubsection{Autor, Designer und Setzer}
 
Für eine Publikation übergab der Autor dem Verleger
traditionell  ein maschinengeschriebenes Manuskript.  Der
Buch-Designer des Verlages entschied dann über das Layout des
Schriftstücks (Länge einer Zeile, Schriftart, Abstände vor
und nach Kapiteln usw.\@) und schrieb dem Setzer die
dafür notwendigen Anweisungen dazu.
\LaTeX{} ist in diesem Sinne der Buch-Designer, 
das Programm \TeX{} ist sein Setzer.
 
Ein menschlicher Buch-Designer erkennt die Absichten des Autors
(z.\,B.\ Kapitel"=Überschriften, Zitate, Beispiele, Formeln
\dots) meistens aufgrund seines Fachwissens aus dem Inhalt des
Manuskripts.  \LaTeX{} dagegen ist "`nur"' ein Programm und
benötigt daher zusätzliche Informationen vom Autor, die die
logische Struktur des Textes beschreiben.
Diese Informationen werden in Form von sogenannten "`Befehlen"'
innerhalb des Textes angegeben.
Der Autor braucht sich also
(weitgehend) nur um die logische Struktur seines Werkes zu kümmern,
nicht um die Details von Gestaltung und Satz.
 
Im Gegensatz dazu steht der visuell orientierte Entwurf eines
Schriftstückes mit Textverarbeitungs- oder \textsc{dtp}-Programmen wie z.\,B.\ 
\textsc{Word}.
In diesem Fall legt der Autor das Layout des Textes gleich bei der
interaktiven Eingabe fest. Dabei sieht er am Bildschirm das, was
auch auf der gedruckten Seite stehen wird. Solche Systeme, die das
visuelle Entwerfen unterstützen, werden auch \textsc{wysiwyg}-Systeme
("`what you see is what you get"') genannt.
 
Bei \LaTeX{} sieht der Autor beim Schreiben des Eingabefiles in
der Regel noch nicht sofort, wie der Text nach dem Formatieren 
aussehen wird. Er kann aber %durch Aufruf des entsprechenden Programms 
jederzeit einen "`Probe-Ausdruck"' seines Schriftstücks auf dem
Bildschirm machen und danach sein Eingabefile entsprechend 
korrigieren und die Arbeit fortsetzen.
 
 
\subsubsection{Layout-Design}
 
Typographisches Design ist ein Handwerk, das erlernt werden muss.
Ungeübte Autoren machen dabei oft gravierende Fehler.
Fälschlicherweise glauben viele Laien, dass Textdesign
vor allem eine Frage der Ästhetik ist -- wenn das
Schriftstück vom künstlerischen Standpunkt aus "`schön"'
aussieht, dann ist es schon gut "`designed"'.
Da Schriftstücke jedoch gelesen und nicht in einem Museum
aufgehängt werden, sind die leichtere Lesbarkeit und bessere
Verständlichkeit wichtiger als das schöne Aussehen.
 
Beispiele:
Die Schriftgröße und Nummerierung von Überschriften soll so
gewählt werden, dass die Struktur der Kapitel und Unterkapitel
klar erkennbar ist.
Die Zeilenlänge soll so gewählt werden, dass anstrengende
Augenbewegungen des Lesers vermieden werden, nicht so, dass der
Text das Papier möglichst schön ausfüllt.
 
Mit interaktiven visuellen Entwurfssystemen ist es leicht,  
Schriftstücke zu erzeugen, die zwar "`gut"' aussehen,
aber ihren Inhalt und dessen Aufbau nur mangelhaft wiedergeben.
\LaTeX{} verhindert solche
Fehler, indem es den Autor dazu zwingt, die logische
Struktur des Textes anzugeben, und dann automatisch ein dafür
geeignetes Layout verwendet.

Daraus ergibt sich, dass \LaTeX{} insbesondere für  Dokumente geeignet 
ist, wo vorgegebene Gestaltungsprinzipien auf sich wiederholende
logische Textstrukturen angewandt werden sollen. 
Für das -- notwendigerweise -- visuell orientierte Gestalten
etwa eines Plakates ist \LaTeX{} hingegen 
aufgrund seiner Arbeitsweise weniger geeignet.

\subsubsection{Vor- und Nachteile}

Gegenüber anderen Textverarbeitungs- oder \textsc{dtp}-Programmen 
zeichnet sich \LaTeX{}
vor allem durch die folgenden Vorteile aus:
\begin{itemize}
\item Der Anwender muss nur wenige, leicht verständliche Befehle
  angeben, die die logische Struktur des Schriftstücks
  betreffen, und braucht sich um die gestalterischen Details
  (fast) nicht zu kümmern.
\item Das Setzen von mathematischen Formeln ist besonders gut
  unterstützt.
\item Auch anspruchsvolle Strukturen wie Fußnoten, Literaturverzeichnisse,
  Tabellen u.\,v.\,a.\  können mit wenig Aufwand erzeugt werden.
% ---- schwammige Formulierung ;-)
\item Routineaufgaben wie das Aktualisieren von Querverweisen
 oder das Erstellen des Inhaltsverzeichnisses 
 werden automatisch erledigt.
\item Es stehen zahlreiche vordefinierte Layouts zur Verfügung.
\item \LaTeX-Dokumente sind zwischen verschiedenen Installationen und
 Rechnerplattformen austauschbar.
\item Im Gegensatz zu vielen \textsc{wysiwyg}-Programmen bearbeitet \LaTeX{} auch
  lange oder komplizierte Dokumente zuverlässig,
  und sein Ressourcenverbrauch (Speicher, Rechenleistung) ist vergleichsweise
  mäßig.
\end{itemize}
Ein Nachteil soll freilich auch nicht verschwiegen werden:
\begin{itemize}
\item Innerhalb der von \LaTeX\ unterstützten Dokument"=Layouts
  können zwar einzelne Parameter leicht variiert werden,
  grundlegende Abweichungen von den Vorgaben sind
  aber nur mit größerem Aufwand möglich (Design einer
  neuen Dokumentklasse, siehe~\cite{clsguide,lay,lay2,typografie}.)
\todo{MD: Ich denke, dass war früher so. Heutzutage gibt es für fast alles Pakete}
\end{itemize}

\subsubsection{Der Arbeitsablauf}
Der typische Ablauf beim Arbeiten mit \LaTeX{} ist:
\begin{enumerate}
  \item Ein Eingabefile schreiben, das den Text und die \LaTeX-Befehle 
  enthält.
  \item Dieses File mit \LaTeX{} bearbeiten; dabei wird eine Datei
  erzeugt, die den gesetzten Text in einem geräteunabhängigen Format
  (\textsc{dvi}, \textsc{pdf} oder auch PostScript) enthält.
  \item Einen "`Probeausdruck"' davon auf dem Bildschirm anzeigen (Preview).
  \item Wenn nötig, die Eingabe korrigieren und zurück zu Schritt~2.
  \item Die Ausgabedatei drucken.
\end{enumerate}
Zeitgemäße Betriebssysteme machen es möglich, dass der Texteditor
und das Preview-Programm gleichzeitig in verschiedenen Fenstern 
"`geöffnet"' sind; beim Durchlaufen des obigen Zyklus brauchen sie 
also nicht immer wieder von neuem gestartet werden.  Nur die 
wiederholte \LaTeX-Bearbeitung des Textes muss noch von Hand 
angestoßen werden und läuft ebenfalls in einem eigenen Fenster ab.
% Danach sollte das Preview-Programm -- im 
% Idealfall -- selbsttätig das veränderte Ergebnis anzeigen; ansonsten 
% kennt es normalerweise einen Menüpunkt oder eine Schaltfläche, um das 
% geänderte Ausgabefile erneut zu laden und anzuzeigen.

Wie man auf die einzelnen Programme -- Editor, \LaTeX, Previewer,
Druckertreiber -- in einer bestimmten
Betriebssystemumgebung zugreift, muss in einem \local{}
beschrieben sein.
\todo{PG: Die Verweise auf den local guide nerven ein wenig, die sollte m.E. raus.\\MD: Richtig}
\todo{MD: Hier sollte auf die Kompilierung selbst eingegangen werden. PG: OK.}


