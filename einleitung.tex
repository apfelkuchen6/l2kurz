\begin{titlepage}
\renewcommand{\thefootnote}{\fnsymbol{footnote}}
{\Huge%
\fontfamily{cmss}\fontseries{sbc}\selectfont
\raggedright
\sbLaTeXe-Kurzbeschreibung
\rule{\textwidth}{0.75pt}
\par
}
\begin{flushleft}
  \normalsize
  \fontfamily{cmss}\fontseries{sbc}\selectfont
  Version \lkver\\
  \lkdate\\[2ex]
  Walter Schmidt\footnote{\texttt{<w-a-schmidt@arcor.de>}}\\
  Jörg Knappen\\
  Hubert Partl%\footnote{Zentraler Informatikdienst der Universität für Bodenkultur, Wien}
    \\
  Irene Hyna%\footnote{Bundesministerium für Wissenschaft und Verkehr, Wien}
  \\
\end{flushleft}

\vfill

{\parindent=0cm
\LaTeX{} ist ein Satzsystem, das für viele Arten von
Schriftstücken verwendet werden kann, von einfachen Briefen bis zu
kompletten Büchern.  Besonders geeignet ist es für 
wissenschaftliche oder technische Dokumente. \LaTeX{} ist für 
praktisch alle verbreiteten Betriebssysteme verfügbar.
 
Die vorliegende Kurzbeschreibung bezieht sich auf die Version
\LaTeXe\ in der Fassung vom Juni~2001 und sollte für den 
Einstieg in \LaTeX{} ausreichen.  
Eine vollständige Beschreibung enthält das \manual{}
in Verbindung mit der Online-Dokumentation.
}
\setcounter{footnote}{0}
\end{titlepage}


{\parindent=0cm\thispagestyle{empty}

Copyright \copyright{} 1998--2003 W.~Schmidt, J.~Knappen, H.~Partl, I.~Hyna

\bigskip

{\selectlanguage{USenglish}
  Permission is granted to copy, distribute and/or modify this document
  under the terms of the GNU Free Documentation License, Version~1.2
  or any later version published by the Free Software Foundation;
  with no Invariant Sections, no Front-Cover Texts, and no Back-Cover Texts.
  A copy of the license is included in the section entitled ``GNU
  Free Documentation License''.\todo{PG: m.E. sollte hier eine URL stehen und dann der Text hinten raus.}
}

\bigskip

Die in dieser Publikation erwähnten Software- und Hardware"=Bezeichnungen sind
in den meisten Fällen auch eingetragene Warenzeichen und unterliegen als
solche den gesetzlichen Bestimmungen.

\bigskip

\vfill

Dieses Dokument wurde mit \LaTeX{} gesetzt.
Es ist als Quelltext und im PDF-Format online erhältlich:
\begin{quote}
\url{http://mirror.ctan.org/info/lshort/german/}
\end{quote}
\bigskip

Die Autoren bedanken sich bei
Luzia Dietsche, 
Michael Hofmann, 
Peter Karp,
Rolf \mbox{Niepraschk},
Heiko Oberdiek,
Bernd Raichle, 
Rainer Schöpf und
Stefan Steffens
für Tipps, Anmerkungen und  Korrekturen.

}
