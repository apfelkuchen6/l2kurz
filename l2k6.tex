% master: l2kurz.tex
% l2k6.tex - 6.Teil der LaTeX2e-Kurzbeschreibung v2.3
% 2003-04-10 (WaS)

\section{Spezialit"aten}
 
Das komplette Men"u der Spezialit"aten, die von \LaTeX\ serviert
werden, ist im \manual\ und in der Online-Dokumentation beschrieben.
Hier soll nur auf einige besondere "`Schmankerln"' hingewiesen
werden.
 
\subsection{Abst"ande}

\subsubsection{Zeilenabstand}

Um in einem Schriftst"uck gr"o"sere Zeilen"-ab"-st"ande zu verwenden,
als es in der Dokumentklasse vorgesehen ist, gibt es in
\LaTeX\ den Befehl \verb:\linespread:, der im Vorspann stehen sollte
und dann auf das gesamte Dokument wirkt.  Das kann beispielsweise
dann notwendig werden, wenn eine Schrift benutzt wird, die eine gr"o"serer x-H"ohe 
hat als die voreingestellte Computer-Modern.  F�r die Schrift "`Palatino"' etwa
ist eine Vergr"o"serung des Zeilenabstandes um ca.\ 5\,\% angemessen:

\begin{quote}
\verb|\usepackage{mathpazo}|\\
\verb|\linespread{1.05}|
\end{quote}

 
 
 
\subsubsection{Spezielle horizontale Abst"ande}\label{abst:horiz}
 
Die Abst"ande zwischen W"ortern und S"atzen werden von \LaTeX\ 
automatisch gesetzt.
Sonstigen horizontalen Ab"-stand kann man mit dem Befehl
\begin{verse}
\verb|\hspace{|\textit{l"ange}\verb|}|
\end{verse}
einf"ugen.
Wenn der Abstand auch am Beginn oder Ende einer Zeile
erhalten bleiben soll, mu"s \verb|\hspace*| statt \verb|\hspace|
geschrieben werden.
Die L"angen"-angabe besteht im einfachsten Fall aus einer Zahl
und einer Einheit.  Die wichtigsten Einheiten sind in
Tabelle~\ref{units} an"-ge"-f"uhrt.
\begin{table}[b]
\caption{Einheiten f"ur L"angenangaben} \label{units}
\oben{11cm}
\begin{tabbing}
\texttt{mm}\qquad \= Millimeter                               \\
\texttt{cm} \> Zentimeter = 10\,mm                            \\
\texttt{in} \> inch \(= 25.4\,\mathrm{mm} \)                  \\
\texttt{pt} \> point \( =(1/72.27)\,\mathrm{in}
                        \approx 0.351\,\mathrm{mm}\)          \\
\texttt{bp} \> big point \( =(1/72)\,\mathrm{in}
                            \approx 0.353\,\mathrm{mm} \)      \\
%\texttt{dd} \> Didot-Punkt \( = (1238/1157)\,\mathrm{pt}
%                              \approx 0.376\,\mathrm{mm} \)   \\
% --- wegen unklarer Definition (0.375 ider 0.376mm) besser nicht benutzen!
\texttt{em} \> Geviert (doppelte Breite einer Ziffer der aktuellen Schrift)\\
\texttt{ex} \> H"ohe des Buchstabens x der aktuellen Schrift
\end{tabbing}                    
\unten
\end{table}
Die Befehle in Tabelle~\ref{hspace} sind Abk"urzungen zum Einf"ugen
besonderer horizontaler Ab"-st"ande.
\begin{table}[t]
\caption{Befehle f"ur horizontale Abst"ande} \label{hspace}
\oben{13cm}
\begin{tabbing}
\texttt{xenspace}\qquad \= \kill
\verb|\,|       \> ein sehr kleiner Abstand (siehe auch Abschnitt~\ref{abstaende})\\
\verb|\enspace| \> so breit wie eine Ziffer \\
\verb|\quad|    \> so breit, wie ein Buchstabe hoch ist
                   ("`wei"ses Quadrat"') \\
\verb|\qquad|   \> doppelt so breit wie ein \verb|\quad| \\
\verb|\hfill|   \> ein Abstand, der sich von 0 bis \(\infty\)
                   ausdehnen kann.
\end{tabbing}
\unten
\end{table}
Der Befehl \verb|\hfill| kann dazu dienen, einen vorgegebenen
Platz aus"-zu"-f"ullen.
\exa
\raggedright
Schafft mir\hspace{1.5cm}Raum! \\
\(\triangleleft\)\hfill \(\triangleright\)\\
\exb
\begin{verbatim}
Schafft mir\hspace{1.5cm}Raum! \\
\(\triangleleft\)\hfill 
\(\triangleright\)
\end{verbatim}
\exc


\subsubsection{Spezielle vertikale Abst"ande} \label{vabstaende}
 
Die Abst"ande zwischen Ab"-s"atzen, Kapiteln usw.\ werden von
\LaTeX\ automatisch bestimmt.
In Spezial"-f"allen kann man zu"-s"atz"-lichen Ab"-stand
\emph{zwischen zwei Ab"-s"atzen} mit dem Befehl
\begin{verse}
\verb|\vspace{|\textit{l"ange}\verb|}|
\end{verse}
bewirken.
Dieser Befehl sollte immer zwischen zwei Leerzeilen angegeben
werden.
Wenn der Abstand auch am Beginn oder Ende einer Seite erhalten
bleiben soll, mu"s \verb|\vspace*| statt \verb|\vspace|
geschrieben werden.
Die Befehle in Tabelle~\ref{vspace} sind Abk"urzungen f"ur
bestimmte vertikale Ab"-st"ande.
\begin{table}[t]
\caption{Befehle f"ur vertikale Abst"ande} \label{vspace}
\oben{13cm}
\begin{tabbing}
\texttt{xsmallskip}\qquad \= \kill
\verb|\smallskip| \> etwa \(\nfrac{1}{4}\) Zeile \\
\verb|\medskip|   \> etwa \(\nfrac{1}{2}\) Zeile \\
\verb|\bigskip|   \> etwa 1 Zeile \\
\verb|\vfill|     \> ein Abstand, der sich von 0 bis \(\infty\)
                     ausdehnen kann
\end{tabbing}
\unten
\end{table}
Der Befehl \verb|\vfill| in Verbindung mit \verb|\newpage|
kann dazu dienen, Text an den unteren Rand einer Seite zu setzen
oder vertikal zu zentrieren.  Beispielsweise enth"alt der Quelltext
f"ur die zweite Seite der vorliegenden Beschreibung:
\begin{quote}
\begin{verbatim}
\vfill

Dieses Dokument wurde mit \LaTeX{} gesetzt.
...
\newpage
\end{verbatim}
\end{quote}
 
Zus"atzlichen Abstand zwischen zwei Zeilen \emph{innerhalb}
eines Absatzes oder einer Tabelle erreicht man mit dem Befehl
\verb|\\[|\textit{l"ange}\verb|]|.
\exa
Albano Cesara \\
Lindenallee 10 \\[1.5ex]
95632 Pestitz
\exb
\begin{verbatim}
Albano Cesara \\
Lindenallee 10 \\[1.5ex]
95632 Pestitz
\end{verbatim}
\exc

\smallskip
 
 
\subsection{Briefe}\label{briefe}
 
Mit der Dokumentklasse \texttt{letter} kann man zwischen
\verb|\begin{document}| und \verb|\end{document}| einen oder
mehrere Briefe schreiben. 
Abbildung~\ref{brief} enth"alt ein Beispiel f"ur einen Brief.

\begin{figure}[ht] %\small
\oben{11cm}
\begin{alltt}
\verb+\documentclass[12pt,a4paper]{letter}+
\verb+\usepackage[latin1]{inputenc}+
\verb+\usepackage{german}+
\verb+\address{EDV-Zentrum der TU Wien \\+
\verb+  Abt. Digitalrechenanlage \\+
\verb+  +Wiedner Hauptstra\ss{}e 8--10 \verb+\\+
\verb+  A-1040 Wien}+
\verb+\signature{Dr. Hubert Partl}+
\verb+\begin{document}+
\verb+\begin{letter}{Frau Mag. Elisabeth Schlegl \\+
\verb+  +EDV-Zentrum der Karl-Franzens-Universit\"at \verb+\\+
\verb+  Attemsgasse 25/II \\+
\verb+  \textbf{A-8010 Graz}}+
\verb+\opening{Liebe Frau Schlegl,}+
herzlichen Dank f\"ur die Zusendung \dots

\dots in etwa 2--3~Wochen fertig zu sein.
\verb+\closing{+Mit freundlichen Gr\"u\ss{}en\verb+}+
\verb+\end{letter}+
\verb+\end{document}+
\end{alltt}
\unten
\caption{Brief von H.\,P. an E.\,S.} \label{brief}
\end{figure}

\begin{sloppypar}
Mit dem Befehl \verb|\address| definiert man die Adresse des Absenders.
\verb|\begin{letter}{...}| beginnt einen Brief an den im
Parameter angegebenen Empf"anger.
\verb|\opening{...}| schreibt die Anrede 
und \verb|\closing{...}| den abschlie"senden Gru"s, 
an den automatisch die eingangs mit
\verb|\signature| vereinbarte Unterschrift an"-ge"-f"ugt wird.
\verb|\end{letter}| beendet den jeweiligen Brief.
\end{sloppypar}

Das von der Dokumentklasse \texttt{letter} bewirkte Layout der Briefe 
orientiert sich an amerikanischen Gepflogenheiten.
Mit vielen \LaTeX-Systemen ist die Klasse 
\texttt{dinbrief} verf"ugbar; sie setzt die Briefe in einer
Anordnung gem"a"s DIN~676, 
die f"ur die Verwendung von A4-Bogen in Fensterkuverts geeignet ist.
Der \local{} sollte Auskunft "uber diese oder andere Alternativen zu
\texttt{letter} geben.

\subsection{Literaturangaben}

Mit der \texttt{thebibliography}-Umgebung kann man ein
Literaturverzeichnis erzeugen.
Darin beginnt jede Literaturangabe mit \verb|\bibitem|.
Als Parameter wird ein Name vereinbart, unter dem die
Literaturstelle im Text zitiert werden kann, und
dann folgt der Text der Literaturangabe.
Die Numerierung erfolgt automatisch.
Der Parameter bei \verb|\begin{thebibliography}| gibt die
maximale Breite dieser Nummern"-angabe an, also z.\,B.\ 
\verb|{99}| f"ur maximal zweistellige Nummern.

Im Text zitiert man die Literaturstelle dann mit dem Befehl \verb|\cite|
und dem vereinbarten Namen als Argument.
\exa
Partl~\cite{pa} hat
vorgeschlagen, da"s \dots
 
\begin{thebibliography}{99}
\bibitem{pa}
H.~Partl: \textit{German \TeX,}
TUG\-boat Vol.~9, No.~1 (1988)
\end{thebibliography}
\exb
\begin{verbatim}
Partl~\cite{pa} hat
vorgeschlagen ...
 
\begin{thebibliography}{99}
\bibitem{pa}
H.~Partl: \textit{German \TeX,}
TUGboat Vol.~9, No.~1 (1988)
\end{thebibliography}
\end{verbatim}
\exc

 
\subsection{Zerbrechliche Befehle}
 
Manche \LaTeX-Befehle "`verfrachten"' ihre Argumente an eine andere
Stelle im Text; beispielsweise kann das Argument von \verb|\section|
auch im Inhaltsverzeichnis und m"oglicherweise in der Kopfzeile auftauchen.  

Bestimmte Befehle "`"uberstehen"' diesen Transport nicht, wenn sie
ohne besondere Ma"snahmen in einem solchen "`beweglichen Argument"'
auftreten.
Derartige Befehle hei"sen "`zerbrechlich"'.  Damit sie dennoch innnerhalb
von beweglichen Argumenten benutzt werden d"urfen, 
mu"s man ihnen einfach den Befehl \verb|\protect| voranstellen.

Zerbrechlich sind insbesondere alle Befehle, die ein optionales Argument
kennen, also auch \verb|\\| (sic!),
au"serdem die Befehle \verb|\(|, \verb|\)| und \verb|\footnote|.

Bewegliche Argumente haben, neben den Gliederungsbefehlen,
auch der Befehl \verb|\caption| und die Umgebung \texttt{letter}.


\endinput
