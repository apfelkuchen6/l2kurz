% master: l2kurz.tex
% l2k5.tex - 5.Teil der LaTeX2e-Kurzbeschreibung v2.3
% 2001-04-10 (WaS)

\section{Schriften}
Normalerweise w"ahlt \LaTeX\ die Gr"o"se und den Stil der Schrift
aufgrund der Befehle aus, die die logische Struktur des Textes angeben:
"Uber"-schriften, Fu"snoten, Hervorhebungen usw.
Im folgenden werden Befehle und Makropakete beschrieben, mit denen
die Schrift auch explizit beeinflu"st werden kann.
Ausf"uhrlichere Erl"auterungen zum Umgang mit Schriften in \LaTeX{}
findet man im \textit{\LaTeX-Begleiter} \cite{wonne}
und in der Online-Dokumentation \cite{fntguide}.



\subsection{Schriftgr"o"sen}
 
Die in der Tabelle~\ref{sizes} an"-ge"-f"uhrten Befehlen 
wechseln die Schriftgr"o"se.
Sie spezifizieren die Gr"o"se relativ
zu der von \verb:\documentclass: festgelegten Grundschrift.
Ihr Wirkung reicht bis zum Ende der aktuellen Gruppe oder Umgebung.


\begin{table}[hb]
\caption{Schriftgr"o"sen} \label{sizes}
\oben{10cm}
\begin{tabbing}
\texttt{xfootnotesizexx}\= und dann der Text \kill
\verb|\tiny|         \> \tiny        winzig kleine Schrift \\
\verb|\scriptsize|   \> \scriptsize  sehr kleine Schrift (wie Indizes)\\
\verb|\footnotesize| \> \footnotesize     kleine Schrift (wie Fu"snoten)\\
\verb|\small|        \> \small            kleine Schrift \\
\verb|\normalsize|   \> \normalsize  normale Schrift \\
\verb|\large|        \> \large       gro"se Schrift \\
\verb|\Large|        \> \Large       gr"o"sere Schrift \\
\verb|\LARGE|        \> \LARGE       sehr gro"se Schrift \\[3pt]
\verb|\huge|         \> \huge        riesig gro"s \\[3pt]
\verb|\Huge|         \> \Huge        gigantisch
\end{tabbing}
\unten
\end{table}
 
Die Gr"o"sen-Befehle ver"andern auch die Zeilen"-ab"-st"ande auf
die jeweils passenden Werte -- aber nur, wenn die
Leerzeile, die den Absatz be\-en\-det, innerhalb des
G"ultigkeitsbereichs des Gr"o"sen-Befehls liegt:
\exa
{\Large zu enger\\
Abstand}\par
\exb
\begin{verbatim}
{\Large zu enger \\
Abstand}\par
\end{verbatim}
\exc
\exa
{\Large richtiger\\
Abstand\par}
\exb
\begin{verbatim}
{\Large richtiger\\
Abstand\par}
\end{verbatim}
\exc
F"ur korrekte
Zeilen"-ab"-st"ande darf die
schlie"-"sende geschwungene Klammer also nicht zu fr"uh kommen,
sondern erst nach einem Absatzende, das "ubrigens nicht nur als
Leerzeile, sondern auch als Befehl \verb|\par|  eingegeben werden 
kann.


\subsection{Schriftstil}
Der Schriftstil wird in \LaTeX{} durch 3~Merkmale definiert:
\begin{description}
\item[Familie] Standardm"a"sig stehen 3~Familien zur Wahl:
  "`roman"' (Antiqua), "`sans serif"' (Serifenlose) und "`typewriter"'
  (Schreibmaschinenschrift).
\item[Serie] Die Serie gibt St"arke und Laufweite der
  Schrift an: "`medium"' (normale Schrift), "`boldface extended"'
  (fett und breiter).
\item[Form] Die Form der Buchstaben: "`upright"'
  (aufrecht), "`slanted"' (geneigt), "`italic"' (kursiv),
  "`caps and small caps"' (Kapit"alchen).
\end{description}
Tabelle~\ref{fonts} zeigt die Befehle, mit denen diese Attribute 
explizit beeinflu"st werden k"onnen.  
Die Befehle der Form \verb|\text...| setzen nur ihr Argument im 
gew"unschten  Stil.  Zu jedem dieser Befehle ist ein Gegenst"uck angegeben, 
das von seinem Auf\/treten an bis zum Ende der laufenden Gruppe oder Umgebung 
wirkt.

Zu beachten ist, da"s W"orter in Schreibmaschinenschrift nicht automatisch
getrennt werden.\par

\begin{table}[hbp]
\caption{Schriftstile} \label{fonts}
\oben{10cm}
\begin{tabbing}\small
\verb|\textnormal|\{\textit{text}\}\qquad\=\verb|\normalfont|\qquad\=\kill
\verb|\textrm|\{\textit{text}\}         \>\verb|\rmfamily|       \>\textrm{Antiqua}\\
\verb|\textsf|\{\textit{text}\}         \>\verb|\sffamily|       \>\textsf{Serifenlose}\\
\verb|\texttt|\{\textit{text}\}         \>\verb|\ttfamily|       \>\texttt{Maschinenschrift}\\[1ex]
\verb|\textmd|\{\textit{text}\}         \>\verb|\mdseries|       \>\textmd{normal}\\
\verb|\textbf|\{\textit{text}\}         \>\verb|\bfseries|       \>\textbf{fett, breiter laufend}\\[1ex]
\verb|\textup|\{\textit{text}\}         \>\verb|\upshape|        \>\textup{aufrecht}\\
\verb|\textsl|\{\textit{text}\}         \>\verb|\slshape|        \>\textsl{geneigt}\\
\verb|\textit|\{\textit{text}\}         \>\verb|\itshape|        \>\textit{kursiv}\\
\verb|\textsc|\{\textit{text}\}         \>\verb|\scshape|        \>\textsc{Kapit"alchen}\\[1ex]
\verb|\textnormal|\{\textit{text}\}     \>\verb|\normalfont|     \>\textnormal{Die Grundschrift des Dokuments}
\end{tabbing}
\unten
\end{table}

Die Befehle f"ur Familie, Serie und Form k"onnen untereinander und mit den
Gr"o"sen-Befehlen kombiniert werden;  allerdings mu"s nicht jede
m"ogliche Kombination tats"achlich als reale Schrift (Font)
zur Verf"ugung stehen.
\exa
{\small Die kleinen
\textbf{fetten} R"omer
beherrschten }{\large das
ganze gro"se \textit{Italien}.}
\\[6ex]
{\Large\sffamily\slshape plakativ}
\exb
\begin{verbatim}
{\small Die kleinen
\textbf{fetten} R"omer
beherrschten }{\large das
ganze gro"se \textit{Italien}.}
{\Large\sffamily\slshape plakativ}
\end{verbatim}
\exc

Je \emph{weniger} verschiedene Schriftarten man verwendet, desto
lesbarer und sch"oner wird das Schrift"-st"uck!


\subsection{Andere Schriftfamilien}
Mit den im vorigen Abschnitt eingef"uhrten Befehlen kann man nicht beeinflussen,
welche Schriftfamilien tats"achlich als Antiqua, Serifenlose und
Maschinenschrift benutzt werden.  \LaTeX{} verwendet als Voreinstellung
die sog.\ Computer-Modern-Schriftfamilien (CM), siehe Tabelle~\ref{families};
der Stil der mathematischen Zeichens"atze pa"st dabei zu CM~Roman.

Will man andere Schriften benutzen, dann ist der einfachste Weg 
das Laden eines Pakets, das eine oder mehrere dieser Schriftfamilien 
komplett ersetzt.
Tabelle~\ref{families} f"uhrt einige derartige Pakete auf%,
% die allerdings nicht in jeder \LaTeX-Installation verf"ugbar sein m"ussen
.

Die Dokumentation Ihres \TeX-Systems \cite{local} sollte dar"uber
informieren, welche Schriften verf"ugbar sind
und wie Sie weitere installieren und verwenden k"onnen.
Insbsondere sollte eine Anzahl von verbreiteten PostScript-Schriften
mit jedem aktuellen \LaTeX-System verwendbar sein \cite{postscript}.

\begin{table}[htb]
\caption[Pakete f"ur alternative Schriftfamilien]
{Pakete f"ur alternative Schriftfamilien (Eine leere
Tabellenspalte bedeutet, da"s das Paket die betreffende Schriftfamilie nicht 
ver"andert; * kennzeichnet die jeweils als Grundschrift eingestellte Familie.)}
\label{families}
{\footnotesize
\begin{center}
\medskip
\renewcommand{\arraystretch}{1.5}
\begin{tabular}{|l|p{2.cm}p{2.2cm}p{2.4cm}p{2.2cm}|}
\hline
Paket            & Antiqua    & Serifenlose   & Schreibmaschine  & math.\ Formeln\\\hline\hline
(keines)         & CM Roman * & CM Sans Serif & CM Typewriter    & $\approx$ CM Roman\\\hline
\texttt{ccfonts} & Concrete *
                 &
                 &
                 & $\approx$ Concrete\\\hline
\texttt{cmbright}&
                 & CM Bright *
                 & {\raggedright CM\ Typewriter\\ Light}
                 & $\approx$ CM Bright\\\hline
% \texttt{pandora} & {\raggedright Pandora\\ Roman *} 
%                  & {\raggedright Pandora \\ Sans Serif} 
%                  &
%                  & \\\hline
\texttt{mathptmx}& Times *
                 &
                 &
                 & $\approx$ Times\\\hline
\texttt{mathpazo}& Palatino *
                 &
                 &
                 & $\approx$ Palatino\\\hline
\texttt{helvet}  & 
                 & Helvetica
                 &
                 & \\\hline
\texttt{courier} &
                 &
                 & Courier 
                 & \\\hline
\end{tabular}
\end{center}
}
\end{table}


\subsection{Die "`europ"aischen"' Zeichens"atze}
\LaTeX{} verwendet standardm"a"sig  Schriften mit einem Umfang von
128~Zeichen.  Umlaute oder akzentuierte Buchstaben sind darin nicht
enthalten; sie werden jeweils aus dem Grundsymbol und dem Akzent
zusammengesetzt.  

Inzwischen stehen die meisten der mit \LaTeX\ verwendbaren Schriften
auch mit einem erweiterten "`europ"aischen"' Zeichenvorrat bereit.
Sie enthalten jetzt 256 Zeichen, welche fast
alle europ"aischen Sprachen abdecken, d.\,h., jedes be\-n"o\-tig\-te
Zeichen ist vorgefertigt in ihnen enthalten.
Das hat nicht nur eine
h"ohere typographische Qualit"at zur Folge; aufgrund der inneren Arbeitsweise
von \TeX{} entfallen damit auch die Einschr"ankungen im Zusammenhang mit
der Silbentrennung, die im Abschnitt~\ref{silb} erw"ahnt wurden:
W"orter mit Umlauten werden nun besser getrennt, und im Argument des
Befehls \verb|\hyphenation| d"urfen auch Umlaute und das scharfe~s stehen.

%% stimmt nur fuer EC
%Weiterhin sind die Unterschneidungen im Vergleich zu den amerikanischen
%\TeX-Originalschriften stark verbessert und nun auch auf h"aufige
%Buchstabenpaarungen in nicht-englischen Sprachen optimiert.

% <------- Formulierung ????
Die europ"aischen Schriften bestehen aus zwei Teilen: Der T1-Zeichensatz
enth"alt Buchstaben, ASCII-Zeichen sowie verschiedene Anf"uhrungszeichen
und Striche, 
w"ahrend ein erg"anzender TS1-Zeichensatz zu\-s"atz\-liche Textsymbole bereitstellt.
% <------- Formulierung ????

\LaTeX{} wird veranla"st, T1-Schriften zu verwenden,
indem man das Paket \texttt{fontenc} mit der Option \texttt{T1} l"adt:
\begin{quote}
  \verb|\usepackage[T1]{fontenc}|
\end{quote}
Das Paket \texttt{textcomp} erm"oglicht den Zugriff auf die Textsymbole:
\begin{quote}
  \verb|\usepackage{textcomp}|
\end{quote}
Welche zus"atzlichen Zeichen mit den T1-Schriften
bereitgestellt werden, ist in \cite{usrguide} zusammengefa"st;
Anhang~\ref{textsymbols} der vorliegenden Kurzbeschreibung
enth"alt eine Liste aller TS1-Textsymbole.  Einige der Textsymbole sind
auch ohne das Paket \texttt{textcomp} verf"ugbar, siehe Abschnitt~\ref{symbole},
dann aber nicht immer in einem zur laufenden Schrift passenden Stil.

Beachten Sie, da"s in Fonts, die nicht speziell f"ur die Verwendung 
mit \TeX\ entworfen wurden, 
nur ein Teil der TS1-Textsymbole enthalten ist.
Das betrifft vor allem die "`handels"ublichen"' PostScript-Schriften.

\endinput

