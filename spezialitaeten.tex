% master: l2kurz.tex
% l2k6.tex - 6.Teil der LaTeX2e-Kurzbeschreibung v2.3
% 2003-04-10 (WaS)

\section{Spezialitäten}
 
Das komplette Menü der Spezialitäten, die von \LaTeX\ serviert
werden, ist im \manual\ und in der Online-Dokumentation beschrieben.
Hier soll nur auf einige besondere "`Schmankerln"' hingewiesen
werden.
\todo[inline]{PG: vielleicht anstelle des Brief-Beispiels ein kleines TikZ Beispiel einbringen, so als Appetithäppchen? Das würde auch die Textlastigkeit von l2kurz etwas aufbrechen. \newline MD: Spricht den Leser auch eher an. }
 
\subsection{Abstände}

\subsubsection{Zeilenabstand}

Um in einem Schriftstück größere Zeilenabstände zu verwenden,
als es in der Dokumentklasse vorgesehen ist, gibt es in
\LaTeX\ den Befehl \verb:\linespread:, der im Vorspann stehen sollte
und dann auf das gesamte Dokument wirkt.  Das kann beispielsweise
dann notwendig werden, wenn eine Schrift benutzt wird, die eine größerer x-Höhe 
hat als die voreingestellte Computer-Modern.  Für die Schrift "`Palatino"' etwa
ist eine Vergrößerung des Zeilenabstandes um ca.\ 5\,\% angemessen:\todo{MD: Paket setspace\\PG: gut}

\begin{quote}
\verb|\usepackage{mathpazo}|\\
\verb|\linespread{1.05}|
\end{quote}

 
 
 
\subsubsection{Spezielle horizontale Abstände}\label{abst:horiz}
 
Die Abstände zwischen Wörtern und Sätzen werden von \LaTeX\ 
automatisch gesetzt.
Sonstigen horizontalen Abstand kann man mit dem Befehl
\begin{beispiel}
\todo{MD: lieber die Sternvariante als Standard empfehlen\newline PG: OK, ich hab' davon keine Ahnung}
\verb|\hspace{|\textit{länge}\verb|}|
\end{beispiel}
einfügen.
Wenn der Abstand auch am Beginn oder Ende einer Zeile
erhalten bleiben soll, muss \verb|\hspace*| statt \verb|\hspace|
geschrieben werden.
Die Längenangabe besteht im einfachsten Fall aus einer Zahl
und einer Einheit.  Die wichtigsten Einheiten sind in
Tabelle~\ref{units} angeführt.
\begin{table}[b]
\caption{Einheiten für Längenangaben} \label{units}
\begin{lminipage}{11cm}
\begin{tabbing}
\texttt{mm}\qquad \= Millimeter                               \\
\texttt{cm} \> Zentimeter = 10\,mm                            \\
\texttt{in} \> inch \(= 25.4\,\mathrm{mm} \)                  \\
\texttt{pt} \> point \( =(1/72.27)\,\mathrm{in}
                        \approx 0.351\,\mathrm{mm}\)          \\
\texttt{bp} \> big point \( =(1/72)\,\mathrm{in}
                            \approx 0.353\,\mathrm{mm} \)      \\
%\texttt{dd} \> Didot-Punkt \( = (1238/1157)\,\mathrm{pt}
%                              \approx 0.376\,\mathrm{mm} \)   \\
% --- wegen unklarer Definition (0.375 ider 0.376mm) besser nicht benutzen!
\texttt{em} \> Geviert (doppelte Breite einer Ziffer der aktuellen Schrift)\\
\texttt{ex} \> Höhe des Buchstabens x der aktuellen Schrift
\end{tabbing}                    
\end{lminipage}
\end{table}
Die Befehle in Tabelle~\ref{hspace} sind Abkürzungen zum Einfügen
besonderer horizontaler Abstände.
\begin{table}[t]
\caption{Befehle für horizontale Abstände} \label{hspace}
\begin{lminipage}{13cm}
\begin{tabbing}
\texttt{xenspace}\qquad \= \kill
\verb|\,|       \> ein sehr kleiner Abstand (siehe auch Abschnitt~\ref{abstaende})\\
\verb|\enspace| \> so breit wie eine Ziffer \\
\verb|\quad|    \> so breit, wie ein Buchstabe hoch ist
                   ("`weißes Quadrat"') \\
\verb|\qquad|   \> doppelt so breit wie ein \verb|\quad| \\
\verb|\hfill|   \> ein Abstand, der sich von 0 bis \(\infty\)
                   ausdehnen kann.
\end{tabbing}
\end{lminipage}
\end{table}
Der Befehl \verb|\hfill|\todo{MD: Oder \texttt{\string\hspace*\{\string\fill\}} kann dazu dienen, einen vorgegebenen}
Platz auszufüllen.
\exa
\raggedright
Schafft mir\hspace{1.5cm}Raum! \\
\(\triangleleft\)\hfill \(\triangleright\)\\
\exb
\begin{verbatim}
Schafft mir\hspace{1.5cm}Raum! \\
\(\triangleleft\)\hfill 
\(\triangleright\)
\end{verbatim}
\exc


\subsubsection{Spezielle vertikale Abstände} \label{vabstaende}
 
Die Abstände zwischen Absätzen, Kapiteln usw.\ werden von
\LaTeX\ automatisch bestimmt.
In Spezialfällen kann man zusätzlichen Abstand
\emph{zwischen zwei Absätzen} mit dem Befehl
\begin{beispiel}
\verb|\vspace{|\textit{länge}\verb|}|
\end{beispiel}
bewirken.
Dieser Befehl sollte immer zwischen zwei Leerzeilen angegeben
werden.
Wenn der Abstand auch am Beginn oder Ende einer Seite erhalten
bleiben soll, muss \verb|\vspace*| statt \verb|\vspace|
geschrieben werden.
Die Befehle in Tabelle~\ref{vspace} sind Abkürzungen für
bestimmte vertikale Abstände.
\begin{table}[t]
\caption{Befehle für vertikale Abstände} \label{vspace}
\begin{lminipage}{13cm}
\begin{tabbing}
\texttt{xsmallskip}\qquad \= \kill
\verb|\smallskip| \> etwa \(\nfrac{1}{4}\) Zeile \\
\verb|\medskip|   \> etwa \(\nfrac{1}{2}\) Zeile \\
\verb|\bigskip|   \> etwa 1 Zeile \\
\verb|\vfill|     \> ein Abstand, der sich von 0 bis \(\infty\)
                     ausdehnen kann
\end{tabbing}
\end{lminipage}
\end{table}
Der Befehl \verb|\vfill| \todo{MD: siehe hspace\\PG: gut} in Verbindung mit \verb|\newpage|
kann dazu dienen, Text an den unteren Rand einer Seite zu setzen
oder vertikal zu zentrieren.  Beispielsweise enthält der Quelltext
für die zweite Seite der vorliegenden Beschreibung:
\begin{quote}
\begin{verbatim}
\vfill

Dieses Dokument wurde mit \LaTeX{} gesetzt.
...
\newpage
\end{verbatim}
\end{quote}
 
Zusätzlichen Abstand zwischen zwei Zeilen \emph{innerhalb}
eines Absatzes oder einer Tabelle erreicht man mit dem Befehl
\verb|\\[|\textit{länge}\verb|]|.
\exa
Albano Cesara \\
Lindenallee 10 \\[1.5ex]
95632 Pestitz
\exb
\begin{verbatim}
Albano Cesara \\
Lindenallee 10 \\[1.5ex]
95632 Pestitz
\end{verbatim}
\exc

\smallskip
 
 
\subsection{Briefe}\label{briefe}
 
\todo[inline]{PG: Briefe würde ich komplett rauslassen. Das macht doch eh kein Anfänger!\newline MD: Meine Meinung}

Mit der Dokumentklasse \texttt{letter} kann man zwischen
\verb|\begin{document}| und \verb|\end{document}| einen oder
mehrere Briefe schreiben.
Abbildung~\ref{brief} enthält ein Beispiel für einen Brief.

\begin{figure}[ht] %\small
\begin{lminipage}{11cm}
\begin{verbatim}
\documentclass[12pt,a4paper]{letter}
\usepackage[latin1]{inputenc}
\usepackage{german}
\address{EDV-Zentrum der TU Wien \\
  Abt. Digitalrechenanlage \\
  Wiedner Hauptstraße 8--10 \\
  A-1040 Wien}
\signature{Dr. Hubert Partl}
\begin{document}
\begin{letter}{Frau Mag. Elisabeth Schlegl \\
  EDV-Zentrum der Karl-Franzens-Universität \\
  Attemsgasse 25/II \\
  \textbf{A-8010 Graz}}
\opening{Liebe Frau Schlegl,}
herzlichen Dank für die Zusendung ...

... in etwa 2--3~Wochen fertig zu sein.
\closing{Mit freundlichen Grüßen}
\end{letter}
\end{document}
\end{verbatim}
\end{lminipage}
\caption{Brief von H.\,P. an E.\,S.} \label{brief}
\end{figure}

Mit dem Befehl \verb|\address| definiert man die Adresse des Absenders.
\verb|\begin{letter}{...}| beginnt einen Brief an den im
Parameter angegebenen Empfänger.
\verb|\opening{...}| schreibt die Anrede 
und \verb|\closing{...}| den abschließenden Gruß, 
an den automatisch die eingangs mit
\verb|\signature| vereinbarte Unterschrift angefügt wird.
\verb|\end{letter}| beendet den jeweiligen Brief.

Das von der Dokumentklasse \texttt{letter} bewirkte Layout der Briefe 
orientiert sich an amerikanischen Gepflogenheiten.
Mit vielen \LaTeX-Systemen ist die Klasse 
\texttt{dinbrief} verfügbar; sie setzt die Briefe in einer
Anordnung gemäß DIN~676, 
die für die Verwendung von A4-Bogen in Fensterkuverts geeignet ist.
Der \local{} sollte Auskunft über diese oder andere Alternativen zu
\texttt{letter} geben.

\subsection{Literaturangaben}

Mit der \texttt{thebibliography}-Umgebung kann man ein
Literaturverzeichnis erzeugen.
Darin beginnt jede Literaturangabe mit \verb|\bibitem|.
Als Parameter wird ein Name vereinbart, unter dem die
Literaturstelle im Text zitiert werden kann, und
dann folgt der Text der Literaturangabe.
Die Nummerierung erfolgt automatisch.
Der Parameter bei \verb|\begin{thebibliography}| gibt die
maximale Breite dieser Nummernangabe an, also z.\,B.\ 
\verb|{99}| für maximal zweistellige Nummern.

Im Text zitiert man die Literaturstelle dann mit dem Befehl \verb|\cite|
und dem vereinbarten Namen als Argument.


\begin{LTXexample}
Partl~\cite{pa} hat
vorgeschlagen ...
 
\begin{thebibliography}{99}
\bibitem{pa}
H.~Partl: \textit{German \TeX,}
TUGboat Vol.~9, No.~1 (1988)
\end{thebibliography}
\end{LTXexample}
\todo{PG: ltxexample verschluckt das \cs{cite} :(\\MD: Sehe ich gerade :-( -- Muss ich mir was einfallen lassen}

\todo[inline]{PG: Üblicherweise wird dafür bibtex oder biblatex/biber benutzt. Ich denke mir, dass wir zumindest auf biblatex (da gibts 'ne Deutsche Übersetzung von) eingehen sollten bzw. erwähnen sollten. \newline Erweitert aber den Text ;-)\newline PG: stimmt auch wieder. Es ist ja 'ne Kurzeinführung und kein Lehrbuch. Also erst mal drin lassen und dann können wir immer noch schauen?}
 
\subsection{Zerbrechliche Befehle}
\todo[inline]{Ich finden den Abschnitt ganz gut, aber ein Beispiel wäre nicht schlecht. Vielleicht zeigen, dass eine Formel mit \cs{(}..\cs{)} nicht in Überschriften funktioniert oder vielleicht ein anderes \emph{griffiges} Beispiel? Mir fällt leider keines ein.\newline MD: Es hat nur auswirkungen, wenn man expandiert bzw. in eine Datei schreibt. Inhaltsverzeichnis ist dabei griffiger, als ein \string\edef }
\todo[inline]{PG: Wenn ich so sehe, was wir alles noch mit reinnehmen wollen, dann würde ich auf diesen Abschnitt auch verzichten.}
 
Manche \LaTeX-Befehle "`verfrachten"' ihre Argumente an eine andere
Stelle im Text; beispielsweise kann das Argument von \verb|\section|
auch im Inhaltsverzeichnis und möglicherweise in der Kopfzeile auftauchen.  

Bestimmte Befehle "`überstehen"' diesen Transport nicht, wenn sie
ohne besondere Maßnahmen in einem solchen "`beweglichen Argument"'
auftreten.
Derartige Befehle heißen "`zerbrechlich"'.  Damit sie dennoch innnerhalb
von beweglichen Argumenten benutzt werden dürfen, 
muss man ihnen einfach den Befehl \verb|\protect| voranstellen.

Zerbrechlich sind insbesondere alle Befehle, die ein optionales Argument
kennen, also auch \verb|\\| (sic!),
außerdem die Befehle \verb|\(|, \verb|\)| und \verb|\footnote|.

Bewegliche Argumente haben, neben den Gliederungsbefehlen,
auch der Befehl \verb|\caption| und die Umgebung \texttt{letter}.


\endinput
