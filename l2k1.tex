% master: l2kurz.tex
% l2k1.tex - 1.Teil der LaTeX2e-Kurzbeschreibung v2.3
% 2003-04-10 (WaS)


\section{Allgemeines}
 
\subsection{The Name of the Game}
 
\subsubsection{\TeX}

\TeX\ (sprich "`Tech"', kann auch "`TeX"' geschrieben werden) ist
ein Computerprogamm von Donald E.~Knuth~\cite{texbook,schwarz}.
Es dient zum Setzen 
% und Drucken 
von Texten und mathematischen Formeln.
 
\subsubsection{\LaTeX}
 
\LaTeX\ (sprich "`Lah-tech"' oder "`Lej-tech"', kann auch
"`LaTeX"' geschrieben werden) ist ein auf \TeX\ auf\/bauendes 
Computerprogramm und wurde von Leslie Lamport~\cite{manual,wonne} 
geschrieben.  Es vereinfacht den Umgang mit \TeX, indem es 
entsprechend der logischen Struktur des Dokuments auf vorgefertigte
Layout-Elemente zurückgreift.

\subsubsection{\LaTeXe}

\LaTeXe\ (sprich "`\LaTeX\ zwei e"') ist die aktuelle Variante von
\LaTeX\ seit dem 1.~Juni 1994.  (Die vorherige hieß \LaTeX~2.09.)
Wenn hier von \LaTeX\ gesprochen wird, so ist normalerweise dieses
\LaTeXe{} gemeint.

Neue Versionen
von \LaTeXe{} (z.\,B. mit Fehlerberichtigungen oder Ergänzungen)
erscheinen jährlich im Juni; die vorliegende Beschreibung setzt 
mindestens diejenige vom Juni 2001 voraus.  

\subsection{Grundkonzept}
 
\subsubsection{Autor, Designer und Setzer}
 
Für eine Publikation übergab der Autor dem Verleger
traditionell  ein maschinengeschriebenes Manuskript.  Der
Buch-Designer des Verlages entschied dann über das Layout des
Schriftstücks (Länge einer Zeile, Schriftart, Abstände vor
und nach Kapiteln usw.\@) und schrieb dem Setzer die
dafür notwendigen Anweisungen dazu.
\LaTeX{} ist in diesem Sinne der Buch-Designer, 
das Programm \TeX{} ist sein Setzer.
 
Ein menschlicher Buch-Designer erkennt die Absichten des Autors
(z.\,B.\ Kapitel-Überschriften, Zitate, Beispiele, Formeln
\dots) meistens aufgrund seines Fachwissens aus dem Inhalt des
Manuskripts.  \LaTeX{} dagegen ist "`nur"' ein Programm und
benötigt daher zusätzliche Informationen vom Autor, die die
logische Struktur des Textes beschreiben.
Diese Informationen werden in Form von sogenannten "`Befehlen"'
innerhalb des Textes angegeben.
Der Autor braucht sich also
(weitgehend) nur um die logische Struktur seines Werkes zu kümmern,
nicht um die Details von Gestaltung und Satz.
 
Im Gegensatz dazu steht der visuell orientierte Entwurf eines
Schriftstückes mit Textverarbeitungs- oder \textsc{dtp}-Programmen wie z.\,B.\ 
\textsc{Word}.
In diesem Fall legt der Autor das Layout des Textes gleich bei der
interaktiven Eingabe fest. Dabei sieht er am Bildschirm das, was
auch auf der gedruckten Seite stehen wird. Solche Systeme, die das
visuelle Entwerfen unterstützen, werden auch \textsc{wysiwyg}-Systeme
("`what you see is what you get"') genannt.
 
Bei \LaTeX{} sieht der Autor beim Schreiben des Eingabefiles in
der Regel noch nicht sofort, wie der Text nach dem Formatieren 
aussehen wird. Er kann aber %durch Aufruf des entsprechenden Programms 
jederzeit einen "`Probe-Ausdruck"' seines Schriftstücks auf dem
Bildschirm machen und danach sein Eingabefile entsprechend 
korrigieren und die Arbeit fortsetzen.
 
 
\subsubsection{Layout-Design}
 
Typographisches Design ist ein Handwerk, das erlernt werden muss.
Ungeübte Autoren machen dabei oft gravierende Fehler.
Fälschlicherweise glauben viele Laien, dass Textdesign
vor allem eine Frage der Ästhetik ist -- wenn das
Schriftstück vom künstlerischen Standpunkt aus "`schön"'
aussieht, dann ist es schon gut "`designed"'.
Da Schriftstücke jedoch gelesen und nicht in einem Museum
aufgehängt werden, sind die leichtere Lesbarkeit und bessere
Verständlichkeit wichtiger als das schöne Aussehen.
 
Beispiele:
Die Schriftgröße und Numerierung von Überschriften soll so
gewählt werden, dass die Struktur der Kapitel und Unterkapitel
klar erkennbar ist.
Die Zeilenlänge soll so gewählt werden, dass anstrengende
Augenbewegungen des Lesers vermieden werden, nicht so, dass der
Text das Papier möglichst schön ausfüllt.
 
Mit interaktiven visuellen Entwurfssystemen ist es leicht,  
Schriftstücke zu erzeugen, die zwar "`gut"' aussehen,
aber ihren Inhalt und dessen Aufbau nur mangelhaft wiedergeben.
\LaTeX{} verhindert solche
Fehler, indem es den Autor dazu zwingt, die logische
Struktur des Textes anzugeben, und dann automatisch ein dafür
geeignetes Layout verwendet.

Daraus ergibt sich, dass \LaTeX{} insbesondere für  Dokumente geeignet 
ist, wo vorgegebene Gestaltungsprinzipien auf sich wiederholende
logische Textstrukturen angewandt werden sollen. 
Für das -- notwendigerweise -- visuell orientierte Gestalten
etwa eines Plakates ist \LaTeX{} hingegen 
aufgrund seiner Arbeitsweise weniger geeignet.

\subsubsection{Vor- und Nachteile}

Gegenüber anderen Textverarbeitungs- oder \textsc{dtp}-Programmen 
zeichnet sich \LaTeX{}
vor allem durch die folgenden Vorteile aus:
\begin{itemize}
\item Der Anwender muss nur wenige, leicht verständliche Befehle
  angeben, die die logische Struktur des Schriftstücks
  betreffen, und braucht sich um die gestalterischen Details
  (fast) nicht zu kümmern.
\item Das Setzen von mathematischen Formeln ist besonders gut
  unterstützt.
\item Auch anspruchsvolle Strukturen wie Fußnoten, Literaturverzeichnisse,
  Tabellen u.\,v.\,a.\  können mit wenig Aufwand erzeugt werden.
% ---- schwammige Formulierung ;-)
\item Routineaufgaben wie das Aktualisieren von Querverweisen
 oder das Erstellen des Inhaltsverzeichnisses 
 werden automatisch erledigt.
\item Es stehen zahlreiche vordefinierte Layouts zur Verfügung.
\item \LaTeX-Dokumente sind zwischen verschiedenen Installationen und
 Rechnerplattformen austauschbar.
\item Im Gegensatz zu vielen \textsc{wysiwyg}-Programmen bearbeitet \LaTeX{} auch
  lange oder komplizierte Dokumente zuverlässig,
  und sein Ressourcenverbrauch (Speicher, Rechenleistung) ist vergleichsweise
  mäßig.
\end{itemize}
Ein Nachteil soll freilich auch nicht verschwiegen werden:
\begin{itemize}
\item Innerhalb der von \LaTeX\ unterstützten Dokument"=Layouts
  können zwar einzelne Parameter leicht variiert werden,
  grundlegende Abweichungen von den Vorgaben sind
  aber nur mit größerem Aufwand möglich (Design einer
  neuen Dokumentklasse, siehe~\cite{clsguide,lay,lay2,typografie}.)
\end{itemize}

\subsubsection{Der Arbeitsablauf}
Der typische Ablauf beim Arbeiten mit \LaTeX{} ist:
\begin{enumerate}
  \item Ein Eingabefile schreiben, das den Text und die \LaTeX-Befehle 
  enthält.
  \item Dieses File mit \LaTeX{} bearbeiten; dabei wird eine Datei
  erzeugt, die den gesetzten Text in einem geräteunabhängigen Format
  (\textsc{dvi}, \textsc{pdf} oder auch PostScript) enthält.
  \item Einen "`Probeausdruck"' davon auf dem Bildschirm anzeigen (Preview).
  \item Wenn nötig, die Eingabe korrigieren und zurück zu Schritt~2.
  \item Die Ausgabedatei drucken.
\end{enumerate}
Zeitgemäße Betriebssysteme machen es möglich, dass der Texteditor
und das Preview-Programm gleichzeitig in verschiedenen Fenstern 
"`geöffnet"' sind; beim Durchlaufen des obigen Zyklus brauchen sie 
also nicht immer wieder von neuem gestartet werden.  Nur die 
wiederholte \LaTeX-Bearbeitung des Textes muss noch von Hand 
angestoßen werden und läuft ebenfalls in einem eigenen Fenster ab.
% Danach sollte das Preview-Programm -- im 
% Idealfall -- selbsttätig das veränderte Ergebnis anzeigen; ansonsten 
% kennt es normalerweise einen Menüpunkt oder eine Schaltfläche, um das 
% geänderte Ausgabefile erneut zu laden und anzuzeigen.

Wie man auf die einzelnen Programme -- Editor, \LaTeX, Previewer,
Druckertreiber -- in einer bestimmten
Betriebssystemumgebung zugreift, muss in einem \local{}
beschrieben sein.



\section{Eingabefile}
Das Eingabefile für \LaTeX{} ist ein Textfile. Es wird mit einem
Editor erstellt und enthält sowohl den Text, der gedruckt
werden soll, als auch die Befehle, aus denen \LaTeX\ erfährt,
wie der Text gesetzt werden soll.


\subsection{Leerstellen}
 
"`Unsichtbare"' Zeichen wie das Leerzeichen, Tabulatoren
und das Zeilenende werden von \LaTeX{}
einheitlich als Leerzeichen behandelt.  \emph{Mehrere}
Leerzeichen werden wie \emph{ein} Leerzeichen behandelt.   
Wenn man andere als die normalen Wort- und Zeilenabstände
will, kann man dies also nicht durch die Eingabe von
zusätzlichen Leerzeichen oder Leerzeilen erreichen, sondern
nur mit entprechenden \LaTeX-Befehlen.

Eine Leerzeile zwischen Textzeilen bedeutet das Ende eines 
Absatzes.  \emph{Mehrere} Leerzeilen werden wie \emph{eine}
Leerzeile behandelt.
 
 
\subsection{\LaTeX-Befehle und Gruppen}
 
Die meisten \LaTeX-Befehle haben eines der beiden folgenden
Formate: Entweder sie beginnen mit einem Backslash~(\verb|\|)
und haben dann einen nur aus Buchstaben bestehenden Namen, der
durch ein oder mehrere Leerzeichen oder durch ein nachfolgendes
Sonderzeichen oder eine Ziffer beendet wird; oder sie bestehen
aus einem Backslash und genau einem Sonderzeichen oder einer
Ziffer.
Groß- und Kleinbuchstaben haben auch in Befehlsnamen
\emph{verschiedene} Bedeutung.
Wenn man nach einem Befehlsnamen eine Leerstelle erhalten will,
muss man~\verb|{}| zur Beendigung des Befehlsnamens oder einen
eigenen Befehl für die Leerstelle verwenden.
\exa
  \renewcommand{\today}{35.~Mai 1998}  % to make sure that the
  % line breaks look good, regardless of the date of printing.
Heute ist der \today.
Oder: Heute ist der \today .
Falsch ist: Am \today regnet es.
Richtig: Am \today{} scheint die Sonne.
Oder: Am \today\ schneit es.
\exb
\begin{verbatim}
Heute ist der \today.
Oder: Heute ist der \today .
Falsch ist:
 Am \today regnet es.
Richtig:
 Am \today{} scheint die Sonne.
 Oder: Am \today\ schneit es.
\end{verbatim}
\exc
 
Manche Befehle haben Parameter, die zwischen geschwungenen
Klammern angegeben werden müssen.
Manche Befehle haben Parameter, die weggelassen oder zwischen
eckigen Klammern angegeben werden können.
Manche Befehle haben Varianten, die durch das Hinzufügen
eines Sterns an den Befehlsnamen unterschieden werden.

Geschwungene Klammern können auch dazu verwendet werden, Gruppen
(groups) zu bilden.
Die Wirkung von Befehlen, die innerhalb von Gruppen oder
Umgebungen (environments) angegeben werden, endet immer mit dem
Ende der Gruppe bzw.\ der Umgebung.  Im obigen Beispiel
ist~\verb|{}| eine leere Gruppe, die außer der Beendigung des
Befehlsnamens \texttt{today} keine Wirkung hat.
 
\subsection{Kommentare}
 
Alles, was hinter einem Prozentzeichen (\verb|%|)
steht (bis zum Ende der Eingabezeile), wird von \LaTeX\ 
ignoriert.
Dies kann für Notizen des Autors verwendet werden, die nicht
oder noch nicht ausgedruckt werden sollen.
\exa
Das ist ein % dummes
% Besser: ein lehrreiches <----
Beispiel.
\exb
\begin{verbatim}
Das ist ein % dummes
% Besser: ein lehrreiches <----
Beispiel.
\end{verbatim}
\exc
 
\subsection{Aufbau}
 
Der erste Befehl in einem \LaTeX-Eingabefile muss der Befehl
\begin{quote}
\verb|\documentclass|
\end{quote}
sein.
Er legt fest, welche Art von Schriftstück überhaupt erzeugt werden soll
(Bericht, Buch, Brief usw.).
Danach können weitere Befehle folgen, die für das gesamte
Dokument gelten sollen.  Dieser Teil des Dokuments wird auch als 
\emph{Vorspann} oder \emph{Präambel} bezeichnet.  Mit dem Befehl
\begin{quote}
\verb|\begin{document}|
\end{quote}
endet der Vorspann, und es 
beginnt das Setzen des Schriftstücks.
Nun folgen der Text und alle \LaTeX-Befehle, die das Ausdrucken
des Schriftstücks bewirken.
Die Eingabe muss mit dem Befehl
\begin{quote}
\verb|\end{document}|
\end{quote}
beendet werden.
Falls nach diesem Befehl noch Eingaben folgen, werden sie von
\LaTeX\ ignoriert.
 
Abbildung~\ref{mini} zeigt ein \emph{minimales} \LaTeX-File.
Ein etwas komplizierteres File ist in Abbildung~\ref{dokument}
skizziert.
 
\begin{figure}[hbp] %\small
\oben{6cm}
\begin{verbatim}
\documentclass{article}
\begin{document}
Small is beautiful.
\end{document}
\end{verbatim}
\unten
\caption{Ein minimales \LaTeX-File} \label{mini}
\end{figure}

% im folgenden Beispiel sollten Umlaute im Eingabefile auftreten!
\begin{figure}[hbtp] %\small
\oben{10cm}
\begin{flushleft}\ttfamily
\verb+\documentclass[11pt,a4paper]{article}+\\
\verb+\usepackage[latin1]{inputenc}+\\
\verb+\usepackage{ngerman}+\\
\verb+\date{29. Februar 1998}+\\
\verb+\author{H.~Partl}+\\
\verb+\title{+"Uber kurz oder lang\verb+}+\\
\ \\
\verb+\begin{document}+\\
\verb+\maketitle+\\
\verb+\begin{abstract}+\\
Beispiel f"ur einen wissenschaftlichen Artikel\\
in deutscher Sprache.\\
\verb+\end{abstract}+\\
\verb+\tableofcontents+\\
\ \\
\verb+\section{Start}+\\
\ \\
Hier beginnt mein schönes Werk \dots\\
\ \\
\verb+\section{Ende}+\\
\ \\
\dots\ und hier endet es.\\
\ \\
\verb+\end{document}+\\[1\baselineskip]
\end{flushleft}
\unten
\caption{Aufbau eines Artikels} \label{dokument}
\end{figure}
 
 
\subsection{Dokumentklassen}\label{docsty}
 
Die am Beginn des Eingabefiles  mit
\begin{verse}
\verb|\documentclass[|\textit{optionen}\verb|]{|%
  \textit{klasse}\verb|}|
\end{verse}
definierte "`Klasse"' eines Dokumentes enthält 
Vereinbarungen über 
das Layout und die logischen Strukturen, z.\,B.\ die 
Gliederungseinheiten (Kapitel etc.\@), 
die für alle Dokumente dieses Typs gemeinsam sind.

Zwischen den geschwungenen Klammern \emph{muss} genau eine Dokumentklasse
angegeben werden.  Tabelle~\ref{docstyles} auf S.~\pageref{docstyles}
führt Klassen auf,
die in jeder \LaTeX-Installation existieren sollten.  
Im \local\ können weitere verfügbare 
Klassen angegeben sein.  
 
Zwischen den eckigen Klammern \emph{können}, durch Kommata getrennt, 
eine oder mehrere Optionen für das Klassenlayout
angegeben werden. Die wichtigsten Optionen sind in der 
Tabelle~\ref{options} auf S.~\pageref{options} angeführt.
Das Eingabefile für diese Beschreibung beginnt z.\,B.\ mit:
\begin{verse}
\verb|\documentclass[11pt,a4paper]{article}|
\end{verse}

\begin{table}[hbpt]
\caption{Dokumentklassen} \label{docstyles}
\oben{11cm}
\begin{ttdescription}%\small
\item [article] für Artikel in wissenschaftlichen Zeitschriften,
  kürzere Berichte u.\,v.\,a.
 
\item [report] für längere Berichte, die aus mehreren Kapiteln
  bestehen, Diplomarbeiten, Dissertationen u.\,ä.
 
\item [book] für Bücher

\item[scrartcl, scrreprt, scrbook]\quad Die sog. KOMA-Klassen 
sind Varianten der o.\,g. Klassen
mit besserer Anpassung an DIN-Papierformate und "`europäische"'
Typographie. 
(Nicht überall vorhanden, siehe \local.)

% \item [proc] für Konferenzbände (Proceedings)
% ist nicht einmal im LaTeX-Handbich beschrieben!

\item [letter] für Briefe (siehe auch Abschnitt~\ref{briefe})

\item [foils] für Folien oder Präsentationen.
(Nicht überall vorhanden, siehe \local.)
  
\end{ttdescription}
\unten
\end{table}

\begin{table}[hbpt]
\caption[Klassenoptionen]{Klassenoptionen (Alternativen sind durch \texttt{|}
  getrennt)} \label{options}
\oben{11cm}
\begin{ttdescription}%\small
\item [10pt|11pt|12pt] wählt die normale Schriftgröße des Dokuments aus.
  10\,pt hohe Schrift ist die Voreinstellung; diese Beschreibung benutzt 11\,pt.

\item[a4paper] für Papier im DIN\,A4-Format. Ohne diese
  Option nimmt \LaTeX\ amerikanisches Papierformat an.
 
\item [fleqn] für linksbündige statt zentrierte mathematische
  Gleichungen
 
\item [leqno] für Gleichungsnummern links statt rechts von jeder
  numerierten Gleichung
 
\item [titlepage|notitlepage] legt fest, ob Titel und Zusammenfassung
  auf einer eigenen Seite erscheinen sollen.  \texttt{titlepage} ist
  die Voreinstellung für die Klassen \texttt{report} und \texttt{book}.
 
\item [onecolumn|twocolumn] für ein- oder zweispaltigen Satz.
 Die Voreinstellung ist immer \texttt{onecolumn}.  
 Die Klassen \texttt{letter} und \texttt{slides} kennen \emph{keinen}
 zweispaltigen Satz.
 
\item [oneside|twoside] legt fest, ob die Seiten für ein- oder
  zweiseitigen  Druck gestaltet werden sollen.  
  \texttt{oneside} ist die Voreinstellung für
  alle Klassen außer \texttt{book}.
  
\end{ttdescription}
\unten
\end{table}



\subsection{Pakete}\label{packages}
 
Mit dem Befehl
\begin{verse}
\verb:\usepackage[:\textit{optionen}\verb:]{:%
  \textit{pakete}\verb:}:
\end{verse}
können im Vorspann ergänzende Makropakete (packages) geladen werden,
die das Layout der Dokumentklasse
modifizieren oder zusätzliche Funktionalität bereitstellen.
Eine Auswahl von Paketen findet sich in der Tabelle~\ref{pack} 
auf S.~\pageref{pack}.
Das Eingabefile für diese Beschreibung enthält beispielsweise:
\begin{verse}
\verb|\usepackage{german,latexsym,alltt,|\\
\verb|            graphicx,textcomp,hyperref}|
\end{verse}


\begin{table}[htbp]
\caption{Pakete (eine Auswahl)}\label{pack}
\oben{11cm}
\begin{ttdescription}%\small
\setlength{\itemsep}{.5\itemsep plus1pt minus1pt}
\item[alltt] Definiert eine Variante der \texttt{verbatim}-Umgebung
\item[amsmath, amssymb] Mathematischer Formelsatz mit erweiterten Fähigkeiten,
  zusätzliche mathematische Schriften und Symbole; Beschreibung siehe
  \cite{ch8}.
%\item[array] Verbesserte und erweiterte Versionen der Umgebungen
%  \texttt{array}, \texttt{tabular} und \texttt{tabular*}.
\item[babel] Anpassungen für viele verschiedene Sprachen. Die
  gewählten Sprachen werden als Optionen angegeben.
\item[color] Unterstützung für Farbausgabe;
  Beschreibung  siehe~\cite{grfguide} und~\cite{grfcomp}.
\item[dcolumn] Unterstützt auf Dezimaltrennzeichen ausgerichtete
  Spalten in den Umgebungen \texttt{array} und \texttt{tabular}
\item[fontenc] Erlaubt die Verwendung von Schriften mit
  unterschiedlicher Kodierung (Zeichenvorrat, Anordnung).
\item[fancyhdr] Flexible Gestaltung von Kopf- und Fußzeilen.
\item[geometry] Manipulation des Seitenlayouts.
\item[german, ngerman] Anpassungen für die deutsche Sprache in
  traditioneller und neuer Rechtschreibung.
\item[graphicx] Einbindung von extern erzeugten Graphiken.
  Die umfangreichen Möglichkeiten dieses Pakets werden 
  in~\cite{grfguide} und~\cite{grfcomp} beschrieben.
\item[hyperref] Ermöglicht Hyperlinks zwischen Textstellen und zu
  externen Dokumenten; besonders sinnvoll einsetzbar, 
  wenn mit \TeX\ eine Ausgabedatei im \textsc{pdf}- oder \textsc{html}-Format 
  erzeugt wird.
\item[inputenc] Deklaration der Zeichenkodierung im
  Eingabefile.
\item[latexsym] Erlaubt einige besondere Symbole wie~\(\Box\),
  die mit \LaTeX~2.09 standardmäßig verfügbar waren.
\item[longtable]
  für Tabellen über mehrere Seiten mit automatischem Seitenumbruch.
\item[makeidx] Unterstützt das Erstellen eines Index.
\item[multicol] Mehrspaltiger Satz mit Kolumnenausgleich.
%\item[showkeys] Druckt die Namen aller verwendeten \verb:\label:s,
%  \verb:\ref:s und \verb:\pageref:s im Text aus.
%\item[tabularx] für Tabellen mit automatisch an den vorhandenen
%  Platz angepasster Breite der Spalten.
\item[textcomp] Bindet Schriften mit zusätzlichen Textsymbolen ein.
%\item[verbatim] Flexible Erweiterung der \texttt{verbatim}-Umgebung.
\end{ttdescription}
\unten
\end{table}


\subsection{Eingabezeichensatz}\label{inputenc}

Bei jedem \LaTeX-System dürfen mindestens die folgenden
Zeichen zur Eingabe von Text verwendet werden:
\begin{quote}
  \ttfamily
  a\dots z A\dots Z 0\dots 9 \\
  . : ; , ? ! ` ' ( ) [ ] - / * @ + =
\end{quote}
Die folgenden Eingabezeichen haben für \LaTeX{} eine Spezialbedeutung
oder sind nur innerhalb von mathematischen Formeln erlaubt:
\begin{quote}
\verb.$ & % # _ { }  ~  ^  "  \  | < >.
\end{quote}
Für Zeichen, die über obige Liste hinausgehen, beispielsweise die Umlaute,
sind je nach Betriebssystem des verwendeten Computers 
unterschiedliche Kodierungen in Gebrauch.  Damit auch diese Zeichen im 
Eingabefile benutzt werden dürfen,  muss man das Paket 
\texttt{inputenc} laden und dabei die jeweilige Kodierung als 
Option angeben: \verb:\usepackage[:\textit{codepage}\verb:]{inputenc}:.
Mögliche Angaben für \textit{codepage} sind u.\,a.:
\begin{ttdescription}
  \item[latin1] Latin-1 (ISO~8859-1), gebräuchlich unter \textsc{Unix} und VMS
  \item[latin9] Latin-9 (ISO~8859-15), Erweiterung von Latin-1, u.\,a. mit Eurozeichen
  \item[ansinew] Microsoft Codepage 1252 für Windows
  \item[cp850] IBM Codepage 850, üblich unter OS/2
  \item[applemac] \textsc{Macintosh}-Kodierung
\end{ttdescription}
Falls \LaTeX{} ein eingegebenes Zeichen nicht darstellen
kann, was meist für die sogenannten "`Pseudografik-Zeichen"' 
gilt,  bekommt man eine entsprechende Fehlermeldung.
Auch sind manche Zeichen nur im Text, andere nur in mathematischen 
Formeln erlaubt.

Man beachte, dass der in der \emph{Ausgabe} darstellbare Zeichenvorrat 
von \LaTeX{} nicht davon abhängt, welche Zeichen als \emph{Eingabe} erlaubt 
sind:
Für jedes überhaupt darstellbare Zeichen -- also auch diejenigen, die
nicht im Zeichensatz des jeweiligen Betriebssystems enthalten sind --
gibt es einen 
\LaTeX-Befehl oder eine Ersatzdarstellung, die ausschließlich mit 
ASCII-Zeichen auskommt.  Näheres darüber erfahren Sie
in Abschnitt \ref{spezial}.



\endinput
